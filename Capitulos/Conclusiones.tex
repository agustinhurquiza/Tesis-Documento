\chapter{Conclusiones y trabajo futuro} \label{cap:conclusiones}

\section{Conclusiones y aportes} \label{sec:conclusionesyaportes}
Durante este trabajo se analizó de forma detallada y objetiva el desafiante problema de ZSD. Desde un principio, sabíamos que era un campo de investigación nuevo y que esto dificultaría el desarrollo de esta tesis. Los objetivos se fueron modificando en transcurso del tiempo, pero aún así logramos genera un aporte en esta disciplina.\\

El primer paso de esta tesis fue la lectura y análisis de los distintos trabajos sobre ZSD. Un aspecto que tenían en común la mayoría es la utilización de incrustaciones visuales y semánticas para abordar el problema. Por este motivo decidimos utilizar esta metodología, para proponer un modelo base, basándonos en el trabajo de Bansal \etal~\cite{bansal2018zero}. La falta de una implementación por parte de este artículo, nos obligo a profundizar en cada etapa del desarrollo, reduciendo las metas planteadas, pero aportando un conocimiento mas detallado de la solución. 

Si bien el modelo de base no fue propuesto por nosotros, aportamos detalles que surgieron de nuestra experimentación y comprendimos como estos afectan a los resultados. También se analizaron aspectos como los generadores de propuestas y las CNN, que al nuestro entender, fueron ignorados por el trabajo original pero resulta cruciales.

Otro aspecto importante analizado son los conjuntos de datos. Aún no existe uno específico para el problema de ZSD, ni una adaptación consensuada de alguno ya existente. Proponemos una manera sencilla de dividir COCO y la comparamos con la división del trabajo original que al parecer beneficia al modelo considerablemente.

Pero lo que creemos que es el mayor aporte, es el análisis de la métricas. Esto es una gran debilidad en los trabajos relacionados actuales, ya que impide una comparación justa y correcta. 

Debido a que los resultados obtenidos no eran los esperados, decidimos investigar sobre las métricas, y nos encontramos con una definición poco específica que genero una mala interpretación. Para resolver esto, se analizaron distintos artículos que señalaban el problema que tuvimos. Pero el trabajo de Padilla \etal~\cite{padilla2020survey} sobresale a los demás, porque pose una clara definición de las métricas y una implementación fácil de utilizar, por lo que se recomida su uso.

Si bien los resultados no representan una mejora respecto al estado del arte, aportan una idea de lo que los modelos de ZSD son capaces de alcanzar. Por otro lado, creemos que este trabajo aporta resultados mas transparente y detallados, haciendo posible agregar mejoras y ver su progreso de una manera cuantitativa.


\section{Trabajo futuro} \label{sec:trabajo futuro}

Teniendo en cuenta los resultados obtenidos en esta tesis, existen distintas alternativas para seguir profundizando. Las cual podemos dividir en tres grupos:

\begin{itemize}
	\item Mejorar el algoritmo que genera propuestas, que afecta sobre todo la etapa de evaluacion. Trabajos actuales utilizan varios generadores simultáneamente, obteniendo algunas mejoras. Otros plantean aumentar el número de propuestas considerablemente y utilizan un criterio mas complejo para eliminar casillas repetidas y de fondo.
	\item Mejorar del modelo propuesto. Existe muchas formas, algunas ideas pueden ser: considerar la fusión de diferentes vectores de palabras (\textit{Word2vec} y \textit{GloVe}); utilizar otro espacio que no sea el semántico y mapear ambas caracteristicas a este; o utilizar una única red unificada entrenada de extremo a extremo, capaz de predecir la ubicación de diferentes objetos y clasificarlos como lo hace \textit{Faster R-CNN}.
	\item Resulta interesante suavizar el problema de ZSD, y en ves de clasificar por clase, se pueden utilizar sub-clases. Si bien esto no es una mejora, puede ayudar a entender si el modelo realmente esta relacionando objetos, ya que no es lo mismo confundir un perro con un auto, que un perro con lobo.
	\item Una debilidad del modelo actual es la forma de calcular el valor de confianza asociado a una predicción. Se puede analizar una manera mas compleja para realizar esto, como por ejemplo agregar al calculo la similitud cosenos con las demás clases y no solo la que obtenga mayor puntaje.
\end{itemize}

