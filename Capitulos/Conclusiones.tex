\chapter{Conclusiones y trabajo futuro} \label{cap:conclusiones}

\section{Conclusiones y aportes} \label{sec:conclusionesyaportes}
Durante este trabajo se analizo de forma detallada y objetiva el desafiante problema de ZSD. Desde un principio, sabíamos que era un campo de investigación nuevo y que esto dificultaría el desarrollo de esta tesis. Los objetivos se fueron modificando en transcurso del tiempo. Aun así logramos genera un aporte en esta disciplina.\\

El primer paso de esta tesis fue la lectura y análisis de los distintos trabajos sobre ZSD. Un aspecto, que tenían la mayoría, es la utilización de incrustaciones visual y semánticas para abordar el problema. Por este motivo decidimos utilizar esta metodología, para proponer un modelo base, basándonos en el trabajo de Bansal \cite{bansal2018zero}. La falta de una implementación, nos obligo a profundizar en cada etapa del desarrollo, reduciendo las metas planteadas. Pero aportando un conocimiento mas detallado de la solución. 

Si bien el modelo base no fue propuesto por nosotros, aportamos detalles que surgieron de nuestra experimentación y como estos afectan a los resultados. También, se analizaron aspectos que al nuestro entender fueron ignorados por el trabajo original, pero resulta cruciales.

Otro aspecto importante analizado son los conjuntos de datos. Aun no existe uno especificado en el problema de ZSD, ni una adaptación consensuada de alguno ya existente. Proponemos una manera sencilla de dividir MSCOCO y la comparamos con la división del trabajo original que al parecer beneficia al modelo considerablemente.

Pero lo que creemos el mayor aporte es el análisis de la métricas. Esto es una gran debilidad en los trabajos relacionados actuales, que impide una comparación, justa y correcta. Debido a que los resultados no eran los esperados y luego de probar todos los cambios y mejoras posibles, nos encontramos con una definición ambigua que genero una mala interpretación. Decidimos investigar sobre las métricas y encontramos muchos documentos que señalaban el problema que tuvimos. Pero el trabajo de Padilla \cite{padilla2020survey} sobre sale a los demás, posea una clara definición de las métricas y una implementación fácil de utilizar, y se recomida su uso.

Si bien los resultados obtenidos no son los mejores y están por abajo de la expectativa. Aportan una idea de lo que son capaces los modelos de ZSD. Creemos que son mas transparente y detallados al trabajo original, haciendo posible agregar mejoras y ver su progreso de una manera cuantitativa.

Por ultimo, la comparación con modelos mas actuales, hace evidente el constante esfuerzo en este campo y sus mejoras demuestra que ZSD no es una fantasía.


\section{Trabajo futuro} \label{sec:trabajo futuro}

Teniendo en cuenta los resultados obtenidos en esta tesis, existen distintas alternativas para seguir profundizando. Las cual podemos dividir en tres grupos. 

El primero, es mejorar el algoritmo que genera propuestas, esto afecta sobre todo a la etapa de evaluacion. Trabajos actuales utilizan varios generadores simultáneamente, obteniendo algunas mejoras. Otros plantean aumentar el numero de propuestas considerablemente y utilizar un criterio  mas complejo a la supresión no máxima, para eliminar casillas repetidas y de fondo.

El segundo, surge de la simplicidad del modelo propuesto. Existe muchas formas de mejorarlo, algunas ideas pueden ser. Considerar la fusión de diferentes vectores de palabras (\textbf{Word2vec} y \textbf{GloVe}). Utilizar otro espacio que no sea el semantico y mapear ambas caracteristicas a este. Otro cambio, que tambien afecta el punto anterior es utilizar una única red unificada de extremo a extremo, capas de predecir  la ubicación de diferentes objetos y clasificarlos como lo es \textbf{Faster R-CNN}.

Por ultimo, resulta interesante suavizar el problema de ZSD. En ves de clasificar por clase, se pude utilizar sub-clases. Si bien esto no es una mejora, puede ayudar a entender si el modelo realmente esta relacionando objetos, ya que no es lo mismo confundir un perro con un auto que con lobo.