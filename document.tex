%Plantilla basada en "Template for Masters / Doctoral Thesis" (plantilla disponible en writeLaTex) que subió LaTeXTemplates.com

\documentclass[12pt,twosided]{book}
\usepackage[paperwidth=17cm, paperheight=22.5cm, bottom=2.5cm, right=2.5cm]{geometry}
\usepackage{amssymb,amsmath,amsthm} %paquete para símbolo matemáticos
\usepackage[spanish]{babel}
\usepackage[utf8]{inputenc} %Paquete para escribir acentos y otros símbolos directamente
\usepackage{enumerate}
\usepackage{graphicx}
\usepackage{subcaption}
\usepackage{float}
\usepackage{tikz-qtree,tikz-qtree-compat}
\usepackage{multicol}
\usepackage{algpseudocode}
%\usepackage{subfig} %para poner subfiguras
\graphicspath{{Img/}} %En qué carpeta están las imágenes
\usepackage[square,sort,comma,numbers]{natbib}

\usepackage[nottoc]{tocbibind}
\usepackage[pdftex,
            pdfauthor={Agustin Horacio Urquiza Toledo},
            pdftitle={Zero-Shot Object Detection},
            pdfsubject={Ciencias de la computación},
            pdfkeywords={PALABRAS CLAVE},
            pdfproducer={Latex con hyperref},
            pdfcreator={pdflatex}]{hyperref}

\usepackage[colorinlistoftodos]{todonotes}
\newcommand{\todos}[1]{{\color{red}[TODO: #1]}}
\begin{document}

%----------------------------------------------------------------------------------------
%	COMANDOS PERSONALIZADOS
%----------------------------------------------------------------------------------------

%SI TU TESIS TIENE TEOREMAS Y DEMOSTRACIONES, PUEDES DESCOMENTAR Y USAR LOS SIGUIENTES COMANDOS

%\renewcommand{\proofname}{Demostración}
%\providecommand{\norm}[1]{\lVert#1\rVert} %Provee el comando para producir una norma.
%\providecommand{\innp}[1]{\langle#1\rangle} 
%\newcommand{\seno}{\mathrm{sen}}
%\newcommand{\diff}{\mathrm{d}}

%\newtheorem{teo}{Teorema}[section] 
%\newtheorem{cor}[teo]{Corolario}
%\newtheorem{lem}[teo]{Lema}

%\theoremstyle{definition}
%\newtheorem{dfn}[teo]{Definición}

%\theoremstyle{remark}
%\newtheorem{obs}[teo]{Observación}

%\allowdisplaybreaks


%----------------------------------------------------------------------------------------
%	PORTADA
%----------------------------------------------------------------------------------------

\title{Framework para aprendizaje activo} %Con este nombre se guardará el proyecto en writeLaTex

\begin{titlepage}
\begin{center}

\textsc{\Large Facultad de Matemática, Astronomía, Física y Computación}\\[2em]

\textsc{Universidad Nacional de Córdoba}

%Figura
\begin{figure}[h]
\begin{center}
\includegraphics[scale=0.2]{img/unc.jpg}
\end{center}
\end{figure}

\vspace{1em}

\textsc{\huge \textbf{Zero-Shot Object Detection}}\\[2em]

\textsc{\large Tesis}\\[1em]

\textsc{que para obtener el título de}\\[1em]

\textsc{Licenciado en Ciencias de la Computación}\\[1em]

\textsc{presenta}\\[1em]

\textsc{\Large Agustin Horacio Urquiza Toledo}\\[1em]

\textsc{\large Director: Jorge Sanchez}

\end{center}

\vspace*{\fill}
\textsc{Córdoba, Argentina \hspace*{\fill} 2019}

\end{titlepage}


%----------------------------------------------------------------------------------------
%	DEDICATORIA
%----------------------------------------------------------------------------------------

\pagestyle{empty}
\frontmatter

\chapter*{}
\begin{flushright}
\textit{DEDICATORIA}
\end{flushright}


%----------------------------------------------------------------------------------------
%	AGRADECIMIENTOS
%----------------------------------------------------------------------------------------

\chapter*{Agradecimientos}
%\markboth{AGRADECIMIENTOS23}{AGRADECIMIENTOS} % encabezado 
Muchas gracias, Muchas gracias, Muchas gracias, Muchas gracias,Muchas gracias,Muchas gracias, Muchas gracias. Son muchas gracias?



%----------------------------------------------------------------------------------------
%	PREFACIO
%----------------------------------------------------------------------------------------

\chapter*{Resumen}

\pagestyle{plain}
Pasado el año 2000 se dan dos hechos que harán que las imágenes en Internet de un gran salto. Por un lado se empiezan a popularizar las cámaras digitales y por otro lado las conexiones de Internet subieron su velocidad. Esto genero la necesidad de crear métodos veloces y eficaces que faciliten la extracción de información en este tipo de datos. Luego, a partir del año 2010 con la  ``Revolución'' del Aprendizaje profundo, surgieron una gran cantidad de métodos para realizar esta tarea, entre ellos los Detectores. Pero esto genero la necesidad de tener una gran cantidad de imágenes anotadas, que en algunos casos no resulta viable. \textbf{Zero-shot Object Detection} intenta atacar este problema. En este artículo, abordamos este desafiante problema utilizando características visuales y descripciones semánticas, que tiene como objetivo detectar y reconocer simultáneamente instancias de conceptos novedosos. Para esto analizamos distintos trabajos y llevaremos a cabo experimentos, que aportaran una noción del estado actual de esta area.
%----------------------------------------------------------------------------------------
%	TABLA DE CONTENIDOS
%---------------------------------------------------------------------------------------

\begingroup
\hypersetup{hidelinks}
\tableofcontents
\endgroup


%----------------------------------------------------------------------------------------
%	TESIS
%----------------------------------------------------------------------------------------
\mainmatter %empieza la numeración de las páginas
\pagestyle{headings}

%  Incluye los capítulos en el folder de capítulos

\chapter{Introducción y Motivación} \label{cap:intro}

\section{Historia} \label{sec:historia}
La detección de objetos es una de las áreas de la visión por computadora que está creciendo más rápidamente. Gracias al aprendizaje profundo, cada año, los nuevos algoritmos/modelos siguen superando a los anteriores. Aunque la visión por computadora recientemente tomó gran importancia (el momento decisivo ocurrió en 2012 cuando AlexNet ganó ImageNet), ciertamente no es un nuevo campo científico.\\

Uno de los artículos más influyentes en Visión Informática fue publicado por dos neurofisiólogos, David Hubel y Torsten Wiesel~\cite{hubel1959receptive}, en 1959. Su publicación, titulada \textit{``Receptive fields of single neurons in the cat’s striate cortex''}, en español ``Campos receptivos de neuronas individuales en la corteza estriada del gato'', describió las propiedades de respuesta central de las neuronas corticales visuales y como la experiencia visual de un gato moldea su arquitectura cortical. Los investigadores establecieron a través de su experimentación (\autoref{fig:ExpermentoHubelTorsten}) que existen neuronas simples y complejas en la corteza visual primaria, y que el procesamiento visual siempre comienza con estructuras simples como los bordes orientados y gradualmente identifica estructuras mas complejas. En la actualidad, este es el principio básico detrás del aprendizaje profundo.\\

\begin{figure}
	\centering
	\includegraphics[width=0.5\textwidth]{img/cat.jpg}
	\caption{Simple explicación del experimento realizado por David Hubel y Torsten Wiesel}
	\label{fig:ExpermentoHubelTorsten}
\end{figure}

Otro echo importante en la historia de la visión por computadora fue en 1957, Russell Kirsch y sus colegas desarrollaron un aparato que permitía transformar imágenes en cuadrículas de números que las máquinas de lenguaje binario podían entender. 

Poco tiempo después, en la década de 1960 fue cuando la IA se convirtió en una disciplina académica y algunos de los investigadores eran extremadamente optimistas sobre el futuro del campo. En este periodo, Seymour Papert, profesor del laboratorio de IA del MIT, decidió lanzar el Proyecto de Verano y resolver, en pocos meses, el problema de la visión artificial. Los estudiantes debían diseñar una plataforma que pudiera realizar automáticamente segmentación de fondo y extraer objetos no superpuestos de imágenes del mundo real. Claro esta que el proyecto no fue un éxito.  Hoy en día, cincuenta años después, todavía no se ha podido resolver la visión por computadora. Sin embargo, ese proyecto fue el nacimiento oficial de esta disciplina como campo científico. 

Los aportes mas influyentes en este campo empezaron a surgir a partir de los años 2000. En 2001 Paul Viola y Michael Jones~\cite{viola2001rapid} presentaron el primer detector de rostros que funcionó en tiempo real. Aunque no se basaba en el aprendizaje profundo, el algoritmo tenía una relación con éste, ya que, al procesar imágenes aprendió qué características podrían ayudar a localizar caras, inspirándose en el experimento de David Hubel y Torsten Wiesel. 

En 2006 comenzó la competencia de Pascal VOC que permitió evaluar el desempeño de diferentes métodos para el reconocimiento de objetos. Mas tarde, en 2010, siguiendo los pasos de Pascal VOC, se inició el concurso de reconocimiento visual a gran escala ImageNet (ILSVRC), cuya tasa de error durante 2010 y 2011, en el desafió de clasificación de imágenes, rondaba el 26\%.  En 2012, un equipo de la Universidad de Toronto ingresó a la competencia con un modelo de red neuronal convolucional (AlexNet)~\cite{krizhevsky2012imagenet} que cambió todo, dado que logró una tasa de error del 16,4\%. En los años siguientes, las tasas de error en la clasificación de imágenes en ILSVRC cayeron a un pequeño porcentaje, como se observa en la \autoref{fig:EvolucionILSVRC} y los ganadores, desde 2012, siempre han sido redes neuronales convolucionales.

\begin{figure}[H]
	\centering
	\includegraphics[width=0.7\textwidth]{img/imgnet-grafico.png}
	\caption{Evolución de los modelos propuestos en la competencia ILSVRC}
	\label{fig:EvolucionILSVRC}
\end{figure}

\section{Detectores y ZSD} \label{sec:detectoresyzsd}
La detección de objetos es un subproblema de la visión artificial, que estudia cómo detectar la presencia de objetos en una imagen. Debido a la complejidad de poder detectar todas las instancias de todos los posibles objectos en una imagen, se dividió en distintas tareas para disminuir la dificultad. 

Par explicar los distintos problemas, es necesario distinguir dos conjuntos. Por un lado, los datos de entrenamiento, que consta de las imágenes que se usan para entrenar el modelo con sus respectivas etiquetas, es decir, que objetos se encuentran en la imagen, localización de los objetos, descripción de la imagen, o cualquier información extra que requiera la tarea. Por otro lado, las imágenes de prueba, que es el conjunto donde se observará o medirá la eficiencia del modelo ya entrenado. 

Supongamos que las etiquetas solo cuenta con dos tipos de información, que clase de objeto es, es decir si es un perro, auto, persona, etc. y su localización en la imagen. A todas las clases de objetos que aparecen en los datos de entrenamiento las llamaremos clases visibles o vistas, y todas aquellas clase que no sea una clase vista las llamearemos invisible o no vista. Dicho esto, los distintos problemas son:

\begin{itemize}
	\item \textbf{Clasificación}: consta de un modelo capás de predecir si una clase específica esta presente en una imagen. 
	\item \textbf{Clasificación mas localización}: además de poder clasificar tiene que ser capas de ubicar el objecto en la imagen.
	\item \textbf{Reconocimiento de imagen}: predice que objetos perteneciente a las clases visibles están presente en la imagen. 
	\item \textbf{La detección de objetos}: además de reconocer objetos visibles, tiene que ser capás de localizar dichos objetos. 
	\item \textbf{Reconocimiento por disparo cero}: tiene que poder reconocer clases vistas y no vistas.
	\item \textbf{Detección de objetos por disparo cero} (\textbf{ZSD} por sus siglas en inglés): debe localizar y clasificar todas las instancias de objetos en la imagen, sin depender si es una clase vista o no.
\end{itemize}

\begin{figure}[]
  \centering
  \subcaptionbox{\tiny{\textbf{ Clasificación}}}{\includegraphics[width=1.5in]{img/expect_3.png}}\hspace{1em}%
  \subcaptionbox{\tiny{\textbf{Detección de objetos}}}{\includegraphics[width=1.5in]{img/expect_2.png}}
  \subcaptionbox{\tiny{\textbf{ZSD}}}{\includegraphics[width=1.5in]{img/expect_1.png}}
  \caption{Ejemplo de tareas de clasificación mas localización, detección de objetos y ZSD. En la escala de los verdes se encuentran las clases vistas \{Caballo, Árbol\}, y en rojo las clases invisibles \{Perro, Persona, Campera, Pantalón, Correa\}.}
  \label{fig:DetectoresYSZD}
\end{figure}

La \autoref{fig:DetectoresYSZD} muestra un ejemplo de las distintas tareas  mencionadas anteriormente.\\

Además de los problemas mencionados anteriormente, existen otros como la segmentación, que no desarrollaremos en este trabajo. Aquí, solo nos enfocaremos en ZSD y sus problemas asociados.
 
Existen muchas técnicas propuestas para resolver ZSD. Cuando se empezó a leer sobre este tema a fines del 2018, la mas utilizada consistía en emplear multimodales. Puntualmente existían tres trabajos en paralelos \cite{rahman2018zero}\cite{zhu2018zero}\cite{bansal2018zero} con una metodología similar. La idea de esta técnica es utilizar un espacio compartido ente las representaciones de visión y del lenguaje. Para lograr esto, se utiliza  \textbf{incrustaciones de palabras} y \textbf{vectores con representaciones visuales}. Las primeras asignan a palabras una representación vectorial continua. Estos vectores se utilizan para medir similitudes semánticas y sintácticas entre palabras. Entre los modelos mas famosos se encuentran Glove~\cite{pennington-etal-2014-glove} y Word2vec~\cite{mikolov2013efficient}. Por otro lado, para obtener los vectores visuales de una imagen se utilizan redes profundas. Entre los mejores modelos se encuentran VGG~\cite{simonyan2014very}, ResNet~\cite{resnet} e Inception~\cite{Szegedy_2015_CVPR}. La \autoref{fig:EjemploZSD} describe como se utiliza la combinación de vectores de palabras y visuales para inferir un objetos nunca antes vistos por el modelo.\\

\begin{figure}[]
	\centering
	\includegraphics[width=1\textwidth]{img/Modelo.png}
	\caption{Descripción de la tarea de detección de objetos por disparo cero utilizando multimodales, donde los objetos ``Auto'', ``Camión'', ``Perro'' y ``Tigre'' se observan  durante el entrenamiento,  ``Gato'' y ``Camioneta'' son clases invisibles. El enfoque localiza estas clases no vistas aprovechando las relaciones del espacio semantico.}
	\label{fig:EjemploZSD}
\end{figure}

\section{Motivación} \label{sec:motivacion}

Hoy en día, hay una gran cantidad de modelos, capaces de detectar objetos en una imagen, como son las redes YOLO o Faster R-CNN. Estos, como otros no mencionados, poseen una excelente rendimiento, pero tienen una gran limitación, necesitan una gran cantidad de imágenes anotadas, para cada clase que se quiere detectar. Conseguir un gran numero de anotaciones, pude resultar un gran desafió, ya sea por la naturaliza del problema o por los grandes costo que esto conlleva. Esta dificultad se intenta mitigar con ZSD dado que puede inferir objetos no anotados.\\ 

ZSD es una habilidad que los humanos ya tienen. De hecho, podemos aprender muchas cosas con solo un ``conjunto de datos mínimo". Por ejemplo, tendemos a diferenciar  variedades de la misma fruta o frutas de aspecto similar, aun si hemos visto muy pocas veces cada tipo de fruta. La situación es diferente para las máquinas. Necesitan muchas imágenes para aprender a adaptarse a la variación que se produce de forma natural en lo humanos. Esta habilidad proviene de nuestra base de conocimientos lingüísticos existente, que proporciona una descripción de alto nivel de una clase nueva o no vista y establece una conexión entre ella y las clases que ya conocemos.\\

Por unos minutos dejemos llevarnos por la imaginación y supongamos que se quiere crear un programa capas de reconocer todos los objeto en una imagen, pero objetos de cualquier índole, animales, plantas, artículos de limpieza, o cualquier cosa que se te venga a la mente. Seria casi imposible, si es que no lo es, generar un conjunto de datos que contenga una cantidad considerable de imágenes de todos los objetos posible. Esta idea puede sonar muy descabellada, o no, pero no se puede negar su potencial y su gran cantidad de usos como en interpretaciones de escenas, seguridad, etc. A medida que ZSD continúa desarrollándose, se espera ver más aplicaciones, como mejores recomendaciones y soluciones más avanzadas que marcan automáticamente el contenido inadecuado dentro de las redes sociales, como así también un fuerte desarrollo en el campo de la robótica.

\section{Estructura de la tesis} \label{sec:estructuradelatesis}

Esta tesis se estructura de la siguiente manera. En el~\autoref{cap:preliminares} se detallan los conceptos fundamentales utilizado a lo largo del trabajo. En el~\autoref{cap:metodologia} se comienza formalizando el problema de ZSD y se define la arquitectura empleada para resolverlo. Ademas se describe los conjuntos de datos utilizados para entrenar y medir el rendimiento del modelo, como así también los detalles de nuestra implementación. Por ultimo se define las distintas métricas utilizadas y describe los distintos experimentos realizados. Luego en el~\autoref{cap:analisideresultado} se analizan los resultados obtenidos y se compara con distintos trabajos. Por ultimo el~\autoref{cap:conclusiones} se escriben las conclusiones que se obtuvieron los aportes realizados por esta tesis, los trabajos futuros y mejoras.


\chapter{Definición del problema}\label{cap:trabajo_relacionado}


\section{Redes neuronales convolucionales}
Las redes neuronales convolucionales CNN por sus siglas en ingles, es un tipo de modelo de aprendizaje profundo para procesar datos que tiene un formato de cuadrícula, como las imágenes. Está inspirado en la organización de la corteza visual de los animales, diseñada para aprender de forma automática y adaptativa, patrones en jerarquías, de bajo a alto nivel. Por lo genera una red CNN se compone de tres tipos de capas: convolución, agrupación y capas completamente conectadas. Las dos primeras, realizan extracción de características, mientras que la tercera, asigna las características extraídas en la salida final. La capa de convolución desempeña un papel clave en CNN, se compone de una pila de operaciones matemáticas, como la convolución, que es un tipo especializado de operación lineal. En las imágenes en 2D estas redes son muy utilizadas, por su alta eficiencia extrayendo características en cualquier parte de la imagen. \\
Algunos ejemplos de redes CNN son, VGG16 que posee 13 capas de convolución, 5 de agrupación y una totalmente conectada. AlexNet conocida por ganar la competencia 2012 ImageNet LSVRC-2012 por un amplio margen, contiene 5 capas convolucionales, 3 capas de agrupación y 3 capas completamente conectadas.\\
Las redes CNN se a utilizado para resolver distintos problemas como, la detección de objetos, Fast R-CNN \cite{girshick2015fast}. La comprensión visual de escenas de calles urbanas \cite{cordts2016cityscapes}.
En este trabajo utilizamos la salida de las redes CNN (la capa completamente conectada), como un vector de características visual de la imagen. Debido a que son muy eficaces reconociendo patrones, los vectores de dos imágenes que tienen un aspecto similar, también tienden a tener una semejanza. 

\begin{figure}
	\centering
	\includegraphics[width=0.9\textwidth]{img/red_cnn.png}
	\caption{Esta imagen muestra una arquitectura simplificada de una rede neuronal convolucional.}
	\label{fig:EvolucionILSVRC}
\end{figure}

\section{Word embedding}
Al igual que con las imágenes, que utilizamos las redes CNN, para obtener un vector que represente a la misma, es necesario un procedimiento para poder representar las palabras, con algún objeto matemático. Hay muchas formas de representar palabras, pero la mas conocida es word embedding, es una técnica de aprendizaje en el campo de procesamiento del lenguaje natural (PLN). Es capas de capturar el contexto de una palabra en un documento, calcular similitud semántica y sintáctica con otras palabras, etc. Word2Vec \cite{mikolov2013distributed} es una de la implementación mas conocida. Fue desarrollado por Tomas Mikolov en 2013. \\
El objetivo, es que las palabras con un contexto similar ocupen posiciones espaciales cercanas, mientras que palabras que no tienen un contexto similar estén espaciadas. Para lograr esto, se introduce cierta dependencia de una palabra de las otras palabras. Se utilizan texto para entran estos modelos, asi las palabras en el contexto de una palabra especifica, obtendrían una mayor proporción de esta dependencia.\\
En este trabajo, aprovechamos la capacidad de capturar similitudes semántica que tiene word embedding, para relacionar las clases vistas con las clases no vistas. Utilizamos un modelo pre-entrenado generado a partir de word2vec para representar las palabras de las distintas clases.

\section{Propuestas de objetos}
En problemas de detección de objetos, generalmente tenemos que encontrar todos los objetos posibles en la imagen como todos los autos todas las bicicletas, etc. La localización de objetos se refiere a identificar la ubicación de uno o varios objetos en la imagen. Un algoritmo de localización de objetos generará las coordenadas de la ubicación de los objetos con respecto a la imagen. En visión artificial, la forma más popular de representar la ubicación de los objetos es con la ayuda de cuadros delimitadores (Bounding Boxs). Existen muchos algoritmos y redes que intenta resolver este problema como por ejemplo, ventana deslizante, Edge-Boxes \cite{zitnick2014edge}, Selective search \cite{uijlings2013selective} ect. En ZSD las propuestas de objetos cumple un papel importante, ya que se necesita extraer todas las instancias de los objetos, pero también tiene que discriminar fondos como cielo, fondo de ciudad, veredas, ect. Es muy difícil encontrar un equilibrio ya que un algoritmo poco ``permisivo'' ignorara muchas instancias de objetos y por el otro extremo, se incluirá fondos y de esta manera introducir ruido en nuestro modelo. En este proyecto se usa Edge-Boxes, ya que este genera una cantidad significativamente menor a algoritmos del estilo de ventana deslizante. Aun asi, procesar todas estas propuestas es engorroso. Esto da lugar a una técnica que filtra las propuestas, denominada Supresión no máxima (NMS) \ref{fig:NMS}. Los criterios de selección de NMS se pueden elegir para llegar a resultados particulares. El criterio mas común es Intersección sobre Unión (IoU), en la imagen ~\ref{fig:IoU} se muestra como se calcula el IoU sobre dos Bounding Boxs.

\begin{figure}[H]
	\begin{subfigure}{.4\textwidth}
		\centering
		\includegraphics[width=0.7\textwidth]{img/iou.png}
		\caption{Iou}
		\label{fig:IoU}
	\end{subfigure}
	\begin{subfigure}{.6\textwidth}
		\centering
		\centering
		\includegraphics[width=1.1\textwidth]{img/NMS.png}
		\caption{Supresión no máxima}
		\label{fig:NMS}
	\end{subfigure}
	\caption{(a)Esta imagen muestra como se calcula el criterio Intersección sobre Unión. (b) Se muestra la salida de la propuesta de objetos y el resultado después de NMS.}
		\label{fig:RP}
\end{figure}

\section{Multimodales}

Nuestra experiencia del mundo es multimodal, vemos objetos, escuchamos sonidos, sentimos la textura, olemos los olores y probamos los sabores. La modalidad se refiere a la forma en que algo sucede o se experimenta y un problema de investigación se caracteriza como multimodal cuando incluye múltiples modalidades. Para que la inteligencia artificial avance en la comprensión del mundo que nos rodea, necesita poder interpretar y relacionar estas señales multimodales. Aunque la combinación de diferentes modalidades o tipos de información para mejorar el rendimiento parece una tarea intuitivamente atractiva, en la práctica, es un desafío combinar el nivel variable de ruido y los conflictos entre las modalidades. Además, las modalidades tienen una influencia cuantitativa diferente sobre el resultado de la predicción. La idea general es, partir de dos objetos matemáticos distintos uno de cada modal y poder transformar a ambos para lograr que pertenezcan a un tercer objeto que es la representación multimodal. Las imágenes suelen estar asociadas con etiquetas y explicaciones de texto. En este trabajo nos aprovechamos esto y tratamos de encontrar un espacio común entre el vector que representan a la imagen del objeto y el que representa la sintaxis del mismo.


\section {Aprendizaje por disparo cero (ZSL)}
Es una conjunto de problemas de aprendizaje automático, donde en el momento de la prueba, se observan muestras de clases que no se observaron durante el entrenamiento y se necesita predecir la categoría a la que pertenecen. A diferencia de las configuraciones estándares en el aprendizaje automático, donde se espera que se clasifiquen correctamente las nuevas muestras en las clases que ya han observado durante el entrenamiento, en ZSL, no se han proporcionado muestras de las clases durante el entrenamiento del clasificador.

El problema de aprendizaje zero-shot se puede dividir en categorías según los datos presentes durante la fase de entrenamiento y la fase de prueba.\\
\textbf{Con base en los datos disponibles en el momento de entrenar un modelo}
\begin{itemize}
	\item \textbf{Aprendizaje zero-shot inductivo:} Se tiene acceso a datos de imágenes etiquetadas de las clases vistas.
	\item \textbf{Aprendizaje zero-shot transductivo:} Además de los datos de imagen etiquetados de las clases vistas, también se tiene acceso a las imágenes no etiquetadas de las clases no vistas.
\end{itemize}
\textbf{Basado en los datos disponibles en el momento de la inferencia}
\begin{itemize}
	\item \textbf{Aprendizaje zero-shot convencional (ZSL):} En las pruebas solo se evalúan las clases no vistas.
	\item \textbf{Aprendizaje zero-shot generalizado (GZSL):} En las pruebas se evalúan tanto las clases vista como las no vistas.
\end{itemize}

\section {Detección de objeto por disparo cero (ZSD)}
La Detección de objeto por disparo cero (ZSD), tiene como objetivo reconocer y localizar simultáneamente instancias de objetos que pertenecen a categorías novedosas sin ningún ejemplo de entrenamiento. Bajo la clasificación anterior es un modelo inductivo. Como estas categorías no están presente en entrenamiento resulta imposible, tener alguna información sobre su aspecto visual, lo cual no nos permite detectarlas ni reconocerlas. Es necesario encontrar algún dominio que tenga la capacidad de guardar la información de todas las clases, para luego relacionarlas con el aspecto visual de las categorías novedosas. El mas utilizado es el espacio semantico.
\begin{figure}
	\centering
	\includegraphics[width=0.9\textwidth]{img/diseno_alto_nievel.png}
	\caption{Paso a paso, como se modifica la imagen para obtener las perdiciones de clases.}
	\label{fig:arqutectura}
\end{figure}
Para formalizar todo lo dicho, denotamos las clases como $\mathcal{C} = \mathcal{S} \cup \mathcal{U}$, donde $\mathcal{S}$ son las clases vistas para entrenamiento y $\mathcal{U}$ las clases no vistas, utilizadas en la etapa de pruebas. Aunque no es necesario definir el conjunto de clases de pruebas, ya que el modelo tiene que ser capas de detectar tanto clases vista como las no vista, se hace para poder tener una evaluación cuantitativa.\\
Denotamos a una imagen como $\mathcal{I} \in \mathbb{D}^{\mathcal{M} \times \mathcal{N} \times 3}$. Donde $\mathbb{D} = \{0,...,255\}$, $\mathcal{M}$  es el largo de la imagen, $\mathcal{N}$ el ancho. Esta es la forma en la que se representa una imagen en el formato \textbf{RGB}, uno de los mas utilizado. Por cada imagen se provee un conjunto de cuadros delimitadores  $\mathbb{B} = \{b_0,...,b_k\mid b_i \in N^4\}$ y sus etiquetas asociadas como $\mathbb{Y} = \{y_0,...,y_k\mid y_i \in \mathcal{S}\}$. Para cada cuadro delimitador $b_i$ extraemos una característica profunda utilizando una red neuronal convolucional denotada como $\phi(b_i) \in \mathbb{R}^{D_1}$. Denotamos las incrustaciones semánticas $w_j \in  \mathbb{R}^{D_2}$ obtenido por algun modelo de \textbf{Word embedding}. El objetivo es encontrar una matriz de proyección $W_p$ \[ \psi_i = W_p\phi(b_i) \:\:\:\mid\:\:\: W_P \in \mathbb{R}^{D_2 \times D_1},\:\:\: \psi_i \in \mathbb{R}^{D_2} \] Notar que $\psi_i$ y las incrustaciones semánticas se encuentran en el mismo dominio. Como mencionamos en secciones anteriores se tiene una gran capacidad de capturar similitudes semántica. Por lo cual resulta clave encontrar una matriz que para cada cuadro delimitador se proyecte lo mas cerca posible a la incrustación semántica de su clase.








\chapter{Diseño y Arquitectura}\label{cap:arquitectura}


\chapter{Experimentos}\label{cap:experimentos}

\section{Experimentación con Propuestas de objetos}
\begin{table}[]
	\centering
	\resizebox{12.5cm}{!} {
		\begin{tabular}{|l|c|r|r|r|c|r|}
			\hline
			\textbf{}                     & \multicolumn{4}{c|}{\textbf{Edge Boxes}}                                                                                                   & \multicolumn{2}{c|}{\textbf{Selective Search}}               \\ \hline
			\textbf{Algoritmo}            & \multicolumn{4}{c|}{\textbf{-}}                                                                                                            & \textbf{Single}         & \multicolumn{1}{c|}{\textbf{Fast}} \\ \hline
			\textbf{Numero de propuestas} & \textbf{100}                 & \multicolumn{1}{c|}{\textbf{500}} & \multicolumn{1}{c|}{\textbf{1000}} & \multicolumn{1}{c|}{\textbf{5000}} & \textbf{$\approx$ 5000} & \multicolumn{1}{c|}{\textbf{$\approx$ 1000}}  \\ \hline
			Tiempo promedio (s)           & \multicolumn{1}{r|}{0.11}    & 0,11                              & 0.12                               & 0,12                               & \multicolumn{1}{r|}{5,48}   &         1,41                           \\ \hline
			Propuetas Totales             & \multicolumn{1}{r|}{4.415.244} & 22.050.071                          & 43.802.935                           & 161.809.194                          & \multicolumn{1}{r|}{350.535.591}   &   95.643.172                                 \\ \hline
			Propuestas con IOU $> 0.5$    & \multicolumn{1}{r|}{86.233}   & 133.942                            & 155.584                             & 194.891                             & \multicolumn{1}{r|}{221.551}   & 203.563                                   \\ \hline
		\end{tabular}
	}
	\caption{Resultados de correr los distintos algoritmos de propuestas de regiones en los datos de entrenamiento. El numero de propuestas verdaderas es 261.258.}
	\label{tab:edgeVSselct}
\end{table}

Como se menciono anteriormente, el numero de propuestas es un parámetro clave. Algunas métricas son muy sensible a la cantidad de propuestas, afectando los resultados finales. Esto surgió, cuando se obtuvieron las primeras métricas, los valores estaban muy lejos de los esperados, y a medida que se aumentaba la cantidad de propuesta, los resultados empeoraban. Por este motivo se probaron dos algoritmos (\textbf{Edge Boxes} y \textbf{Selective Search}) con algunas combinaciones de sus parámetros. Con el objetivo de obtener una cantidad de propuestas que se superponga con el mayor numero de objetos sin afectar las métricas.\\

Para no sesgar al experimento con los datos de de prueba, se definió la metodología de la siguiente manera. Por cada imagen de entrenamiento se corrió el generador de propuestas, se calculo el tiempo y la cantidad de cuadros verdaderos que tenían un IoU $> 0.5$, con algún cuadro verdadero. El tiempo es un parámetro importante ya que algunos algoritmos soy muy lentos y resulta imposible usarlos. Como se puede observar en el Cuadro \ref{tab:edgeVSselct}, \textbf{Selective Search} obtiene una mayor cantidad de superposición, pero con un numero exageradamente grande de propuestas. La mejor opción es usar \textbf{Edge-boxes}. En cuanto numero de propuestas totales resulta mas conveniente entre 100 y 500 propuestas como máximo, ya que al aumentar este numero no se generan mejoras en superposición pero si aumenta el numero de propuestas. Si tenemos en cunta el tiempo, resulta mejor \textbf{Edge-boxes}, ya que demora una fracción de lo que tarda \textbf{Selective Search}.\\

\section{Experimentación con CNN}
Se deicidio analizar la CNN ya que el modelo final e muy dependiente de esta red y su capacidad de extraer caracteristicas visuales. Lo que se quiere aquí es que la red sea capas de asociar las caracteristicas visuales de objetos similares, y diferenciar los elementos de distinta naturaleza. En otras palabras, el espacio resultante tiene que distribuirse de tal manera que por ejemplo las imágenes de los animales estén muy cerca y a su ves alejado de vehículos o electrodomésticos, pero también tiene que mantener una separación entre los distintos animales como perro y gato. Bansal en su trabajo, propone utilizar \textbf{Inception ResNet V2}, pero esta red puede resulta muy pesada en cuanto a tiempo de ejecución y memoria. Por este motivo se decidió intentar con \textbf{VGG16}, que reduce el número de parámetros en las capas convolucionales y mejorar el tiempo de ejecución, ademas es una de la mas utilizada.\\

El experimento consistió en comparo miles de recuadros de 3 clases de entrenamiento, caballo, perro y camión.  Por cada cuadro se genero el vector de caracteristicas visuales. Luego se comparo utilizando la similitud coseno, entre todas las caracteristicas de caballo vs caballo, caballo vs camión y caballo vs perro. Se graficaron (Figura \ref{fig:vgg-vs-resnet}) las frecuencias de los resultados para cada CNN. Con esto se intenta observar como se distribuyen en el espacio visual, las distintas clases. Como se esperaba la similitud entre entre animales es mas grande que con un vehiculo. Pero, se observo que para \textbf{Inception ResNet V2} existe una mayor separación entre clases, aunque sus similitudes están mas dispersas. \textbf{VGG16}, parece tener una menor dispersión, pero la similitud coseno entre distintas clases tiene valores muy cercanos. Esto puede afectar de manera negativa ya que camión y caballo no poseen una gran diferencia y el modelo podría interpretarlo como clases similares.\\

% para generar tablas de latex
\begin{figure}
	\centering
	\includegraphics[width=1\linewidth]{img/vgg-vs-resnet}
	\caption{Frecuencia de la similitud coseno de los vectores de caracteristicas visuales, ente la misma y distintas clases, para las CNN  \textbf{Inception ResNet V2} y \textbf{VGG16}.}
	\label{fig:vgg-vs-resnet}
\end{figure}

\section{Definición de métricas}
Entre los diferentes conjuntos de datos anotados utilizados por los desafíos de detección de objetos y la comunidad científica, la métrica más común utilizada para medir la precisión de las detecciones es el  \textbf{Mean Average Precision (mAP)}, seguida por \textbf{Recall}. Un problema que tienen la métricas en detección de objetos, es la fata de una implementación estándar para calcularlas. Ademas, aquellas implementaciones publicas, están muy encapsuladas al código y resulta muy difícil adaptarlo, para medir rendimientos de modelos propios. Como ya se menciono, el código de Bansal \cite{bansal2018zero} no esta disponible, por este motivo fue necesario encontrar alguna implementación de estas métricas. Se encontraron varias y luego de hacer cambios para utilizarlos, los resultados variaban mucho de un código a otro.
Fue asi que se encontró el trabajo de Padilla Rafael \cite{padilla2020survey}, que explica lo planteado:\\

\begin{center}
	 \textit{``La falta de consenso en diferentes trabajos e implementaciones de AP es un problema al que se enfrentan las comunidades académicas y científicas. Las implementaciones métricas escritas en diferentes lenguajes y plataformas computacionales generalmente se distribuyen con los conjuntos de datos correspondientes que comparten una descripción determinada del cuadro delimitador. De hecho, estos proyectos ayudan a la comunidad con las herramientas de evaluación, pero exigen trabajo adicional para adaptarse a otros conjuntos de datos y formatos de cuadro delimitador.''}\\
\end{center}

Ademas Padilla, propone una definición y un código para estandarizar las métricas, de esta manera se pueden comprar distintos modelos, de una forma ``justa''. Por estos motivos decidimos utilizar este trabajo, aunque los resultados de nuestros modelos, no sean exactos a los reportados por Bansal \cite{bansal2018zero}.\\

Ahora definamos las métricas, basándonos en el trabajo \cite{padilla2020survey}. Primero es necesario, estandarizar cuando un cuadro es:
\begin{itemize}
	\item Falso negativo (\textbf{FN}): No se obtuvo ninguna detección en absoluto, o para un cuadro delimitador verdadero el IoU $> 0.5$ y no se predijo correctamente la clase
	\item Falso positivo (\textbf{FP}): Para un cuadro delimitador verdadero, se predijo correctamente la clase pero el IoU $< 0.5$, o es un predicción duplicada, es decir, ya se marco otra con mayor IOU como \textbf{TP}.
	\item Verdadero positivo (\textbf{TP}): Para un cuadro delimitador verdadero, se obtuvo  una propuesta con un IoU $> 0.5$ y se predijo correctamente la clase.
	\item Verdadero negativo (\textbf{TN}): Esto solo tiene sentido si, se quisiera medir propuestas que no tenían un IoU $> 0.5$ con todos los cuadros verdaderos, y ademas se predijo como clase de fondo. Pero en este trabajo no es utilizada.
\end{itemize}


La \textbf{Recall}, también conocida como sensibilidad, mide la probabilidad de que los objetos verdaderos (los que se encuentran en la imagen) se detecten correctamente, viene dado por: \[Recall =\frac{TP}{FN+TP}\] El trabajo de Bansal, define Recall de la siguiente manera: 
\begin{center}
	\textit{``Un cuadro delimitador predicho se marca como verdadero positivo solo si tiene una superposición de IoU mayor que un cierto umbral t con un cuadro delimitador de verdad del terreno y no se ha asignado ningún otro cuadro delimitador de mayor confianza al mismo cuadro de verdad del terreno. De lo contrario, se marca como falso positivo.''}\\
\end{center}

Según lo que se puede interpretar, utiliza los falsos positivos para calcular la \textbf{recall} en ves de usar Falso negativo. Sin poner en tela de juicio, si esto esta bien o mal, es claro que de esta forma solo se tiene en cuenta, los objetos que tuvieron al menos una propuesta con un IoU $> 0.5$ y el resto, quedan fuera del calculo de esta métrica. Esto genera una diferencia enorme en los resultados y dificulta la tarea de comprar con otros modelos, es  por esto que en este trabajo reportamos ambas. Bansal ademas, calcula una variación denominada K@Recall, donde solo se tienen en cuentan las K mejores propuestas basándose en la confianza de la predicción y el resto son descartadas.\\

\begin{equation} 
	\label{eqn:precision}
	Precision =\frac{TP}{FP+TP}
\end{equation}

AP, es una métrica popular para evaluar la precisión de los detectores de objetos mediante la estimación del área bajo la curva (AUC) de la relación \textbf{precisión} \ref{eqn:precision} x \textbf{recall}. La curva de \textbf{precisión} x \textbf{recall} puede verse como una compensación entre ambas métricas para diferentes valores de confianza asociados a los cuadros delimitadores generados por un detector. Si la confianza de un detector es tal que su FP es bajo, la precisión será alta. Sin embargo, en este caso, se pueden pasar por alto muchos aspectos positivos, lo que produce un FN alto y, por lo tanto, una recall baja. Por el contrario, si uno acepta más positivos, el recuerdo aumentará, pero el FP también puede aumentar, disminuyendo la precisión. Sin embargo, un buen detector de objetos debe encontrar todos los objetos reales  mientras identifica solo los objetos relevantes . Por lo tanto, un detector de objetos en particular puede considerarse bueno si su precisión permanece alta a medida que aumenta su recuperación, lo que significa que si el umbral de confianza varía, la precisión y la recall seguirán siendo altas. Por lo tanto, un área alta bajo la curva (AUC) tiende a indicar tanto una alta precisión como una alta recuperación. \textbf{mAP} para la detección de objetos es el promedio del AP calculado para todas las clases. Por lo general también se indica sobre que IoU se calcula, por ejemplo, mAP@0.5, o un conjunto de umbrales como mAP@[x, y]. El trabajo de Bansal reporta \textbf{mAP}, pero no indica sobre que IoU se calcula, a si que se asume que utilizo un valor de 0,5. Muchos trabajos que utilizan COCO, reportan mAP@[.5, .95]. Esta métrica resulta muy útil si se quiere comparar rendimientos entre distinto trabajos.


\section{Detalles de metodología de evaluación}
El principal experimento consto en replicar los resultados de Bansal \cite{bansal2018zero}. No se realizo exactamente sus experimentos, ya que esto no aportaría nada nuevo. Así que, se decidió analizar sus resultados y solo replicar los que consideramos indispensable y que seria un buen punto de partida. Por ejemplo, los experimentos con clases de fondo, no obtuvieron buenos resultados, en comparación con los que no la utiliza. Es por esto que no consideramos realizaros. El principal objetivo fue obtener un modelo que obtenga resultados lo mas similar posible a los reportados. Después de varias iteraciones, no se pudo lograr, y el principal motivo es la falta de una implementación para calcular las métricas. Se utiliza el código desarrollado por \cite{padilla2020survey}, pero sin ninguna garantía de que las métricas se calculen de la misma manera.\\

Ahora definamos la metodología de evaluación. El primer paso consiste generar propuestas para cada imagen, luego cada cuadro propuesto es reescalado al tamaño de la capa de entrada que tiene la CNN, y se le extrae el vector de características visuales. Después, se utiliza el modelo entrenado para inferir el vector de características semánticas, y se calcula la similitud coseno con los vectores semánticos de todas las clases o solo las invisibles, dependiendo si se quiere evaluar ZSDG o ZSD. Aquella clase que obtenga el mayor puntaje es asignada a la propuesta. También, se guarda la puntuación como la confianza de predicción.  Por ultimo, se agrupan todas las propuestas que se tengan asignada la misma clase y se corre un algoritmo de supresión no máxima. NMS elimina las predicciones repetidas y retorna las mejores propuestas de cada grupo. Al final obtenemos como resultado un conjunto de propuestas, sus clases y su respectivo puntaje. Estos datos se guardan en un archivo y luego se corre la implementación de Padilla, para obtener los resultados de las métricas.\\




\chapter{Conclusiones y trabajo futuro} \label{cap:conclusiones}

\section{Conclusiones y aportes} \label{sec:conclusionesyaportes}
Durante este trabajo se analizo de forma detallada y objetiva el desafiante problema de ZSD. Desde un principio, sabíamos que era un campo de investigación nuevo y que esto dificultaría el desarrollo de esta tesis. Los objetivos se fueron modificando en transcurso del tiempo. Aun así logramos genera un aporte en esta disciplina.\\

El primer paso de esta tesis fue la lectura y análisis de los distintos trabajos sobre ZSD. Un aspecto, que tenían la mayoría, es la utilización de incrustaciones visual y semánticas para abordar el problema. Por este motivo decidimos utilizar esta metodología, para proponer un modelo base, basándonos en el trabajo de Bansal \cite{bansal2018zero}. La falta de una implementación, nos obligo a profundizar en cada etapa del desarrollo, reduciendo las metas planteadas. Pero aportando un conocimiento mas detallado de la solución. 

Si bien el modelo base no fue propuesto por nosotros, aportamos detalles que surgieron de nuestra experimentación y como estos afectan a los resultados. También, se analizaron aspectos que al nuestro entender fueron ignorados por el trabajo original, pero resulta cruciales.

Otro aspecto importante analizado son los conjuntos de datos. Aun no existe uno especificado en el problema de ZSD, ni una adaptación consensuada de alguno ya existente. Proponemos una manera sencilla de dividir MSCOCO y la comparamos con la división del trabajo original que al parecer beneficia al modelo considerablemente.

Pero lo que creemos el mayor aporte es el análisis de la métricas. Esto es una gran debilidad en los trabajos relacionados actuales, que impide una comparación, justa y correcta. Debido a que los resultados no eran los esperados y luego de probar todos los cambios y mejoras posibles, nos encontramos con una definición ambigua que genero una mala interpretación. Decidimos investigar sobre las métricas y encontramos muchos documentos que señalaban el problema que tuvimos. Pero el trabajo de Padilla \cite{padilla2020survey} sobre sale a los demás, posea una clara definición de las métricas y una implementación fácil de utilizar, y se recomida su uso.

Si bien los resultados obtenidos no son los mejores y están por abajo de la expectativa. Aportan una idea de lo que son capaces los modelos de ZSD. Creemos que son mas transparente y detallados al trabajo original, haciendo posible agregar mejoras y ver su progreso de una manera cuantitativa.

Por ultimo, la comparación con modelos mas actuales, hace evidente el constante esfuerzo en este campo y sus mejoras demuestra que ZSD no es una fantasía.


\section{Trabajo futuro} \label{sec:trabajo futuro}

Teniendo en cuenta los resultados obtenidos en esta tesis, existen distintas alternativas para seguir profundizando. Las cual podemos dividir en tres grupos. 

El primero, es mejorar el algoritmo que genera propuestas, esto afecta sobre todo a la etapa de evaluacion. Trabajos actuales utilizan varios generadores simultáneamente, obteniendo algunas mejoras. Otros plantean aumentar el numero de propuestas considerablemente y utilizar un criterio  mas complejo a la supresión no máxima, para eliminar casillas repetidas y de fondo.

El segundo, surge de la simplicidad del modelo propuesto. Existe muchas formas de mejorarlo, algunas ideas pueden ser. Considerar la fusión de diferentes vectores de palabras (\textbf{Word2vec} y \textbf{GloVe}). Utilizar otro espacio que no sea el semantico y mapear ambas caracteristicas a este. Otro cambio, que tambien afecta el punto anterior es utilizar una única red unificada de extremo a extremo, capas de predecir  la ubicación de diferentes objetos y clasificarlos como lo es \textbf{Faster R-CNN}.

Por ultimo, resulta interesante suavizar el problema de ZSD. En ves de clasificar por clase, se pude utilizar sub-clases. Si bien esto no es una mejora, puede ayudar a entender si el modelo realmente esta relacionando objetos, ya que no es lo mismo confundir un perro con un auto que con lobo.

\thispagestyle{empty}

%----------------------------------------------------------------------------------------
%	APÉNDICES
%----------------------------------------------------------------------------------------

% \addtocontents{toc}{\vspace{2em}} % Agrega espacios en la toc

% \appendix % Los siguientes capítulos son apéndices

%  Incluye los apéndices en el folder de apéndices

% \include{Apendices/Ap}
% \thispagestyle{empty}
%\include{Apendices/AppendixB}
%\include{Apendices/AppendixC}

\addtocontents{toc}{\vspace{2em}} % Agrega espacio en la toc


%----------------------------------------------------------------------------------------
%	BIBLIOGRAFÍA
%----------------------------------------------------------------------------------------
\backmatter
\nocite{*}
\bibliographystyle{plain}
\bibliography{bibliografia.bib} %Aquí ponen el nombre del archivo .bib

\end{document}