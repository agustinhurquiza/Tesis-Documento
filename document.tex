%Plantilla basada en "Template for Masters / Doctoral Thesis" (plantilla disponible en writeLaTex) que subió LaTeXTemplates.com

\documentclass[12pt,twosided]{book}
\usepackage[paperwidth=17cm, paperheight=22.5cm, bottom=2.5cm, right=2.5cm]{geometry}
\usepackage{amssymb,amsmath,amsthm} %paquete para símbolo matemáticos
\usepackage[spanish]{babel}
\usepackage[utf8]{inputenc} %Paquete para escribir acentos y otros símbolos directamente
\usepackage{enumerate}
\usepackage{graphicx}
\usepackage{subcaption}
\usepackage{multirow}
\usepackage{float}
\usepackage{tikz-qtree,tikz-qtree-compat}
\usepackage{multicol}
\usepackage{algpseudocode}
\usepackage[nottoc]{tocbibind}
\usepackage[pdftex,
            pdfauthor={Agustin Horacio Urquiza Toledo},
            pdftitle={Zero-Shot Object Detection},
            pdfsubject={Ciencias de la computación},
            pdfkeywords={PALABRAS CLAVE},
            pdfproducer={Latex con hyperref},
            pdfcreator={pdflatex}]{hyperref}

\usepackage[colorinlistoftodos]{todonotes}
\usepackage[square,sort,comma,numbers]{natbib}

%\hypersetup{pdfborder = {0 0 0}}
\graphicspath{{img/}} %En qué carpeta están las imágenes

\begin{document}

%----------------------------------------------------------------------------------------
%	COMANDOS PERSONALIZADOS
%----------------------------------------------------------------------------------------
\newcommand{\etal}{\textit{et al}.}
\newcommand{\todos}[1]{{\color{red}[TODO: #1]}}
%Cambiar Cuadros por Tablas y lista de...
\renewcommand{\listtablename}{Índice de tablas}
\renewcommand{\tablename}{Tabla}
%----------------------------------------------------------------------------------------
%	PORTADA
%----------------------------------------------------------------------------------------
\title{Zero-Shot Object Detection} %Con este nombre se guardará el proyecto en writeLaTex

\begin{titlepage}
\begin{center}

\textsc{\Large Facultad de Matemática, Astronomía, Física y Computación}\\[2em]

\textsc{Universidad Nacional de Córdoba}

%Figura
\begin{figure}[h]
\begin{center}
\includegraphics[scale=0.2]{img/unc.jpg}
\end{center}
\end{figure}

\vspace{1em}

\textsc{\huge \textbf{Zero-Shot Object Detection}}\\[2em]

\textsc{\large Tesis}\\[1em]

\textsc{ para obtener el título de}\\[1em]

\textsc{Licenciado en Ciencias de la Computación}\\[1em]

\textsc{}\\[1em]

\textsc{\Large Agustin Horacio Urquiza Toledo}\\[1em]

\textsc{\large Director: Jorge Sanchez}

\end{center}

\vspace*{\fill}
\textsc{Córdoba, Argentina \hspace*{\fill} 2021}

\end{titlepage}

%----------------------------------------------------------------------------------------
%	DEDICATORIA
%----------------------------------------------------------------------------------------
\pagestyle{empty}
\frontmatter

\begin{flushright}
\textit{DEDICATORIA}
\end{flushright}

%----------------------------------------------------------------------------------------
%	AGRADECIMIENTOS

%----------------------------------------------------------------------------------------
\chapter*{Agradecimientos}
A la Universidad Nacional de Córdoba por haberme dado la
oportunidad de formarme. Quiero agradecer a mi tutor Dr. Jorge Sanchez, que me guió y me enseñó lo necesario para realizar este trabajo. A mi familia, y en especial a mis padres, Graciela Toledo y Horacio Urquiza, quienes sin entender del todo lo que hago están siempre apoyándome. A mis compañeros de la Licenciatura, que crecí junto a ellos no sólo en lo profesional sino también en lo personal. Y por último a mis amigos de la vida, que siempre me escucharon y me ayudaron a despejarme de tanto estudio.

%----------------------------------------------------------------------------------------
%	PREFACIO
%----------------------------------------------------------------------------------------
\chapter*{Resumen}

\pagestyle{plain}

En el año 2010 surgió la llamada ``Revolución'' del Aprendizaje profundo, y con esto, surgieron una gran cantidad de métodos capases de detectar objetos en una imagen. Estos algoritmos o modelos fueron mejorando en cada año, hasta hoy en día, que alcanzaron un excelente rendimiento e innumerables aplicaciones. Pero estos poseen una limitación, necesitan tener una gran cantidad de imágenes anotadas que en algunos casos resulta inviable. Para resolver este problema surgieron técnicas como \textbf{Zero-shot Object Detection} (ZSD). Este es un problema de investigación recientemente propuesto, que tiene como objetivo localizar y reconocer simultáneamente objetos de clases nunca antes vistas. Generalmente se proponen modelos de ZSD como una combinación características visuales y descripciones semánticas, aprendiendo un mapeo visual-semántico de las características visuales de las imágenes a las incrustaciones semánticas de etiquetas de cada clases de objeto. Este ultimo es un espacio donde las caracteristicas de lo objetos vistos son transferidos a los novedoso.

En este tesis, analizamos distintos trabajos y llevamos a cabo experimentos en el conjunto de datos COCO, con la idea de aportar una noción del estado actual en esta área. \\\\\\\\\\\\
{\huge\textbf{Summary}}\\\\

\textcolor{red}{/* ToDo: Traducir la nueva versión del Resumen. */}
In the 2000s, the popularization of digital cameras and the increase of the Internet speed caused the rapid development of Internet images. This fact created the need of developing faster and more efficient methods to facilitate the extraction of information from these data. In 2010, with the Deep Learning ``revolution'', many different methods emerged to carry out this task, among them Detectors. These algorithms use a large number of image annotations that, in some cases, results impossible to obtain. As an alternative to solve this problem, techniques such as \textbf{Zero-shot Object Detection} emerged. 

This thesis tackles this problem using deep features and semantic embedding, with the aim of detecting and recognizing novel object instances. Hence, we analyzed different works from literature and we performed experiments that provide a notion of the current state in this area.

%----------------------------------------------------------------------------------------
%	TABLA DE CONTENIDOS
%---------------------------------------------------------------------------------------
\begingroup
\hypersetup{hidelinks}
\tableofcontents
\endgroup

%----------------------------------------------------------------------------------------
%	TESIS
%----------------------------------------------------------------------------------------
\mainmatter %empieza la numeración de las páginas
\pagestyle{headings}

%  Incluye los capítulos en el folder de capítulos

\chapter{Introducción y Motivación} \label{cap:intro}

\section{Historia} \label{sec:historia}
La detección de objetos es una de las áreas de la visión por computadora que está creciendo más rápidamente. Gracias al aprendizaje profundo, cada año, los nuevos algoritmos/modelos siguen superando a los anteriores. Aunque la visión por computadora recientemente tomó gran importancia (el momento decisivo ocurrió en 2012 cuando AlexNet ganó ImageNet), ciertamente no es un nuevo campo científico.\\

Uno de los artículos más influyentes en Visión Informática fue publicado por dos neurofisiólogos, David Hubel y Torsten Wiesel~\cite{hubel1959receptive}, en 1959. Su publicación, titulada \textit{``Receptive fields of single neurons in the cat’s striate cortex''}, en español ``Campos receptivos de neuronas individuales en la corteza estriada del gato'', describió las propiedades de respuesta central de las neuronas corticales visuales y como la experiencia visual de un gato moldea su arquitectura cortical. Los investigadores establecieron a través de su experimentación (\autoref{fig:ExpermentoHubelTorsten}) que existen neuronas simples y complejas en la corteza visual primaria, y que el procesamiento visual siempre comienza con estructuras simples como los bordes orientados y gradualmente identifica estructuras mas complejas. En la actualidad, este es el principio básico detrás del aprendizaje profundo.\\

\begin{figure}
	\centering
	\includegraphics[width=0.5\textwidth]{img/cat.jpg}
	\caption{Simple explicación del experimento realizado por David Hubel y Torsten Wiesel}
	\label{fig:ExpermentoHubelTorsten}
\end{figure}

Otro echo importante en la historia de la visión por computadora fue en 1957, Russell Kirsch y sus colegas desarrollaron un aparato que permitía transformar imágenes en cuadrículas de números que las máquinas de lenguaje binario podían entender. 

Poco tiempo después, en la década de 1960 fue cuando la IA se convirtió en una disciplina académica y algunos de los investigadores eran extremadamente optimistas sobre el futuro del campo. En este periodo, Seymour Papert, profesor del laboratorio de IA del MIT, decidió lanzar el Proyecto de Verano y resolver, en pocos meses, el problema de la visión artificial. Los estudiantes debían diseñar una plataforma que pudiera realizar automáticamente segmentación de fondo y extraer objetos no superpuestos de imágenes del mundo real. Claro esta que el proyecto no fue un éxito.  Hoy en día, cincuenta años después, todavía no se ha podido resolver la visión por computadora. Sin embargo, ese proyecto fue el nacimiento oficial de esta disciplina como campo científico. 

Los aportes mas influyentes en este campo empezaron a surgir a partir de los años 2000. En 2001 Paul Viola y Michael Jones~\cite{viola2001rapid} presentaron el primer detector de rostros que funcionó en tiempo real. Aunque no se basaba en el aprendizaje profundo, el algoritmo tenía una relación con éste, ya que, al procesar imágenes aprendió qué características podrían ayudar a localizar caras, inspirándose en el experimento de David Hubel y Torsten Wiesel. 

En 2006 comenzó la competencia de Pascal VOC que permitió evaluar el desempeño de diferentes métodos para el reconocimiento de objetos. Mas tarde, en 2010, siguiendo los pasos de Pascal VOC, se inició el concurso de reconocimiento visual a gran escala ImageNet (ILSVRC), cuya tasa de error durante 2010 y 2011, en el desafió de clasificación de imágenes, rondaba el 26\%.  En 2012, un equipo de la Universidad de Toronto ingresó a la competencia con un modelo de red neuronal convolucional (AlexNet)~\cite{krizhevsky2012imagenet} que cambió todo, dado que logró una tasa de error del 16,4\%. En los años siguientes, las tasas de error en la clasificación de imágenes en ILSVRC cayeron a un pequeño porcentaje, como se observa en la \autoref{fig:EvolucionILSVRC} y los ganadores, desde 2012, siempre han sido redes neuronales convolucionales.

\begin{figure}[H]
	\centering
	\includegraphics[width=0.7\textwidth]{img/imgnet-grafico.png}
	\caption{Evolución de los modelos propuestos en la competencia ILSVRC}
	\label{fig:EvolucionILSVRC}
\end{figure}

\section{Detectores y ZSD} \label{sec:detectoresyzsd}
La detección de objetos es un subproblema de la visión artificial, que estudia cómo detectar la presencia de objetos en una imagen. Debido a la complejidad de poder detectar todas las instancias de todos los posibles objectos en una imagen, se dividió en distintas tareas para disminuir la dificultad. 

Par explicar los distintos problemas, es necesario distinguir dos conjuntos. Por un lado, los datos de entrenamiento, que consta de las imágenes que se usan para entrenar el modelo con sus respectivas etiquetas, es decir, que objetos se encuentran en la imagen, localización de los objetos, descripción de la imagen, o cualquier información extra que requiera la tarea. Por otro lado, las imágenes de prueba, que es el conjunto donde se observará o medirá la eficiencia del modelo ya entrenado. 

Supongamos que las etiquetas solo cuenta con dos tipos de información, que clase de objeto es, es decir si es un perro, auto, persona, etc. y su localización en la imagen. A todas las clases de objetos que aparecen en los datos de entrenamiento las llamaremos clases visibles o vistas, y todas aquellas clase que no sea una clase vista las llamearemos invisible o no vista. Dicho esto, los distintos problemas son:

\begin{itemize}
	\item \textbf{Clasificación}: consta de un modelo capás de predecir si una clase específica esta presente en una imagen. 
	\item \textbf{Clasificación mas localización}: además de poder clasificar tiene que ser capas de ubicar el objecto en la imagen.
	\item \textbf{Reconocimiento de imagen}: predice que objetos perteneciente a las clases visibles están presente en la imagen. 
	\item \textbf{La detección de objetos}: además de reconocer objetos visibles, tiene que ser capás de localizar dichos objetos. 
	\item \textbf{Reconocimiento por disparo cero}: tiene que poder reconocer clases vistas y no vistas.
	\item \textbf{Detección de objetos por disparo cero} (\textbf{ZSD} por sus siglas en inglés): debe localizar y clasificar todas las instancias de objetos en la imagen, sin depender si es una clase vista o no.
\end{itemize}

\begin{figure}[]
  \centering
  \subcaptionbox{\tiny{\textbf{ Clasificación}}}{\includegraphics[width=1.5in]{img/expect_3.png}}\hspace{1em}%
  \subcaptionbox{\tiny{\textbf{Detección de objetos}}}{\includegraphics[width=1.5in]{img/expect_2.png}}
  \subcaptionbox{\tiny{\textbf{ZSD}}}{\includegraphics[width=1.5in]{img/expect_1.png}}
  \caption{Ejemplo de tareas de clasificación mas localización, detección de objetos y ZSD. En la escala de los verdes se encuentran las clases vistas \{Caballo, Árbol\}, y en rojo las clases invisibles \{Perro, Persona, Campera, Pantalón, Correa\}.}
  \label{fig:DetectoresYSZD}
\end{figure}

La \autoref{fig:DetectoresYSZD} muestra un ejemplo de las distintas tareas  mencionadas anteriormente.\\

Además de los problemas mencionados anteriormente, existen otros como la segmentación, que no desarrollaremos en este trabajo. Aquí, solo nos enfocaremos en ZSD y sus problemas asociados.
 
Existen muchas técnicas propuestas para resolver ZSD. Cuando se empezó a leer sobre este tema a fines del 2018, la mas utilizada consistía en emplear multimodales. Puntualmente existían tres trabajos en paralelos \cite{rahman2018zero}\cite{zhu2018zero}\cite{bansal2018zero} con una metodología similar. La idea de esta técnica es utilizar un espacio compartido ente las representaciones de visión y del lenguaje. Para lograr esto, se utiliza  \textbf{incrustaciones de palabras} y \textbf{vectores con representaciones visuales}. Las primeras asignan a palabras una representación vectorial continua. Estos vectores se utilizan para medir similitudes semánticas y sintácticas entre palabras. Entre los modelos mas famosos se encuentran Glove~\cite{pennington-etal-2014-glove} y Word2vec~\cite{mikolov2013efficient}. Por otro lado, para obtener los vectores visuales de una imagen se utilizan redes profundas. Entre los mejores modelos se encuentran VGG~\cite{simonyan2014very}, ResNet~\cite{resnet} e Inception~\cite{Szegedy_2015_CVPR}. La \autoref{fig:EjemploZSD} describe como se utiliza la combinación de vectores de palabras y visuales para inferir un objetos nunca antes vistos por el modelo.\\

\begin{figure}[]
	\centering
	\includegraphics[width=1\textwidth]{img/Modelo.png}
	\caption{Descripción de la tarea de detección de objetos por disparo cero utilizando multimodales, donde los objetos ``Auto'', ``Camión'', ``Perro'' y ``Tigre'' se observan  durante el entrenamiento,  ``Gato'' y ``Camioneta'' son clases invisibles. El enfoque localiza estas clases no vistas aprovechando las relaciones del espacio semantico.}
	\label{fig:EjemploZSD}
\end{figure}

\section{Motivación} \label{sec:motivacion}

Hoy en día, hay una gran cantidad de modelos, capaces de detectar objetos en una imagen, como son las redes YOLO o Faster R-CNN. Estos, como otros no mencionados, poseen una excelente rendimiento, pero tienen una gran limitación, necesitan una gran cantidad de imágenes anotadas, para cada clase que se quiere detectar. Conseguir un gran numero de anotaciones, pude resultar un gran desafió, ya sea por la naturaliza del problema o por los grandes costo que esto conlleva. Esta dificultad se intenta mitigar con ZSD dado que puede inferir objetos no anotados.\\ 

ZSD es una habilidad que los humanos ya tienen. De hecho, podemos aprender muchas cosas con solo un ``conjunto de datos mínimo". Por ejemplo, tendemos a diferenciar  variedades de la misma fruta o frutas de aspecto similar, aun si hemos visto muy pocas veces cada tipo de fruta. La situación es diferente para las máquinas. Necesitan muchas imágenes para aprender a adaptarse a la variación que se produce de forma natural en lo humanos. Esta habilidad proviene de nuestra base de conocimientos lingüísticos existente, que proporciona una descripción de alto nivel de una clase nueva o no vista y establece una conexión entre ella y las clases que ya conocemos.\\

Por unos minutos dejemos llevarnos por la imaginación y supongamos que se quiere crear un programa capas de reconocer todos los objeto en una imagen, pero objetos de cualquier índole, animales, plantas, artículos de limpieza, o cualquier cosa que se te venga a la mente. Seria casi imposible, si es que no lo es, generar un conjunto de datos que contenga una cantidad considerable de imágenes de todos los objetos posible. Esta idea puede sonar muy descabellada, o no, pero no se puede negar su potencial y su gran cantidad de usos como en interpretaciones de escenas, seguridad, etc. A medida que ZSD continúa desarrollándose, se espera ver más aplicaciones, como mejores recomendaciones y soluciones más avanzadas que marcan automáticamente el contenido inadecuado dentro de las redes sociales, como así también un fuerte desarrollo en el campo de la robótica.

\section{Estructura de la tesis} \label{sec:estructuradelatesis}

Esta tesis se estructura de la siguiente manera. En el~\autoref{cap:preliminares} se detallan los conceptos fundamentales utilizado a lo largo del trabajo. En el~\autoref{cap:metodologia} se comienza formalizando el problema de ZSD y se define la arquitectura empleada para resolverlo. Ademas se describe los conjuntos de datos utilizados para entrenar y medir el rendimiento del modelo, como así también los detalles de nuestra implementación. Por ultimo se define las distintas métricas utilizadas y describe los distintos experimentos realizados. Luego en el~\autoref{cap:analisideresultado} se analizan los resultados obtenidos y se compara con distintos trabajos. Por ultimo el~\autoref{cap:conclusiones} se escriben las conclusiones que se obtuvieron los aportes realizados por esta tesis, los trabajos futuros y mejoras.


\chapter{Marco teórico}\label{cap:marcoteorico}

Para poder entender como funcionan los modelos de ZSD, primero, es necesario comprender como se componen. Por lo general utilizan multimodales, es decir, combinan los espacios semánticos y visuales para poder detectar objetos no vistos. A su ves para poder extraer caracteristicas visuales se necesita utilizar redes neuronales convolucionales y un algoritmo capas de indicar donde hay objetos en una imagen, por otro lado, se requiere un mecanismo para generar las caracteristicas semanticas como por ejemplo Word embedding.

En este capitulo se exponen cada componete que utilizan los modelos de ZSD, como estos se emplean y relacionan.

\section {Zero-shot learning (ZSL)} \label{sec:zeroshotlearning}
ZSL es un conjunto de problemas de aprendizaje automático, donde en el momento de la prueba, se observan muestras de clases que no se observaron durante el entrenamiento y se necesita predecir la categoría a la que pertenecen. Esta se diferencia de las configuraciones estándares en el aprendizaje automático, donde se espera que se clasifiquen correctamente las nuevas muestras en las clases que ya se han observado durante el entrenamiento.
Podemos diferenciar dos tipos de clases, las vistas que están presente en el entrenamiento y las invisibles o novedosas que no estuvieron en el mismo. Se debe proporcionar algún tipo de información complementaria sobre estas clases invisibles, este tipo de dato puede ser variado. Como por ejemplo, una descripción estructurada predefinida, que al tenerla en cuenta mejora el aprendizaje, o una descripción textual, donde las clases van acompañadas de un comentario en lenguaje natural, etc. Por último, tanto las clases visibles como las invisibles están relacionadas en un espacio vectorial, donde el conocimiento de las clases vistas se puede transferir a clases invisibles.\\

Para poder entender mejor como funciona ZSL, supongamos que queremos detectar un animal en extinción para poder rastrearlos y tener una forma de contabilizarlos, por ejemplo el yaguareté que habita el Chaco Argentino. Poder conseguir un conjunto considerable de fotos del mismo en distintos entornos, situaciones y ubicaciones, como de día, noche, corriendo, descansando, etc. puede resultar un gran desafió y demasiado tarde si nos lleva mucho tiempo. Pero se puede proponer un modelo de ZSL, entrenado con  imágenes de animales similares como leopardo, puma, tigre, etc. y reforzar el entrenamiento con algunas imagenes de yaguareté. \textcolor{red}{/* ToDo: Como formalizar el ejemplo?*/}\\

El problema de zero-shot learning se puede dividir en categorías según los datos presentes durante la fase de entrenamiento y la fase de prueba:
\begin{itemize}
	\item En base a los datos disponibles en el momento de entrenar un modelo.
	\begin{itemize}
		\item \textbf{Zero-shot learning inductivo:} Se tiene acceso a los datos y a la información complementario de solo las clases vistas.
		\item \textbf{Zero-shot learning transductivo:} Además de los datos y la información complementario de las clases vistas,  se tiene acceso a los datos de las clases no vistas.
	\end{itemize}
	\item Basado en los datos disponibles en el momento de la inferencia.
	\begin{itemize}
		\item \textbf{Zero-shot learning convencional (ZSL):} En las pruebas solo se evalúan las clases no vistas.
		\item \textbf{Zero-shot learning generalizado (GZSL):} En las pruebas se evalúan tanto las clases vista como las no vistas.
	\end{itemize}
\end{itemize}

\section {Zero-shot object detection (ZSD)} \label{sec:zeroshotobjectdetection}
Zero-shot object detection, tiene como objetivo reconocer y localizar simultáneamente instancias de objetos que pertenecen a categorías novedosas sin ningún ejemplo de entrenamiento. Como estas categorías no están presente en el entrenamiento, no se tiene ninguna información sobre su aspecto visual, lo cual no nos permite detectarlas ni reconocerlas. Por lo tanto, es necesario encontrar algún dominio que tenga la capacidad de guardar la información de todas las clases, para luego relacionarlas con el aspecto visual de las categorías novedosas.\\

Este trabajo se basa en el artículo científico de Bansal \etal ~\cite{bansal2018zero}, además utiliza muchos conceptos sobre zero-shot learning generalizado~\cite{zero-shot-generalizado}. Se propone un modelo zero-shot  inductivo, es decir, solo se observan imágenes de clases vistas y etiquetas que indican a que clase pertenece. Estas etiquetas son palabras del lenguaje natural sin ninguna estructura. Luego, se puede inferir todas las clases o solo las invisibles, dependiendo de si se quiere evaluar aprendizaje por zero-shot  generalizado o convencional, respectivamente.\\ 

El requisito estricto de no utilizar ninguna imagen de clase invisible durante el entrenamiento es una condición difícil. Además, existen otras dificultades en la tarea de ZSD relacionadas al conjunto de datos de entrenamiento y prueba, es decir entre las clases vistas e invisibles. Estas dificultades son:

\begin{itemize}
	\item \textbf{Rareza}: los conjuntos de datos, por lo general, contiene un problema de distribución, es decir, muchas clases raras tienen menos cantidad de instancias. Este problema hace que las clases con mayor cantidad de instancias sesguen el modelo y las clases más raras sean marcadas incorrectamente en la etapa de prueba. Esto es un problema al momento de comparar dos modelos que fueron entrenados con distintas clases, ya que algunas separaciones  de las clases resultan mejores que otras.
	
	\item \textbf{Tamaño del objeto}: algunas clases de objetos raros (tijeras, lápices, celulares, etc.), suelen tener un tamaño pequeño. Los objetos más pequeños son difíciles de detectar y reconocer. También, tienen el problema de que por lo general están junto a objetos más grandes como una mesa o una persona y se ven opacadas por estas clases.
	
	\item \textbf{Diversidad}: cuando una clase invisible no tiene otras clases visualmente similares, resulta muy difícil aprender el aspecto visual de esta. Por ejemplo, ``auto'' tiene muchas clases similares en comparación con ``cartel''. Esto permite una descripción visual inadecuada de la clase invisible ``cartel'' que eventualmente afectará el rendimiento de ZSD, a diferencia de lo que sucede con la clase ``auto''.
	
	\item \textbf{Ruido en el espacio semántico}: cuando se utiliza los vectores de incrustación semántica no supervisados como Word2Vec~\cite{mikolov2013distributed} o GloVe~\cite{pennington2014glove}, las embeddings resultante en general son ruidosas, ya que se generan automáticamente a partir de la minería de texto no anotado. Esto también afecta significativamente el rendimiento de ZSD.
\end{itemize}

Existen algunas variaciones de ZSD que intentan atenuar estos problemas, agrupando las clases en grupos o metaclases. Estas variaciones son: 

\begin{itemize}
	
	\item \textbf{Zero-shot metaclass detection (ZSMD)}, que dada una imagen de prueba, el objetivo es localizar cada instancia de una clase de objeto invisible y categorizarla en una de las superclases.
	
	\item \textbf{Zero-shot tagging (ZST)}, consiste en reconocer una o más clases invisibles en una imagen de prueba, sin identificar su ubicación.
	
	\item \textbf{Zero-shot metaclass tagging (ZSMT)}, reconoce una o más metaclases en una imagen de prueba, sin identificar su ubicación.
	
\end{itemize}

Estas tareas son atractivas a la hora de calcular métricas, pero no para ser aplicadas a problemas reales.


\section{Redes neuronales convolucionales} \label{sec:redesneuronalesconvolucionales}
Las redes neuronales convolucionales, o CNN por sus siglas en inglés, son un tipo de modelos de aprendizaje profundo (\textit{Deep learning}) utilizadas  para procesar distintos tipos de datos, pero empleadas generalmente en el dominio de las imágenes. Está inspirado en la organización de la corteza visual de los animales, diseñada para aprender de forma automática y adaptativa patrones en jerarquías, de bajo a alto nivel. Esto quiere decir que las primeras capas pueden detectar lineas, curvas y se van especializando hasta llegar a capas más profundas que reconocen formas complejas como un rostro. Por lo genera una red CNN se compone de tres tipos de capas: convolución, agrupación y capas completamente conectadas, como se puede ver en la \autoref{fig:CNNEjemplo}. El rol de cada capa es:

\begin{itemize}
	\item Convolución: somete los datos de entrada a un conjunto de filtros convolucionales, cada uno de los cuales activa ciertas características de los datos. Generalmente esta acompañada de una capa de activación, que permite un entrenamiento más rápido y eficaz al asignar los valores negativos a cero y mantener los valores positivos, de esta manera permitir que solo las características activadas pasen a la siguiente capa.
	\item Agrupación: se coloca generalmente después de la capa convolucional. Su utilidad principal radica en la reducción de su entrada para la siguiente capa convolucional, reduciendo así el número de parámetros que la red necesita aprender.
	\item Capas completamente conectadas: Es la encargada de relacionar los datos de las capas anteriores y generar una salida, que por lo general es utilizada para clasificar los datos de entrada.
\end{itemize}


Algunos ejemplos de redes CNN son: VGG16~\cite{simonyan2014very} (que posee 13 capas de convolución, 5 de agrupación y una totalmente conectada) y AlexNet~\cite{krizhevsky2012imagenet} (que contiene 5 capas convolucionales, 3 capas de agrupación y 3 capas completamente conectadas).\\

Estas redes son fundamentales para un modelo de detección de objeto y por lo tan tanto para ZSD. Debido a que las CNN son muy eficaces reconociendo patrones, si dos imágenes tienen un aspecto similar, los vectores también tendrán una semejanza. Esta es la forma que tenemos en este trabajo de relacionar los aspectos visuales de las clases vista e invisibles. Por ejemplo, supongamos un lavarropa y un lavavajilla que tienen un aspecto similar, donde la primera es una clase visible y la segunda es invisible, gracias a que las CNN generan vectores muy parecidos se puede obtener la información de que estas clases son similares sin a ver visto antes un lavavajilla. Por lo general se elimina la capa totalmente conectada y se utiliza la salida de la etapa previa como un vector de las caracteristicas visuales.

\begin{figure}
	\centering
	\includegraphics[width=0.9\textwidth]{img/red_cnn.png}
	\caption{Una arquitectura simplificada de una red neuronal convolucional.}
	\label{fig:CNNEjemplo}
\end{figure}

\section{Word embedding} \label{sec:wordembedding}
Así como en las imágenes utilizamos las redes CNN, para obtener un vector que represente a la misma, es necesario un procedimiento para representar palabras con algún objeto matemático. Hay muchas formas de representar palabras, la más usada son los \textit{word embedding}. Esta es una técnica de aprendizaje en el campo de procesamiento del lenguaje natural (PLN), capaz de capturar el contexto de una palabra en un documento, calcular similitud semántica y sintáctica con otras palabras.\\

Para entender como funcionan, consideremos las oraciones con un significado similar: ``Que tengas un buen día.'' y ``Que tengas un gran día.''. Si construimos un vocabulario exhaustivo:
 \[ V = \{que, tengas, un, buen, gran, dia\}. \]
A partir de esto, se puede crear un vector codificado para cada una de estas palabras, en donde cada vector tenga el tamaño de $V$, cuyos componentes sean todos 0 excepto por el elemento en el índice que representa la palabra correspondiente en el vocabulario, que contiene un 1. Esta representación no resulta conveniente ya que la distancia entre \textit{gran} y \textit{buen} es la misma que entre \textit{tengas} y \textit{buen}.  El objetivo es que las palabras con un contexto similar ocupen posiciones espaciales cercanas. Para lograr esto, se introduce cierta dependencia de una palabra con las otras.\\

Word2Vec~\cite{mikolov2013distributed} desarrollado por Tomas Mikolov en 2013. Es un modelo particularmente eficiente desde el punto de vista computacional. Este modelo se encuentra disponible de dos formas: \textit{Continuous Bag-of-Words} (CBOW) o el modelo \textit{Skip-Gram}. En CBOW, las representaciones distribuidas de contexto (o palabras circundantes) se combinan para predecir la palabra en el medio. En nuestro ejemplo \textit{gran} y \textit{buen} están rodeado de un contexto similar por lo cual resultan en vectores similares. Es varias veces más rápido de entrenar que el \textit{Skip-gram}, y tiene una precisión ligeramente mejor para las palabras frecuentes. Mientras que en el modelo \textit{Skip-gram}, la representación distribuida de la palabra de entrada se usa para predecir el contexto. Se entrena con una tarea falsa que, dada una palabra, intenta predecir las palabras vecinas. En realidad, el objetivo es solo aprender los pesos de la capa oculta que corresponden a los vectores de palabras que estamos tratando de aprender. Por ejemplo, \textit{Gran} se entrena para predecir el contexto \textit{un} y  \textit{día}, al igual que \textit{buen}. Funciona bien con una pequeña cantidad de datos de entrenamiento.

En este trabajo, aprovechamos la capacidad de capturar similitudes semántica que tiene \textit{word embedding}, para relacionar las clases vistas con las clases invisibles.\\


\section{Propuesta de objetos} \label{sec:propuestadeobjetos}
En problemas de detección de objetos, generalmente se tiene que encontrar todos los objetos posibles en la imagen. La localización de objetos se refiere a identificar la ubicación de uno o varios objetos en la imagen. Un algoritmo de localización de objetos generará las coordenadas de la ubicación de los objetos con respecto a la imagen. En visión artificial, la forma más popular de representar la ubicación de los objetos es con la ayuda de cuadros delimitadores (\textit{Bounding Boxes}). Existen muchos algoritmos y redes que intenta resolver este problema, algunos ejemplos son ventana deslizante (\textit{slide window}), Edge-Boxes~\cite{zitnick2014edge} y búsqueda selectiva (\textit{selective search})~\cite{uijlings2013selective}. En ZSD la propuesta de objetos cumple un papel importante, ya que se necesita extraer todas las instancias de los objetos, pero también tiene que discriminar fondos como cielo, ciudades, veredas, etc.\\

En este proyecto, como veremos en el \autoref{cap:experimentos} se experimentó con Edge-Boxes y \textit{selective search}, ya que estas generan una cantidad de propuestas significativamente menor a algoritmos del estilo de ventana deslizante. Aun así, procesar todas estas propuestas es engorroso. Adamas, estos modelos por lo general dan como resultados muchos cuadros con una gran superposición. Esto da lugar a una técnica denominada supresión de no máximos (NMS), ejemplificada en la~\autoref{fig:NMS}. Este algoritmo necesita de un puntaje que indica la confianza del cuadro delimitador y un criterio para comparar entre distintos cuadros. El criterio más común es Intersección sobre Unión (IoU), en la~\autoref{fig:IoU} se muestra como se calcula sobre dos \textit{Bounding Boxes}. La salida de NMS es un conjunto más reducido de propuestas, en la cual se filtraron todas las que se consideran repetidas y retorna solo las más representativa. A continuación se muestra el pseudocódigo de NMS.

\begin{center}
\noindent\fbox{
	\begin{minipage}{1\textwidth}
		\begin{algorithmic}[1]
			\Procedure{NMS}{B, S, t}
				\State{D = $\emptyset$ }
				\For {B $\neq \emptyset$ }
					\State{$b_i$ = \textbf{SelPropuestaMaxPuntaje}(B, S)}
					\State{\textbf{Eliminar}($b_i$, B)}
					\For {$b_j$ $\in$ B}
						\If {\textbf{IoU}($b_i$, $b_j$) $>$ t}
							\State{\textbf{Eliminar}($b_j$, B)}
						\EndIf
					\EndFor				
				\EndFor
			\State{\Return {D}}
			\EndProcedure
		\end{algorithmic}
	\end{minipage}
}
\end{center}

\begin{figure}[]
	\begin{subfigure}{.5\textwidth}
		\centering
		\includegraphics[width=0.7\textwidth]{img/iou.png}
		\caption{IoU}
		\label{fig:IoU}
	\end{subfigure}
	\begin{subfigure}{.5\textwidth}
		\centering
		\includegraphics[width=1.1\textwidth]{img/NMS.png}
		\caption{Supresión de no máximos}
		\label{fig:NMS}
	\end{subfigure}
	\caption{(a) Calculo de Intersección sobre Unión. (b) Salida de la propuesta de objetos y el resultado después de NMS.}
		\label{fig:RP}
\end{figure}

\section{Multimodales} \label{sec:multimodales}

Nuestra experiencia del mundo es multimodal, vemos objetos y sus entornos, escuchamos música y ruidos, sentimos la textura de los distintos materiales, etc.  Inconscientemente asociamos una situación a los distintos estímulos que recibimos en ese momento y los relacionamos entre si, para genera una idea de lo que esta sucediendo. De esta manera, sabemos que si algo huele mal, lo más probable es que sepa igual, o podemos relacionar una imagen del campo con el sonido de los pájaros.\\

La modalidad se refiere a la forma en que algo sucede o se experimenta. Un problema de investigación se caracteriza como multimodal cuando incluye datos de distinta naturaleza. Para que la inteligencia artificial avance en la comprensión del mundo que nos rodea, necesita poder interpretar y relacionar estas distintas señales. Si bien la combinación de diferentes modalidades para mejorar el rendimiento parece una tarea ``atractiva'', en la práctica, es un desafío, ya que una ineducada combinación de distintos tipos de información, puede generar confusión y conflictos. 

La idea general es, a partir de dos objetos matemáticos distintos, uno de cada modal, poder transformarlos para lograr que pertenezcan a un tercer objeto que corresponde a la representación multimodal. Las imágenes suelen estar asociadas con etiquetas y explicaciones de texto. En este trabajo nos aprovechamos de esto y tratamos de encontrar un espacio común entre el vector que representan a la imagen del objeto y el que representa la sintaxis de la clase.

La idea de multimodales es utilizar un espacio compartido entre las representaciones de visión y del lenguaje. Para lograr esto, se utiliza  \textbf{Words embeddings} y \textbf{vectores con representaciones visuales}. Las primeras asignan a palabras una representación vectorial continua. Estos vectores se utilizan para medir similitudes semánticas y sintácticas entre palabras. Entre los modelos más famosos se encuentran Glove~\cite{pennington-etal-2014-glove} y Word2vec~\cite{mikolov2013efficient}. Por otro lado, para obtener los vectores visuales de una imagen se utilizan redes profundas. Entre los mejores modelos se encuentran VGG~\cite{simonyan2014very}, ResNet~\cite{resnet} e Inception~\cite{Szegedy_2015_CVPR}. La~\autoref{fig:EjemploZSD} describe como se utiliza la combinación de vectores de palabras y visuales para inferir un objetos nunca antes vistos por el modelo.\\

\begin{figure}[]
	\centering
	\includegraphics[width=0.7\textwidth]{img/Modelo.png}
	\caption{Descripción de la tarea de detección de objetos por disparo cero utilizando multimodales, donde los objetos ``Auto'', ``Camión'', ``Perro'' y ``Tigre'' se observan  durante el entrenamiento,  ``Gato'' y ``Camioneta'' son clases invisibles. El enfoque localiza estas clases no vistas aprovechando las relaciones del espacio semántico.}
	\label{fig:EjemploZSD}
\end{figure}


\chapter{Metodología}\label{cap:metodologia}

\section{Formalización de ZSD} \label{ssec:formalizaciondezsd}
Para formalizar ZSD denotamos el conjunto de las clases como $\mathcal{C} = \mathcal{S} \cup \mathcal{U}$, donde $\mathcal{S}$ son las clases vistas para entrenamiento y $\mathcal{U}$ las clases no vistas, utilizadas en la etapa de pruebas. Además se tiene que $\mathcal{S} \cap \mathcal{U} = \emptyset$. Aunque no es necesario definir el conjunto de clases de pruebas, ya que el modelo tiene que ser capás de detectar tanto clases vista como las no vista, esto se hace para poder tener una evaluación cuantitativa.

Denotamos a una imagen como $\mathcal{I} \in \mathbb{D}^{\mathcal{M} \times \mathcal{N} \times 3}$. Donde $\mathbb{D} = \{0,...,255\}$, $\mathcal{M}$  es el largo de la imagen, $\mathcal{N}$ el ancho. Esta es la forma en la que se representa cada pixel de la imagen en el formato \textbf{RGB}, donde se tiene 3 canales que caracterizan la intensidad de los colores rojo, verde y azul. 

Por cada imagen se provee un conjunto de cuadros delimitadores  $\mathbb{B} = \{b_0,...,b_k\mid b_i \in N^4\}$ y sus etiquetas asociadas como $\mathbb{Y} = \{y_0,...,y_k\mid y_i \in \mathcal{C}\}$. Para cada cuadro delimitador $b_i$ extraemos una característica profunda utilizando una red neuronal convolucional denotada como $\phi(b_i) \in \mathbb{R}^{D_1}$. 

Denotamos las incrustaciones semánticas $w_j \in  \mathbb{R}^{D_2}$ obtenido por algún modelo como Wor2Vec. El conjunto de todas las imágenes de entrenamiento se indica con $\mathcal{X}^s$, que contiene ejemplos de todas las clases de objetos visibles.  El conjunto de todas las imágenes de prueba que contienen muestras de clases de objetos invisibles se indica con  $\mathcal{X}^u$. En particular, no hay ningún objeto de clase invisible en $\mathcal{X}^s$, pero $\mathcal{X}^u$ puede contener objetos vistos.\\

El objetivo es encontrar una matriz de proyección $W_p$, tal que \[ \psi_i = W_p\phi(b_i) \:\:\:\mid\:\:\: W_P \in \mathbb{R}^{D_2 \times D_1},\:\:\: \psi_i \in \mathbb{R}^{D_2} \] Note que $\psi_i$ y las incrustaciones semánticas se encuentran en el mismo dominio. Como mencionamos en secciones anteriores, el espacio vectorial semántico, tiene una gran capacidad de capturar similitudes semánticas. Por lo cual, resulta clave encontrar una matriz que para cada cuadro delimitador se proyecte lo mas cerca posible a la incrustación semántica de su clase. 

El resultado es una función \[f : \mathcal{X}, W_p  \to \{y_0,...,y_k\mid y_i \in \mathcal{C}\} \quad \operatorname{con}\quad \mathcal{X} =  \mathcal{X}^s \cup \mathcal{X}^u\] con un riesgo empírico regularizado mínimo definido de la siguiente manera $\mathcal{R}$ de la siguiente manera: \[ \arg_{}\min_{f \in F} \mathcal{R}(f(x,W_p))\quad, \] donde $x \in \mathcal{X}^s$ durante el entrenamiento. La función de mapeo utilizada en la etapa de inferencia, tiene la siguiente forma \[ f(x,W_p) = \arg_{}\max_{y \in \mathcal{C}}\max_{b \in \mathbb{B}(x)} (F(x,y,b,W_p)) \quad,\] donde los $\mathbb{B}(x)$ es el conjunto de propuestas de la imagen $x$. Intuitivamente se buscan los cuadros delimitadores de mejor puntuación y se les asigna la categoría de objeto de puntuación máxima.

\section{Arquitectura y Diseño} \label{sec:arquitecturaydiseno}

\subsection{Arquitectura} \label{ssec:arquitectura}
\begin{figure}
	\centering
	\includegraphics[width=0.9\textwidth]{img/arquitectura.png}
	\caption{Arquitectura propuesta para ZSD, utilizando incrustaciones semánticas y visuales, ademas se muestra la dimensión en cada paso.}
	\label{fig:arqutectura}
\end{figure}
Como se menciono anteriormente, hemos decido basarnos en el modelo propuesto por Bansal \etal~\cite{bansal2018zero} para abordar el problema de ZSD. La arquitectura propuesta en este  trabajo se puede dividir en las siguientes etapas:
\begin{itemize}
	\item \textbf{Pre-procesamiento:} Se recorren todas las imagen de entrenamiento y se extrae una característica profunda por cada cuadro delimitador, utilizando una red neuronal convolucional. Éste se asocia con el vector semántico de la clase que tiene asignada el cuadro, que se puede obtener con modelos de incrustación de palabras previamente entrenados, como Glove o Word2vec. Esta etapa nos genera como salida dos listas 
	\[X = [\phi(b_0),...,\phi(b_k) \mid \phi(b_i) \in \mathbb{R}^{D_1}]\] 
	\[W = [w_0,...,w_k \mid w_i \in \mathbb{R}^{D_2}]\]
	
	\item \textbf{Entrenamiento:} Utilizamos el espacio de incrustación común (${R}^{D_2}$) para calcular una medida de similitud entre las proyecciones  $\phi(b_i)$ y los vectores densos de palabras $w_i$. Luego, se entrena la proyección usando una función de pérdida, que impone la restricción que el puntaje de la similitud de un cuadro delimitador, con su clase verdadera, debe ser más alto que el de otras clases. 

	La función de pérdida definida como: 
	
	\[\mathcal{L}(\psi_i, w_i) = \sum_{j \in \mathcal{S}, j\neq i} max(0, m - S_{ii} + S_{ij})\] 
	donde $m$ es el margen máximo, y $S_{ij}$ es la similitud entre la proyección $i$-$esima$ y la incrustación $j$-$esima$.
 
	También se agrega una función de pérdida de reconstrucción como sugieren Kodirov \etal~\cite{kodirov2017semantic}. Se utilizan las características del cuadro delimitador proyectadas para reconstruir las características profundas originales, y calcular la pérdida de reconstrucción como la distancia $L2$  entre la característica reconstruida y la característica profunda original.
	\[\mathcal{L}_r = \Vert{\phi(b_i) - \psi_iW_p^T}\Vert^2 \] 
	Luego, definimos $\lambda$ como un coeficiente de ponderación que controla la importancia del primer y segundo término, que corresponden a las pérdidas de proyección y reconstrucción respectivamente. Por lo cual la función de perdida total es: 
	\[\mathcal{L}_t = \lambda \mathcal{L} + (1-\lambda) \mathcal{L}_r \]
 	 
 	En la \autoref{fig:arqutectura} se puede apreciar la arquitectura completa propuesta en este trabajo.
 	 
 	\item \textbf{Evaluación:} Por cada imagen de entrenamiento se genera un conjunto de propuestas de cuadros delimitadores. Luego, se eliminan todos los que no tienen un puntaje de confianza mayor a un umbral $t$. Para cada cuadro obtenido se computa la característica profunda $\phi(b_i)$ y utilizando la matriz de proyección $W_p$, se predicen las característica semánticas. Por último, se calcula la similitud con las características semántica de todas clases invisibles, asignando al cuadro delimitador la que tenga mayor puntaje.
\end{itemize}

Es común que en la detección de objetos se incluya una clase de fondo, para obtener un detector robusto que pueda discriminar eficazmente entre objetos de primer plano y objetos de fondo. En ZSD, esto no es un problema trivial, ya que no se sabe si un cuadro de fondo incluye elementos como cielo, tierra, bosque, etc. o una instancia de una clase de objeto invisible. En muchos trabajos se proponen distintas técnicas para abordar este problema, pero no presentan mejoras en evaluaciones cuantitativas. Es por esto que no se incluye una arquitectura que discrimine cuadros de fondos.

\subsection{Conjuntos de datos} \label{ssec:conjuntosdedatos}
\textit{Common Objects in Context} (COCO) es una base de datos que tiene como objetivo ayudar en la investigación de detección de objetos, posee varias caracteristicas como segmentación de instancias, subtítulos de imágenes y localización de puntos clave de personas. Este conjunto de datos contiene 91 tipos de objetos o  clases, con un total de 2.5 millones de instancias etiquetadas en 328.000 imágenes.

La gran cantidad de instancias de objetos y de categorías, resulta en un conjunto ideal para entrenar y evaluar modelos de ZSD. Además, la mayoría de la imágenes consta de una gran cantidad de objetos, a diferencia de conjuntos como Visual Genome. Esto generan un contexto en el que varios objetos se relacionan y se superponen, emulando de una mejor manera situaciones de la vida real. 

En este trabajo se utilizan las imágenes de entrenamiento del conjunto COCO 2014 e imágenes del conjunto de validación para realizar las pruebas de ZSD.
\begin{figure}
	\begin{center}
		\centering
		\includegraphics[width=1\textwidth]{img/data_set.png}
		\caption{División de las clases para entrenamiento (verde) y pruebas (azul).}
		\label{fig:data_set}
	\end{center}	
\end{figure}

Como COCO no provee una separación de los datos para evaluar modelos de ZSD, es necesario crear una forma de dividir las clases en vistas e invisibles. Esta separación resulta de suma importancia, ya que se debe cumplir que para todo objeto del conjunto prueba, exista otro de aspecto similar que este presente durante el entrenamiento. Además, no se puede encontrar ningún objeto de prueba en los datos de entrenamiento. Para esto, se aprovecha de que COCO tiene agrupadas las clases por ``Clases superiores'', y de cada uno de estos grupos se elige de forma aleatoria un 70\% de clases para entrenamiento y un 30\% para pruebas. Es decir, 47 y 18 clases respectivamente, de un total de 65 clases de COCO 2014. En la \autoref{fig:data_set} se puede observar el resultado de esta división. Por último se eliminaron todas las imágenes de entrenamiento que contengan al menos una instancia de las clases de prueba. Esto resulta en 42564 imágenes con 261258 instancias de entrenamiento, y 3008 imágenes con 10878 instancias de prueba. 

Bansal \etal~\cite{bansal2018zero}, divide el conjunto de datos de manera similar, utilizando la misma cantidad de clases para las etapas de prueba y entrenamiento. Pero la diferencia radica en que utiliza los vectores densos de palabras para agrupar las clases, utilizando la  similitud coseno entre los vectores como métrica. Por último, elige de forma aleatoria las clases visibles e invisibles de cada grupo. En este trabajo también se utiliza esta separación, para logra una comparación de modelos mas justa.

\begin{figure}[H]
	\begin{center}
	\begin{subfigure}{.3\textwidth}
		\includegraphics[width=1\textwidth]{img/cifar-zsd-test400.jpg}
		\label{fig:ex1}
	\end{subfigure}
	\begin{subfigure}{.3\textwidth}
		\includegraphics[width=1\textwidth]{img/cifar-zsd-test379.jpg}
		\label{fig:ex2}
	\end{subfigure}
	\begin{subfigure}{.3\textwidth}
		\includegraphics[width=1\textwidth]{img/cifar-zsd-test283.jpg}
		\label{fig:ex3}
	\end{subfigure}
	\caption{Ejemplos de imágenes del conjunto de datos CIFAR-ZSD.}
	\label{fig:CIFAR-ZSD}
	\end{center}
\end{figure}

COCO puede resultar pesado en término computacional. Para soluciona esto se creó un conjunto de datos sintético basado en CIFAR-100 datasets, el cual denominamos CIFAR-ZSD. Éste consta de imágenes localizadas, rotadas y re-escalada aleatoriamente con un fondo de otra imagen (algunos ejemplos se pueden ver en la \autoref{fig:CIFAR-ZSD}). Con esto se intenta simular imágenes reales en la cual un objeto puede aparecer con distintos aspectos y escalas. Este conjunto esta dividido para que ninguna instancia de prueba  aparezca en el conjunto de entrenamiento.

Aunque este conjunto resulta muy útil para hacer pruebas de modelos, no es bueno para reportar métricas reales, pero en combinación con COCO, que si lo es, ayudan a enfrentar el problema de ZSD de una forma mas práctica.

\subsection{Detalles de la implementación} \label{ssec:detallesdelaimplementacion}

El paper de Bansal \etal~\cite{bansal2018zero}, carece de una implementación de acceso público, por lo cual, se realizó una implementación propria, basándose en los detalles que se pueden extraer del documento publicado. Fue necesario realizar algunas suposiciones, pero también, nos otorgo flexibilidades a la hora de codificar. Se buscó obtener resultados lo más cercanos posibles a los reportados, siendo fiel a la información disponible. Para esto se decidió utilizar Python 3 con el framework Keras ejecutadose sobre TensorFlow.\\

Primero se realiza un preprocesamiento a todas las imagines del conjunto de datos de entrenamiento, que consiste en generar propuestas de objetos utilizando \textit{Edge Boxes} o \textit{Selective Search}. En la \autoref{cap:experimentos} se muestran los resultados de cada algoritmo. Luego por cada propuesta se calcula la intersección sobre unión con todos los cuadros delimitadores verdaderos. Si el IoU $> 0.5$ con algún cuadro verdadero, se guarda la propuesta y se la asocia con la clase del cuadro delimitador verdadero. Este preprocesamiento aumenta significativamente la cantidad de instancias para la etapa de entrenamiento, que resulta en un total de 1.436.835 instancias para COCO y 137.204 para CIFAR-ZSD. El siguiente paso consiste en generar el vector de características visuales y semánticas de cada cuadro delimitador. Como las CNN, utilizan un tamaño de entrada fijo, es necesario rescalar todos los cuadros.  Para \textit{VGG16} utilizamos $224 \times 224$, y en \textit{Inception ResNet V2} $299 \times 299$, que son las dimensiones por defecto. El tamaño del vector de salida de cada red es de 512 en VGG y 1536 ResNet. Para ambas usamos pesos preentrenados en Imagenet. Este paso es diferente a como se hace en el artículo de Bansal \etal~\cite{bansal2018zero}, ya que en este trabajo el modelo se entrena de extremo a extremo, es decir los pesos de la CNN se ajustan en la etapa de entrenamiento. Para obtener las características semánticas, utilizamos \textit{Word2Vec} previamente entrenado por \textit{Google News 300}. Este incluye vectores para un vocabulario de 3 millones de palabras y frases, entrenado en aproximadamente 100 mil millones de palabras de un conjunto de datos de Google News. La longitud del vector resultante es de 300. Luego se guardan dos archivos por cada imagen, en uno se encuentran los vectores visuales de cada clase y en el otro los semánticos.\\

Por último, para el entrenamiento, se creo una red con una sola capa oculta, una capa de entrada del tamaño del vector de características visuales, y una capa de salida de la dimensión de las características semánticas. Lo cual resulta en 153.900 parámetros entrenables con \textit{VGG16} y 461.100 con \textit{Inception ResNet V2}. Para esta etapa, se utilizó el optimizador Adam, un taza de aprendizaje de 10e-3, un tamaño del lote de 64 muestras y no se uso ninguna activación. Para la función de perdida se uso un lambda de 10e-3 y un margen máximo de 1.\\


\section{Experimentos}\label{cap:experimentos}

\subsection{Experimentación con propuesta de objetos} \label{ssec:experimentacionconpropuestadeobjetos}
Como se mencionó anteriormente, el número de propuestas es un parámetro clave. Algunas métricas son muy sensible a la cantidad de propuestas, afectando así los resultados finales. Esto se observó cuando se obtuvieron las primeras métricas, donde los valores estaban muy lejos de los esperados, y a medida que se aumentaba la cantidad de propuestas, los resultados empeoraban. Por este motivo, se probaron dos algoritmos (\textit{Edge Boxes} y \textit{Selective Search}), con algunas combinaciones de sus parámetros, con el objetivo de obtener una cantidad de propuestas que se superponga con el mayor número de objetos sin afectar las métricas.\\

Para no sesgar el experimento con los datos de prueba, se definió la metodología de la siguiente manera: por cada imagen de entrenamiento se corrió el generador de propuestas, y se calculó el tiempo y la cantidad de cuadros verdaderos que tenían un IoU $> 0.5$, con algún cuadro verdadero. El tiempo es un parámetro importante ya que algunos algoritmos soy muy lentos y resultan imposible de usar. 

Como se puede observar en la \autoref{tabla:edgeVSselct}, \textit{Selective Search} obtiene una mayor cantidad de superposición, pero con un número exageradamente grande de propuestas, por lo que la mejor opción es usar \textit{Edge-boxes}. En cuanto número de propuestas totales, resulta mas conveniente entre 100 y 500 propuestas como máximo, ya que al aumentar este numero no se generan mejoras en superposición pero si aumenta el número de propuestas. Si tenemos en cunta el tiempo, resulta mejor \textit{Edge-boxes}, ya que demora una fracción de lo que tarda \textit{Selective Search}.\\

\begin{table}[]
	\centering
	\resizebox{12.5cm}{!} {
		\begin{tabular}{|l|c|r|r|r|c|r|}
			\hline
			\textbf{}                     & \multicolumn{4}{c|}{\textbf{Edge Boxes}}                                                                                                   & \multicolumn{2}{c|}{\textbf{Selective Search}}               \\ \hline
			\textbf{Algoritmo}            & \multicolumn{4}{c|}{\textbf{-}}                                                                                                            & \textbf{Single}         & \multicolumn{1}{c|}{\textbf{Fast}} \\ \hline
			\textbf{Numero de propuestas} & \textbf{100}                 & \multicolumn{1}{c|}{\textbf{500}} & \multicolumn{1}{c|}{\textbf{1000}} & \multicolumn{1}{c|}{\textbf{5000}} & \textbf{$\approx$ 5000} & \multicolumn{1}{c|}{\textbf{$\approx$ 1000}}  \\ \hline
			Tiempo promedio (s)           & \multicolumn{1}{r|}{0.11}    & 0,11                              & 0.12                               & 0,12                               & \multicolumn{1}{r|}{5,48}   &         1,41                           \\ \hline
			Propuetas Totales             & \multicolumn{1}{r|}{4.415.244} & 22.050.071                          & 43.802.935                           & 161.809.194                          & \multicolumn{1}{r|}{350.535.591}   &   95.643.172                                 \\ \hline
			Propuestas con IOU $> 0.5$    & \multicolumn{1}{r|}{86.233}   & 133.942                            & 155.584                             & 194.891                             & \multicolumn{1}{r|}{221.551}   & 203.563                                   \\ \hline
		\end{tabular}
	}
	\caption{Resultados de correr los distintos algoritmos de propuestas de regiones en los datos de entrenamiento. El número de propuestas verdaderas es 261.258.}
	\label{tabla:edgeVSselct}
\end{table}


\subsection{Experimentación con CNN} \label{ssec:experimentacionconcnn}
% para generar tablas de latex
\begin{figure}
	\centering
	\includegraphics[width=1\linewidth]{img/vgg-vs-resnet}
	\caption{Frecuencia de la similitud coseno de los vectores de caracteristicas visuales, ente la misma y distintas clases, para las CNN  \textbf{Inception ResNet V2} y \textbf{VGG16}.}
	\label{fig:vgg-vs-resnet}
\end{figure}

Se decidió analizar la CNN ya que el modelo final es muy dependiente de esta red y de su capacidad de extraer características visuales. Lo que se quiere aquí es que la red sea capaz de asociar las caracteristicas visuales de objetos similares, y diferenciar los elementos de distinta naturaleza. En otras palabras, el espacio resultante tiene que distribuirse de tal manera que, por ejemplo, las imágenes de los animales estén muy cerca y a su ves alejadas de vehículos o electrodomésticos, pero también mantengan una separación entre los distintos animales como perro y gato. Bansal \etal~\cite{bansal2018zero} propone utilizar \textit{Inception ResNet V2}, el problema de esta red es que puede resultar pesada en cuanto a tiempo de ejecución y memoria. Por este motivo se decidió intentar con \textit{VGG16}, que reduce el número de parámetros en las capas convolucionales y mejorar el tiempo de ejecución, además es una de la mas utilizada.\\

El experimento consistió en comparara miles de recuadros de 3 clases de entrenamiento, caballo, perro y camión.  Por cada cuadro se generó el vector de caracteristicas visuales. Luego se comparó utilizando la similitud coseno, entre todas las caracteristicas de caballo vs caballo, caballo vs camión y caballo vs perro. En la \autoref{fig:vgg-vs-resnet} se graficaron las frecuencias de los resultados para cada CNN. Con esto se intenta observar como se distribuyen en el espacio visual las distintas clases. Como se esperaba, la similitud entre entre animales es mas grande que con un vehículo. Se observó que para \textit{Inception ResNet V2} existe una mayor separación entre clases, aunque sus similitudes están mas dispersas. \textit{VGG16} parece tener una menor dispersión, pero la similitud coseno entre distintas clases tiene valores muy cercanos. Esto puede afectar de manera negativa ya que camión y caballo no poseen una gran diferencia y el modelo podría interpretarlo como clases similares.\\


\subsection{Definición de métricas} \label{ssec:definiciondemetricas}
Entre los diferentes conjuntos de datos anotados, utilizados por los desafíos de detección de objetos y la comunidad científica, la métrica más común utilizada para medir la precisión de las detecciones es el  \textit{Mean Average Precision (mAP)}, seguida por \textit{Recall}. Un problema que tienen la métricas en detección de objetos, es la fata de una implementación estándar para calcularlas. Además, aquellas implementaciones públicas están muy encapsuladas al código, y resulta muy difícil adaptarlo para medir rendimientos de modelos propios. Como ya se mencionó anteriormente, el código de Bansal \etal~\cite{bansal2018zero} no esta disponible, por este motivo fue necesario encontrar alguna implementación de estas métricas. A partir de estas búsqueda se encontraron varias opciones, sin embargo los resultados variaban mucho de un código a otro. Esto se debe a la falta de consenso en diferentes trabajos e implementaciones de AP, que es un problema al que se enfrentan las comunidades académicas y científicas, tal como se plantea en artículo de Padilla \etal~\cite{padilla2020survey}. Además, propone una definición y un código para estandarizar las métricas, de manera que se puedan comprar distintos modelos de una forma ``justa''. Por estos motivos decidimos utilizar este trabajo y su implementación para calcular nuestras métricas, aunque los resultados no den exactos a los reportados por Bansal \etal~\cite{bansal2018zero}.\\

Ahora definamos las métricas, basándonos en el trabajo \cite{padilla2020survey}. Primero es necesario estandarizar:
\begin{itemize}
	\item Falso negativo (\textbf{FN}): Para un cuadro delimitador verdadero no se obtuvo ninguna detección en absoluto, o una propuesta tiene IoU $> umbral$ con algún cuadro verdadero y no se predijo correctamente la clase.
	\item Falso positivo (\textbf{FP}): Una propuesta predijo correctamente la clase de un cuadro delimitador verdadero pero el IoU $< umbral$, o es un predicción duplicada, es decir, ya se marco otra con mayor IoU como \textbf{TP}, o se detecto un objeto inexistente con IoU $< umbral$ para todo cuadro verdadero.
	\item Verdadero positivo (\textbf{TP}): Una propuesta predijo correctamente la clase y obtuvo un IoU $> umbral$ con algún cuadro verdadero.
	\item Verdadero negativo (\textbf{TN}): Esto sólo tiene sentido si se quisiera medir propuestas que no tienen un IoU $> umbral$ con todos los cuadros verdaderos, y además se predijo como clase de fondo. Pero en este trabajo no es utilizada.
\end{itemize}
El umbral por lo general es 0.5, pero se puede cambiar para exigir que tenga una mayor superposición.\\

La \textit{Recall}, también conocida como sensibilidad, mide la probabilidad de que los objetos verdaderos (los que se encuentran en la imagen) se detecten correctamente, viene dado por: 

\begin{equation}
	\label{eqn:recall}
	Recall =\frac{TP}{FN+TP}
\end{equation}

El trabajo de Bansal \etal~\cite{bansal2018zero}, define \textit{Recall} de la siguiente manera: 
\begin{center}
	\textit{``Un cuadro delimitador predicho se marca como verdadero positivo solo si tiene una superposición de IoU mayor que un cierto umbral t con un cuadro delimitador de verdadero y no se ha asignado ningún otro cuadro delimitador de mayor confianza al mismo cuadro de verdadero. De lo contrario, se marca como falso positivo.''}\\
\end{center}

Según esta definición solo se tienen en cuenta los objetos que tuvieron al menos una propuesta con un IoU $> 0.5$, y el resto quedan fuera del cálculo de esta métrica. Esto genera una diferencia enorme entre los resultados calculados con esta definición y con los obtenidos usando la \autoref{eqn:recall}. Esto dificulta la tarea de comprar con otros modelos, con lo cual es en este trabajo reportamos ambas formas. 

Bansal \etal~\cite{bansal2018zero} además calcula una variación denominada \textit{K@Recall}, donde sólo se tienen en cuentan las \textit{K} mejores propuestas basándose en la confianza de la predicción y el resto son descartadas.\\


\textit{AP}, es una métrica popular para evaluar la precisión de los detectores de objetos mediante la estimación del área bajo la curva (AUC), que viene dada por la relación de la \textit{precisión} y la \textit{recall}. Donde la precisión consiste en medir el porcentaje de predicciones positivas correctas entre todas las predicciones realizadas y se define como:

\begin{equation} 
\label{eqn:precision}
Precision =\frac{TP}{FP+TP}
\end{equation}


Para dibujar la curva AUC necesitamos obtener múltiples pares de valores de \textit{precisión} y \textit{recall}, esto se logra cambiando un límite de puntuación. Este limite trata como un falso positivo o todas aquellas propuesta que tengan un puntaje de confianza menor.

Para entender mejor supongamos un limite tal que genera un numero de FP bajo, la \textit{precisión} será alta. Sin embargo, en este caso, se pueden pasar por alto muchos aspectos interesantes de analizar, como por ejemplo un numero de FN alto y por lo tanto una \textit{recall} baja. Pero si uno baja el limite se aceptaran más positivos y la \textit{recall} aumentará, pero el numero FP también puede aumentar, disminuyendo la \textit{precisión}. De esta manera a media que aumentamos la \textit{recall} (bajamos el limite) la \textit{presision} se tiene que mantener alta. Por esto una área alta bajo la curva (AUC) tiende a indicar tanto una alta \textit{recall} como una alta \textit{precisión}.


Se define \textit{mAP} para la detección de objetos como el promedio del AP calculado para todas las clases. Por lo general, se indica sobre que IoU se calcula, puede ser un único valor, como por ejemplo mAP@0.5, o un conjunto de umbrales, como \textit{mAP@[x, y]} promediando el valor de \textit{mAP} para cada IoU. El trabajo de Bansal \etal~\cite{bansal2018zero} reporta \textit{mAP}, pero no indica sobre que IoU se calcula, por lo cual se asume que se utilizo un valor de 0,5. Muchos trabajos que utilizan COCO, reportan \textit{mAP@[.5, .95]}. Esta métrica resulta muy útil si se quiere comparar rendimientos entre distinto trabajos.


\subsection{Detalles de metodología de evaluación} \label{ssec:detallesdemetodologiadeevaluacion}
El principal experimento consistió en replicar los resultados de Bansal \etal~\cite{bansal2018zero}, aunque no se realizaron exactamente los mismos experimentos, ya que esto no aportaría nada nuevo. Así que, se decidió analizar sus resultados y solo replicar los que consideramos indispensable y que sería un buen punto de partida. Por ejemplo, los experimentos con clases de fondo, no obtuvieron buenos resultados, en comparación con los que no la utilizan, por este motivo no lo replicamos.\\

Ahora definamos la metodología de evaluación. El primer paso consiste generar propuestas para cada imagen, luego cada cuadro propuesto es reescalado al tamaño de la capa de entrada que tiene la CNN, y se le extrae el vector de características visuales. Después, se utiliza el modelo entrenado para inferir el vector de características semánticas, y se calcula la similitud coseno con los vectores semánticos de todas las clases o solo las invisibles, dependiendo si se quiere evaluar GZSD o ZSD. Aquella clase que obtenga el mayor puntaje es asignada a la propuesta. También, se guarda la puntuación como la confianza de predicción.  Por último, se agrupan todas las propuestas que se tengan asignada la misma clase y se corre un algoritmo de supresión no máxima. Este elimina las predicciones repetidas y retorna las mejores propuestas de cada grupo. Al final obtenemos como resultado un conjunto de propuestas, sus clases y su respectivo puntaje. Estos datos se guardan en un archivo y luego se corre la implementación de Padilla \etal~\cite{padilla2020survey}, para obtener los resultados de las métricas.\\


\chapter{Análisis de resultados} \label{cap:analisideresultado}
En este capitulo se analizaran los distintos resultados obtenidos por nosotros en los experimentos detallados en el capitulo \ref{cap:experimentos}, se comparan con los de Bansal y se analiza el por que de las diferencias obtenidas.
\section{Resultados cuantitativos} \label{sec:resultadoscuantitativos}

Esta sección desarrolla de una forma numérica los resultados obtenidos por los distintos modelos y en las distintas métricas definidas en el capitulo anterior. Desde que se empezó con este trabajo, se presentaros varios documentos sobre ZSD, resulto indispensable comparar con un trabajo mas actual. El motivo de esto es solo ilustrar la evolución que obtuvo ZSD en los últimos años.\\

\begin{table}[]
	\centering
	\resizebox{12.5cm}{1cm} {
		\begin{tabular}{|l|c|c|c|c|c|}
			\hline
			\multicolumn{1}{|c|}{\textbf{Metrica}} & \textbf{Baseline Bansal} & \multicolumn{1}{l|}{\textbf{Mejor resultado Bansal}} & \textbf{Nuestros con VGG} & \textbf{Nuestros con ResNet} & \multicolumn{1}{l|}{\textbf{Mejor  resultado de \cite{rahman2020zero}}} \\ \hline
			\textbf{100@Recall (Bansal)}           & 22.14                    & 27.19                                                & 26.34                     & 28.91                        & -                                                                      \\ \hline
			\textbf{100@Recall}                    & -                        & -                                                    & 2.72                      & 4.26                         & 12.27                                                                  \\ \hline
			\textbf{mAP@0.5}                       & 0.32                     & 0.54                                                 & 0.19                     & 0.23                        & 5.05                                                                   \\ \hline
			\textbf{mAP@[.5, .95]}                 & -                        & -                                                    & 0.17                     & 0.21                        & -                                                                      \\ \hline
		\end{tabular}
	}
	\caption{Resutados obtenidos por Bansal  \cite{bansal2018zero}, nosotros y Rahman \cite{rahman2020zero}.}
	\label{tab:resultadosZSD}
\end{table}

El Cuadro \ref{tab:resultadosZSD}, muestra los valores de las métricas 100@Recall, en la versión desarrollada por Bansal y la de Padilla, tambien se muestran los resultados para mAP@0.5 y mAP[.5, .95]. Se analizan 2 modelos propuestos por nosotros utilizando \textbf{VGG16} y \textbf{Inception ResNet V2} y el mejor resultado presentado por el trabajo de Rahman Shafin \cite{rahman2020zero}. Se eligió este documento ya que aborda de una manera similar el problema de ZSD. Pero presenta algunas mejoras y un modelo mas complejo.\\

El mejor resultado de Bansal el cual denomina \textit{Densely Sampled Embedding Space} \textbf{DSES} consiste en aumentar el procedimiento de entrenamiento con datos adicionales de fuentes externas que contienen casillas que pertenecen a clases distintas de las clases invisibles. Obtiene 27.19 puntos, el cual nuestro baseline iguala e incluso supera con 26.34 con \textbf{VGG} y 28.91 usando \textbf{ResNet}. El motivo de esto, se debe a que en un principio se calculaba 100@Recall, con una implementación distinta de Bansal. Al no poder igualar sus resultados, se experimento mucho con los distintos parámetros de cada etapa. Luego al calcular con una implementación como define Bansal, esta experimentación resulta en una pequeña mejora en el baseline.

En cuanto mAP, Bansal no aporta mucha información solo reporta los valores para COCO. La implementación es desconocida, por lo cual asumimos que lo reportado es mAP@0.5. De inmediato se puede observar que los valores son muy bajos 0.54. Aunque nosotros obtuvimos resultados aun mas bajos, con un máximo de 0.23 puntos. La justificación de esta diferencia es, que Bansal solo tiene en cuenta aquellos cuadros verdaderos que tienen un IOU $> 0.5$ con alguna propuesta. Pero nosotros utilizamos todos los cuadros verdaderos, independientemente si se detectaron o no. 
	
Si aislamos nuestros resultados, sabiendo que es solo un baseline y un modelo muy sencillo sin utilizar ningún tipo de información extra, obtuvimos valores aceptables para 100@Recall con 6.26 puntos. Pero resulta importante resaltar su bajo desempeño en mAP. El principal motivo de este resultado es el algoritmo para proponer objetos. Se necesitan muchos cuadros delimitadores y aun así no se alcanza a superponer con la mayoría de cuadros verdaderos.

Comparando con un trabajo mas actual, nuestros resultados son coherentes. Pero claro esta que obtiene valores 3 veces mejores en  100@Recall y un excelente desempeño en mAP con 5.05 puntos.\\

También se calcularon las métricas para el conjunto de datos CIFAR-ZSD. Pero fueron necesario algunas mejoras para adaptarse a las diferencias con COCO. Primero se utilizo una tamaño de la entrada de la CNN mas pequeña de 32x32. Esto se debe a que las imágenes no tienen una gran resolución. También, se reduzco considerablemente el numero máximo de de propuestas, del orden de 50. La justificación de esto es que los objetos sobresalen del fondo de la imagen y es mas fácil su detección.

Dicho esto, los resultados obtenidos fueron, 7.46 en 100@Recall (implementación de Rafael) y 0.72 para mAP@0.5. Estos valores al contrario de los reportados para COCO, se ven influenciado por la calidad de la imagen, lo que hace muy difícil de diferenciar el aspecto visual de las distintas clases. Pero ayuda a entender como influye en las métricas el generar una cantidad reducida de propuestas y que tengan una porcentaje de superposición alta.\\

Por ultimo, se analizaron los resultados en el desafió de \textbf{ZSDG}. La configuración generalizada de aprendizaje de disparo cero es más realista que la configuración de disparo cero discutida anteriormente, porque tanto las clases visibles como las invisibles están presentes durante la evaluación.

El Cuadro \ref{tab:resultados-zsdg}, se muestran los resultados para ZSDG. Bansal solo reporta 100@Recall. Este modifica el método de evaluación, calculando la clase visible e invisible mas probable y si la no vista supera un umbral t, le asigna esa clase caso contrario utiliza la clase vista. Nosotros por otro lado utilizamos el mismo método de evaluacion, lo única diferencia es que se tienen en cuenta todas las clases. Dicho esto, Bansal obtiene un promedio de 15.17 y nosotros obtenemos una mejora de 4 puntos. En cuanto a mAP, conseguimos una media de 0.13.\\

Si comparamos los resultados de ZSD vs ZSDG, se observa una baja en las métricas. El motivo es que las clases vistas al estar en entrenamiento, obtienen un mejor puntaje en la etapa de evaluacion que las clases invisibles. Por esto muchos objetos que en la configuración anterior, predecía correctamente ahora una clase visible obtiene mejor puntaje.\\

Estos resultados fueron, calculados sobre la división de clases propuesta por Bansal ya que ambos trabajos con cuales comparamos utilizan esta. Pero también se corrieron las evaluaciones con la partición propuesta en este trabajo. Los resultados finales fue de un 4\% y 7\% menos. Esta reducción se debe a que el documento de Bansal utiliza como criterio de división los vectores semánticos de las clases, esto afecta positivamente ya que es el mismo espacio utilizado para inferir las clases inviables.

\begin{table}[]
	\centering
	\resizebox{12.5cm}{1.2cm} {
	\begin{tabular}{|l|c|c|c|}
		\hline
		\multicolumn{1}{|c|}{\multirow{3}{*}{Modelo}} & \multicolumn{3}{c|}{ZSDG}                                                       \\ \cline{2-4} 
		\multicolumn{1}{|c|}{}                        & Clases vistas             & Clases Invisibles        & Media                    \\ \cline{2-4} 
		\multicolumn{1}{|c|}{}                        & mAP/Recall Bansal/Recall  & mAP/Recall Bansal/Recall & mAP/Recall Bansal/Recall \\ \hline
		Bansal                                        & -/15.02/-                 & -/15.32/-                & -/15.17/-                \\ \hline
		Nuestro Baseline                              & 0.15/20.98/3.36           & 0.11/18.53/2.49          & 0.13/19.75/2.9           \\ \hline
		Mejor  resultado de \cite{rahman2020zero}     & 13.93/-/20.42             & 2.55/-/12.42             & 4.31/-/15.45             \\ \hline
	\end{tabular}
	}
	\caption{Resultados obtenidos, en el desafió ZSDG, para los modelos de Bansal \cite{bansal2018zero}, nuestro (ResNet) y Rahman \cite{rahman2020zero}}
	\label{tab:resultados-zsdg}
\end{table}

\section{Resultados cualitativos} \label{sec:resultadoscualitativos}

La Figura [N] muestra las detecciones del modelo propuesto en el conjunto de datos MSCOCO. Los cuadros azules muestran detecciones correctas y los cuadros rojos muestran falsos positivos. Estos ejemplos confirman que los modelos propuestos son capaces de detectar clases invisibles sin observar ninguna muestra durante el entrenamiento. Para estas pruebas se reduzco el numero de propuestas al orden de 10.

\chapter{Conclusiones y trabajo futuro} \label{cap:conclusiones}

\section{Conclusiones y aportes} \label{sec:conclusionesyaportes}
Durante este trabajo se analizo de forma detallada y objetiva el desafiante problema de ZSD. Desde un principio, sabíamos que era un campo de investigación nuevo y que esto dificultaría el desarrollo de esta tesis. Los objetivos se fueron modificando en transcurso del tiempo. Aun así logramos genera un aporte en esta disciplina.\\

El primer paso de esta tesis fue la lectura y análisis de los distintos trabajos sobre ZSD. Un aspecto, que tenían la mayoría, es la utilización de incrustaciones visual y semánticas para abordar el problema. Por este motivo decidimos utilizar esta metodología, para proponer un modelo base, basándonos en el trabajo de Bansal \cite{bansal2018zero}. La falta de una implementación, nos obligo a profundizar en cada etapa del desarrollo, reduciendo las metas planteadas. Pero aportando un conocimiento mas detallado de la solución. 

Si bien el modelo base no fue propuesto por nosotros, aportamos detalles que surgieron de nuestra experimentación y como estos afectan a los resultados. También, se analizaron aspectos que al nuestro entender fueron ignorados por el trabajo original, pero resulta cruciales.

Otro aspecto importante analizado son los conjuntos de datos. Aun no existe uno especificado en el problema de ZSD, ni una adaptación consensuada de alguno ya existente. Proponemos una manera sencilla de dividir MSCOCO y la comparamos con la división del trabajo original que al parecer beneficia al modelo considerablemente.

Pero lo que creemos el mayor aporte es el análisis de la métricas. Esto es una gran debilidad en los trabajos relacionados actuales, que impide una comparación, justa y correcta. Debido a que los resultados no eran los esperados y luego de probar todos los cambios y mejoras posibles, nos encontramos con una definición ambigua que genero una mala interpretación. Decidimos investigar sobre las métricas y encontramos muchos documentos que señalaban el problema que tuvimos. Pero el trabajo de Padilla \cite{padilla2020survey} sobre sale a los demás, posea una clara definición de las métricas y una implementación fácil de utilizar, y se recomida su uso.

Si bien los resultados obtenidos no son los mejores y están por abajo de la expectativa. Aportan una idea de lo que son capaces los modelos de ZSD. Creemos que son mas transparente y detallados al trabajo original, haciendo posible agregar mejoras y ver su progreso de una manera cuantitativa.

Por ultimo, la comparación con modelos mas actuales, hace evidente el constante esfuerzo en este campo y sus mejoras demuestra que ZSD no es una fantasía.


\section{Trabajo futuro} \label{sec:trabajo futuro}

Teniendo en cuenta los resultados obtenidos en esta tesis, existen distintas alternativas para seguir profundizando. Las cual podemos dividir en tres grupos. 

El primero, es mejorar el algoritmo que genera propuestas, esto afecta sobre todo a la etapa de evaluacion. Trabajos actuales utilizan varios generadores simultáneamente, obteniendo algunas mejoras. Otros plantean aumentar el numero de propuestas considerablemente y utilizar un criterio  mas complejo a la supresión no máxima, para eliminar casillas repetidas y de fondo.

El segundo, surge de la simplicidad del modelo propuesto. Existe muchas formas de mejorarlo, algunas ideas pueden ser. Considerar la fusión de diferentes vectores de palabras (\textbf{Word2vec} y \textbf{GloVe}). Utilizar otro espacio que no sea el semantico y mapear ambas caracteristicas a este. Otro cambio, que tambien afecta el punto anterior es utilizar una única red unificada de extremo a extremo, capas de predecir  la ubicación de diferentes objetos y clasificarlos como lo es \textbf{Faster R-CNN}.

Por ultimo, resulta interesante suavizar el problema de ZSD. En ves de clasificar por clase, se pude utilizar sub-clases. Si bien esto no es una mejora, puede ayudar a entender si el modelo realmente esta relacionando objetos, ya que no es lo mismo confundir un perro con un auto que con lobo.

\thispagestyle{empty}

%----------------------------------------------------------------------------------------
%	BIBLIOGRAFÍA
%----------------------------------------------------------------------------------------
\backmatter
\nocite{*}
\bibliographystyle{ieeetr}
\bibliography{bibliografia.bib} %Aquí ponen el nombre del archivo .bib

\end{document}