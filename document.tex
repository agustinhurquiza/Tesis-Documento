\documentclass[12pt,twosided]{book}
\usepackage[paperwidth=17cm, paperheight=22.5cm, bottom=2.5cm, right=2.5cm]{geometry}
\usepackage{amssymb,amsmath,amsthm} %paquete para símbolo matemáticos
\usepackage[spanish]{babel}
\usepackage[utf8]{inputenc} %Paquete para escribir acentos y otros símbolos directamente
\usepackage{enumerate}
\usepackage{graphicx}
\usepackage{subcaption}
\usepackage{multirow}
\usepackage{float}
\usepackage{tikz-qtree,tikz-qtree-compat}
\usepackage{multicol}
\usepackage{pdfpages}
\usepackage{algpseudocode}
\usepackage[nottoc]{tocbibind}
\usepackage[pdftex,
            pdfauthor={Agustin Horacio Urquiza Toledo},
            pdftitle={Zero-Shot Object Detection},
            pdfsubject={Ciencias de la computación},
            pdfkeywords={PALABRAS CLAVE},
            pdfproducer={Latex con hyperref},
            pdfcreator={pdflatex}]{hyperref}
\usepackage[colorinlistoftodos]{todonotes}
\usepackage[square,sort,comma,numbers]{natbib}

\graphicspath{{img/}} % En qué carpeta están las imágenes

\begin{document}

%----------------------------------------------------------------------------------------
%	COMANDOS PERSONALIZADOS
%----------------------------------------------------------------------------------------
\newcommand{\etal}{\textit{et al}.}
\newcommand{\todos}[1]{{\color{red}[TODO: #1]}}
% Cambiar Cuadros por Tablas y lista de...
\renewcommand{\listtablename}{Índice de tablas}
\renewcommand{\tablename}{Tabla}

%----------------------------------------------------------------------------------------
%	PORTADA
%----------------------------------------------------------------------------------------
\title{Detección de objetos en imágenes mediante aprendizaje sin ejemplos} % Con este nombre se guardará el proyecto en writeLaTex

\begin{titlepage}
\begin{center}

\textsc{\Large Facultad de Matemática, Astronomía, Física y Computación}\\[1em]

\textsc{Universidad Nacional de Córdoba}

%Figura
\begin{figure}[h]
\begin{center}
\includegraphics[scale=0.2]{img/unc.jpg}
\end{center}
\end{figure}

\vspace{1em}

\textsc{\huge \textbf{Detección de objetos en imágenes mediante aprendizaje sin ejemplos}}\\[2em]

\textsc{Licenciado en Ciencias de la Computación}

\textsc{}

\textsc{\Large Agustin Horacio Urquiza Toledo}

\textsc{\large Director: Jorge Sanchez}

\end{center}

\begin{center}
\begin{figure}[h]
\begin{center}
\includegraphics[]{lic.png}
\end{center}
\end{figure}
\textit{Detección de objetos en imágenes mediante aprendizaje sin ejemplos por Agustin Horacio Urquiza Toledo se distribuye bajo una \href{http://creativecommons.org/licenses/by-sa/4.0/}{\color{blue}Licencia Creative Commons Atribución-CompartirIgual 4.0 Internacional.}}
\end{center}

\textsc{Córdoba, Argentina \hspace*{\fill} 2021}
\end{titlepage}


%----------------------------------------------------------------------------------------
%	AGRADECIMIENTOS
%----------------------------------------------------------------------------------------
\chapter*{Agradecimientos}
A la Universidad Nacional de Córdoba por haberme dado la
oportunidad de formarme. Quiero agradecer a mi tutor Dr. Jorge Sanchez, que me guió y me enseñó lo necesario para realizar este trabajo. A mi familia, y en especial a mis padres, Graciela Toledo y Horacio Urquiza, quienes sin entender del todo lo que hago están siempre apoyándome. A mis compañeros de la Licenciatura, que crecí junto a ellos no sólo en lo profesional sino también en lo personal. Y por último a mis amigos de la vida, que siempre me escucharon y me ayudaron a despejarme de tanto estudio.

%----------------------------------------------------------------------------------------
%	PREFACIO
%----------------------------------------------------------------------------------------
\chapter*{Resumen}

\pagestyle{plain}

En el año 2010 surgió la llamada ``revolución'' del aprendizaje profundo, y con esto, los métodos capaces de detectar objetos en una imagen progresaron considerablemente. Estos algoritmos o modelos fueron mejorando en cada año, hasta hoy en día, que alcanzaron un excelente rendimiento e innumerables aplicaciones. Pero estos poseen una limitación, necesitan tener una gran cantidad de imágenes anotadas, que en algunos casos resulta inviable. Para resolver este problema surgieron técnicas como la detección de objetos sin ejemplos (ZSD) por sus siglas en ingles \textit{Zero-shot Object Detection}. Este es un problema de investigación recientemente propuesto, que tiene como objetivo localizar y reconocer simultáneamente objetos de clases nunca antes vistas. Para suplir la falta de ejemplos de entrenamiento de algunas clases, se combinan distintos atributos de las imágenes y de las clases para inferir los objetos novedosos. Generalmente, se proponen modelos de ZSD como una combinación de características visuales y descripciones semánticas, aprendiendo un mapeo visual-semántico. 

En la actualidad existen modelos que obtienen información de las descripciones semánticas capaces de asociar clases similares con una gran eficiencia, en este trabajo se utiliza esta cualidad para poder transferir el conocimiento de las clases presentes en el entrenamiento a las que no estuvieron presentes.


En esta tesis, analizamos distintos artículos científicos y llevamos a cabo experimentos en el conjunto de datos COCO, con la idea de aportar una noción del estado actual en esta área.

\section*{Palabras claves}
Detección de objetos en imágenes mediante aprendizaje sin ejemplos, aprendizaje automático sin ejemplos, aprendizaje profundo, deep learning, zero-shot learning, zero-shot object detection.\\

\section*{Sistema de clasificación} 

Detección de objetos.

\chapter*{Summary}

In 2010 the deep learning revolution started and made possible the development of methods that are able to detect objects contained in images. These algorithms were improved with the years, and nowadays, they have achieved excellent performance and a great number of applications. However, these methods still have some limitations, mostly related to the great number of tagged images that need to be considered, which in some cases are impossible to obtain. In order to solve this problem, several techniques have been developed, such as the zero shoot detection (ZSD). This method has been recently proposed and aims to simultaneously localize and recognize objects of unseen classes. In order to compensate for the lack of ``examples of training'' of some classes, different attributes of the images and the classes are combined to infer the new objects. In general, ZSD models are designed as a combination of visual characteristics and semantic descriptions, which make it possible to learn a visual-semantic mapping.

Today, there are models that obtain information from semantic descriptions, and are able to associate similar classes in a very efficient way. This property is used in present work to transfer the knowledge of the classes seen during the training to the ones that were not present in it.

This thesis analyzes different scientific articles and presents experiments that were carried out within the data set COCO, with the objective of contributing to the current knowledge in this field. 

\section*{Keywords}
Deep learning, zero-shot learning, zero-shot object detection.

\section*{Computing Classification System} 

Object detection.


%----------------------------------------------------------------------------------------
%	TABLA DE CONTENIDOS
%---------------------------------------------------------------------------------------
\begingroup
\hypersetup{hidelinks}
\tableofcontents
\endgroup

%----------------------------------------------------------------------------------------
%	TESIS
%----------------------------------------------------------------------------------------
\mainmatter %empieza la numeración de las páginas
\pagestyle{headings}

%  Incluye los capítulos en el folder de capítulos

\chapter{Introducción y Motivación} \label{cap:intro}

\section{Historia} \label{sec:historia}
La detección de objetos es una de las áreas de la visión por computadora que está creciendo más rápidamente. Gracias al aprendizaje profundo, cada año, los nuevos algoritmos/modelos siguen superando a los anteriores. Aunque la visión por computadora recientemente tomó gran importancia (el momento decisivo ocurrió en 2012 cuando AlexNet ganó ImageNet), ciertamente no es un nuevo campo científico.\\

Uno de los artículos más influyentes en Visión Informática fue publicado por dos neurofisiólogos, David Hubel y Torsten Wiesel~\cite{hubel1959receptive}, en 1959. Su publicación, titulada \textit{``Receptive fields of single neurons in the cat’s striate cortex''}, en español ``Campos receptivos de neuronas individuales en la corteza estriada del gato'', describió las propiedades de respuesta central de las neuronas corticales visuales y como la experiencia visual de un gato moldea su arquitectura cortical. Los investigadores establecieron a través de su experimentación (\autoref{fig:ExpermentoHubelTorsten}) que existen neuronas simples y complejas en la corteza visual primaria, y que el procesamiento visual siempre comienza con estructuras simples como los bordes orientados y gradualmente identifica estructuras mas complejas. En la actualidad, este es el principio básico detrás del aprendizaje profundo.\\

\begin{figure}
	\centering
	\includegraphics[width=0.5\textwidth]{img/cat.jpg}
	\caption{Simple explicación del experimento realizado por David Hubel y Torsten Wiesel}
	\label{fig:ExpermentoHubelTorsten}
\end{figure}

Otro echo importante en la historia de la visión por computadora fue en 1957, Russell Kirsch y sus colegas desarrollaron un aparato que permitía transformar imágenes en cuadrículas de números que las máquinas de lenguaje binario podían entender. 

Poco tiempo después, en la década de 1960 fue cuando la IA se convirtió en una disciplina académica y algunos de los investigadores eran extremadamente optimistas sobre el futuro del campo. En este periodo, Seymour Papert, profesor del laboratorio de IA del MIT, decidió lanzar el Proyecto de Verano y resolver, en pocos meses, el problema de la visión artificial. Los estudiantes debían diseñar una plataforma que pudiera realizar automáticamente segmentación de fondo y extraer objetos no superpuestos de imágenes del mundo real. Claro esta que el proyecto no fue un éxito.  Hoy en día, cincuenta años después, todavía no se ha podido resolver la visión por computadora. Sin embargo, ese proyecto fue el nacimiento oficial de esta disciplina como campo científico. 

Los aportes mas influyentes en este campo empezaron a surgir a partir de los años 2000. En 2001 Paul Viola y Michael Jones~\cite{viola2001rapid} presentaron el primer detector de rostros que funcionó en tiempo real. Aunque no se basaba en el aprendizaje profundo, el algoritmo tenía una relación con éste, ya que, al procesar imágenes aprendió qué características podrían ayudar a localizar caras, inspirándose en el experimento de David Hubel y Torsten Wiesel. 

En 2006 comenzó la competencia de Pascal VOC que permitió evaluar el desempeño de diferentes métodos para el reconocimiento de objetos. Mas tarde, en 2010, siguiendo los pasos de Pascal VOC, se inició el concurso de reconocimiento visual a gran escala ImageNet (ILSVRC), cuya tasa de error durante 2010 y 2011, en el desafió de clasificación de imágenes, rondaba el 26\%.  En 2012, un equipo de la Universidad de Toronto ingresó a la competencia con un modelo de red neuronal convolucional (AlexNet)~\cite{krizhevsky2012imagenet} que cambió todo, dado que logró una tasa de error del 16,4\%. En los años siguientes, las tasas de error en la clasificación de imágenes en ILSVRC cayeron a un pequeño porcentaje, como se observa en la \autoref{fig:EvolucionILSVRC} y los ganadores, desde 2012, siempre han sido redes neuronales convolucionales.

\begin{figure}[H]
	\centering
	\includegraphics[width=0.7\textwidth]{img/imgnet-grafico.png}
	\caption{Evolución de los modelos propuestos en la competencia ILSVRC}
	\label{fig:EvolucionILSVRC}
\end{figure}

\section{Detectores y ZSD} \label{sec:detectoresyzsd}
La detección de objetos es un subproblema de la visión artificial, que estudia cómo detectar la presencia de objetos en una imagen. Debido a la complejidad de poder detectar todas las instancias de todos los posibles objectos en una imagen, se dividió en distintas tareas para disminuir la dificultad. 

Par explicar los distintos problemas, es necesario distinguir dos conjuntos. Por un lado, los datos de entrenamiento, que consta de las imágenes que se usan para entrenar el modelo con sus respectivas etiquetas, es decir, que objetos se encuentran en la imagen, localización de los objetos, descripción de la imagen, o cualquier información extra que requiera la tarea. Por otro lado, las imágenes de prueba, que es el conjunto donde se observará o medirá la eficiencia del modelo ya entrenado. 

Supongamos que las etiquetas solo cuenta con dos tipos de información, que clase de objeto es, es decir si es un perro, auto, persona, etc. y su localización en la imagen. A todas las clases de objetos que aparecen en los datos de entrenamiento las llamaremos clases visibles o vistas, y todas aquellas clase que no sea una clase vista las llamearemos invisible o no vista. Dicho esto, los distintos problemas son:

\begin{itemize}
	\item \textbf{Clasificación}: consta de un modelo capás de predecir si una clase específica esta presente en una imagen. 
	\item \textbf{Clasificación mas localización}: además de poder clasificar tiene que ser capas de ubicar el objecto en la imagen.
	\item \textbf{Reconocimiento de imagen}: predice que objetos perteneciente a las clases visibles están presente en la imagen. 
	\item \textbf{La detección de objetos}: además de reconocer objetos visibles, tiene que ser capás de localizar dichos objetos. 
	\item \textbf{Reconocimiento por disparo cero}: tiene que poder reconocer clases vistas y no vistas.
	\item \textbf{Detección de objetos por disparo cero} (\textbf{ZSD} por sus siglas en inglés): debe localizar y clasificar todas las instancias de objetos en la imagen, sin depender si es una clase vista o no.
\end{itemize}

\begin{figure}[]
  \centering
  \subcaptionbox{\tiny{\textbf{ Clasificación}}}{\includegraphics[width=1.5in]{img/expect_3.png}}\hspace{1em}%
  \subcaptionbox{\tiny{\textbf{Detección de objetos}}}{\includegraphics[width=1.5in]{img/expect_2.png}}
  \subcaptionbox{\tiny{\textbf{ZSD}}}{\includegraphics[width=1.5in]{img/expect_1.png}}
  \caption{Ejemplo de tareas de clasificación mas localización, detección de objetos y ZSD. En la escala de los verdes se encuentran las clases vistas \{Caballo, Árbol\}, y en rojo las clases invisibles \{Perro, Persona, Campera, Pantalón, Correa\}.}
  \label{fig:DetectoresYSZD}
\end{figure}

La \autoref{fig:DetectoresYSZD} muestra un ejemplo de las distintas tareas  mencionadas anteriormente.\\

Además de los problemas mencionados anteriormente, existen otros como la segmentación, que no desarrollaremos en este trabajo. Aquí, solo nos enfocaremos en ZSD y sus problemas asociados.
 
Existen muchas técnicas propuestas para resolver ZSD. Cuando se empezó a leer sobre este tema a fines del 2018, la mas utilizada consistía en emplear multimodales. Puntualmente existían tres trabajos en paralelos \cite{rahman2018zero}\cite{zhu2018zero}\cite{bansal2018zero} con una metodología similar. La idea de esta técnica es utilizar un espacio compartido ente las representaciones de visión y del lenguaje. Para lograr esto, se utiliza  \textbf{incrustaciones de palabras} y \textbf{vectores con representaciones visuales}. Las primeras asignan a palabras una representación vectorial continua. Estos vectores se utilizan para medir similitudes semánticas y sintácticas entre palabras. Entre los modelos mas famosos se encuentran Glove~\cite{pennington-etal-2014-glove} y Word2vec~\cite{mikolov2013efficient}. Por otro lado, para obtener los vectores visuales de una imagen se utilizan redes profundas. Entre los mejores modelos se encuentran VGG~\cite{simonyan2014very}, ResNet~\cite{resnet} e Inception~\cite{Szegedy_2015_CVPR}. La \autoref{fig:EjemploZSD} describe como se utiliza la combinación de vectores de palabras y visuales para inferir un objetos nunca antes vistos por el modelo.\\

\begin{figure}[]
	\centering
	\includegraphics[width=1\textwidth]{img/Modelo.png}
	\caption{Descripción de la tarea de detección de objetos por disparo cero utilizando multimodales, donde los objetos ``Auto'', ``Camión'', ``Perro'' y ``Tigre'' se observan  durante el entrenamiento,  ``Gato'' y ``Camioneta'' son clases invisibles. El enfoque localiza estas clases no vistas aprovechando las relaciones del espacio semantico.}
	\label{fig:EjemploZSD}
\end{figure}

\section{Motivación} \label{sec:motivacion}

Hoy en día, hay una gran cantidad de modelos, capaces de detectar objetos en una imagen, como son las redes YOLO o Faster R-CNN. Estos, como otros no mencionados, poseen una excelente rendimiento, pero tienen una gran limitación, necesitan una gran cantidad de imágenes anotadas, para cada clase que se quiere detectar. Conseguir un gran numero de anotaciones, pude resultar un gran desafió, ya sea por la naturaliza del problema o por los grandes costo que esto conlleva. Esta dificultad se intenta mitigar con ZSD dado que puede inferir objetos no anotados.\\ 

ZSD es una habilidad que los humanos ya tienen. De hecho, podemos aprender muchas cosas con solo un ``conjunto de datos mínimo". Por ejemplo, tendemos a diferenciar  variedades de la misma fruta o frutas de aspecto similar, aun si hemos visto muy pocas veces cada tipo de fruta. La situación es diferente para las máquinas. Necesitan muchas imágenes para aprender a adaptarse a la variación que se produce de forma natural en lo humanos. Esta habilidad proviene de nuestra base de conocimientos lingüísticos existente, que proporciona una descripción de alto nivel de una clase nueva o no vista y establece una conexión entre ella y las clases que ya conocemos.\\

Por unos minutos dejemos llevarnos por la imaginación y supongamos que se quiere crear un programa capas de reconocer todos los objeto en una imagen, pero objetos de cualquier índole, animales, plantas, artículos de limpieza, o cualquier cosa que se te venga a la mente. Seria casi imposible, si es que no lo es, generar un conjunto de datos que contenga una cantidad considerable de imágenes de todos los objetos posible. Esta idea puede sonar muy descabellada, o no, pero no se puede negar su potencial y su gran cantidad de usos como en interpretaciones de escenas, seguridad, etc. A medida que ZSD continúa desarrollándose, se espera ver más aplicaciones, como mejores recomendaciones y soluciones más avanzadas que marcan automáticamente el contenido inadecuado dentro de las redes sociales, como así también un fuerte desarrollo en el campo de la robótica.

\section{Estructura de la tesis} \label{sec:estructuradelatesis}

Esta tesis se estructura de la siguiente manera. En el~\autoref{cap:preliminares} se detallan los conceptos fundamentales utilizado a lo largo del trabajo. En el~\autoref{cap:metodologia} se comienza formalizando el problema de ZSD y se define la arquitectura empleada para resolverlo. Ademas se describe los conjuntos de datos utilizados para entrenar y medir el rendimiento del modelo, como así también los detalles de nuestra implementación. Por ultimo se define las distintas métricas utilizadas y describe los distintos experimentos realizados. Luego en el~\autoref{cap:analisideresultado} se analizan los resultados obtenidos y se compara con distintos trabajos. Por ultimo el~\autoref{cap:conclusiones} se escriben las conclusiones que se obtuvieron los aportes realizados por esta tesis, los trabajos futuros y mejoras.


\chapter{Marco teórico}\label{cap:marcoteorico}

En este capitulo se proporciona los antecedentes y detallan los conceptos teóricos en lo que se basa esta tesis. Empezando con la detección de objetos tradicional y sus componentes. Luego, se desarrolla el problema de aprendizaje sin ejemplos y como este se usa en la detección de objetos sin ejemplos. 

\section{El problema de detección de objetos} \label{sec:elproblemadedetcciondeobjetos}
Con los avances recientes en los modelos de visión por computadora basados en el aprendizaje profundo, las aplicaciones de detección de objetos son más fáciles de desarrollar que nunca. Además de importantes mejoras de rendimiento, estas técnicas también han aprovechado conjuntos de datos de imágenes masivos para reducir la necesidad de generar conjuntos de datos grandes. Con los enfoques actuales que se centran en arquitecturas con un pipeline de extremo a extremo, el rendimiento también ha mejorado significativamente, lo que permite casos de uso en tiempo real.\\


La detección de objetos es una importante tarea de visión por computadora, que consiste en localizar la presencia de objetos de determinadas clases como: humanos, automóviles, frutas, o animales, en imágenes digitales. El objetivo de la detección de objetos es desarrollar modelos y técnicas computacionales que proporcionen una de las piezas de información más básicas que necesitan las aplicaciones de visión por computadora: ¿Qué objetos están y dónde? En la \autoref{fig:example_detection}, se muestra un ejemplo de cómo se ve esta clase de algoritmos en la práctica.\\

\begin{figure}
	\centering
	\includegraphics[width=\textwidth]{ejemplo_detecion_de_objetos.png}
	\caption{Ejemplo de detección de objetos, el modelo utilizado es Faster R-CNN~\cite{ren2015faster}. Las clases que puede detectar esta implementación son: ``Auto (\textit{car})'',  ``Persona (\textit{person})'',  ``Planta en maceta (\textit{potted plant})'', entre otras.}
	\label{fig:example_detection}
\end{figure}

Se puede considerar que la deteeccion de objetos tiene asociado dos sub-problemas:

\begin{itemize}
	\item \textbf{Localización:} localiza la presencia de un objeto en la imagen y lo representa con un cuadro delimitador (\textit{Bounding box}). Toma una imagen como entrada y muestra la ubicación del cuadro delimitador en algún formato, por ejemplo: (posición, alto y ancho).
	\item \textbf{Reconocimiento:} tiene como entrada una sub-figura de una imagen (generalmente indicada por un cuadro delimitador) y genera la etiqueta de clase de esa figura con una medida de la confianza, que indica que tan ``buena'' es la predicción. Este problema esta asociado con la clasificación de imágenes, el cual genera una etiqueta y una medida de confianza para una imagen completa.
\end{itemize}


Teniendo en cuenta estos dos componentes, existen principalmente dos tipos de detectores de objetos. Por un lado, tenemos detectores de dos etapas (\textit{two-stage}), como Faster R-CNN~\cite{ren2015faster}, que utilizan una red (\textit{region proposal network}, RPN, por su denominación en inglés) para generar regiones candidatas a contener un objeto de cualquier clase, para luego enviar estas propuestas a una red de reconocimiento. Estos modelos alcanzan las tasas de precisión más altas, pero suelen ser más lentos. Por otro lado, tenemos detectores de una etapa (\textit{one-stage}), como YOLO~\cite{redmon2016you}, que tratan la detección de objetos como un simple problema de regresión, tomando una imagen de entrada y aprendiendo la predicción de clase y coordenadas del cuadro delimitadores en una misma red. Estos últimos, tienen un diseño más eficiente y elegante. 

Los detectores de dos etapas filtran la mayoría de las propuestas negativas, pasando a la etapa de clasificación un numero menor de cuadros, mientras que los detectores de una etapa se enfrentan directamente a todas las regiones de la imagen, reduciendo su desempeño.

En este trabajo nos centraremos en detectores de objetos de dos etapas, la \autoref{fig:OneStage} muestra un ejemplo de como funcionan estos tipos de detectores.\\

\begin{figure}
	\centering
	\includegraphics[width=1\textwidth]{one_stage.png}
	\caption{Las distintas etapas de un detector de objetos \textit{two-stage}. Imagen adaptada del trabajo Faster R-CNN~\cite{ren2015faster}.}
	\label{fig:OneStage}
\end{figure}

\section{Datos de entrenamiento} \label{sec:datosdeentrenamiento}
Los datos, en general, son un elemento vital de los proyectos de aprendizaje automático supervisado. Cuantos más datos tenga, mejor será el producto final. Sin embargo, no basta con tener datos brutos. Debe tener estos datos anotados, estas anotaciones surgen del proceso de etiquetado que consiste en indicar la ``cosa'' de interés. Dependiendo de los datos y el objetivo final, esta tarea varia. Las etiquetas junto a los datos brutos, ``alimentan'' al algoritmo de aprendizaje automático que luego permiten identificar correctamente los objetos en una imagen, comprender el habla humana y muchas otras funcionalidades. Según algunas estimaciones, el 80\% del tiempo de desarrollo del proyecto de aprendizaje automático se dedica a preparar los datos, y ademas, el costo de esta tarea es muy alta. La razón por la que la anotación de datos es tan importante es que incluso el más mínimo error podría resultar en un modelo deficiente. Esta es una de las áreas en las que los humanos tenemos una ventaja en las computadoras, ya que podemos lidiar mejor con la ambigüedad, descifrar la intención y muchos otros factores que intervienen en la anotación de datos.

Por otro lado, por lo general las anotaciones sufren de un fenómeno denominado ``cola larga'' (o mas conocido como \textit{long tail}). Este fenómeno surge cuando  el número de ejemplos de entrenamiento por clase varía significativamente de miles para las clases principales a tan solo unos pocos para las clases finales. El problema de ``cola larga'' se puede observar en los datos anotados de COCO  \autoref{fig:COCOAnotaciones}, donde la clase persona tiene aproximadamente 260.000 y otras clases como tijeras, oso o parquímetro solo 1000 anotaciones.

En el contexto de la detección de objetos los datos anotados son cuadros delimitadores y sus etiquetas. Una pregunta que siempre se hace es la siguiente: para un modelo de detección de objetos en un problema X, ¿cuántas imágenes se necesitan? La respuesta mas común que se encuentra en la red es 1000 imágenes por clase que se quiere detectar. El origen de este número mágico, proviene del desafío de clasificación de ImageNet original, donde el conjunto de datos tenía, 1000 categorías cada una con 1000 imágenes aproximadamente y por lo general un objeto por imagen. Pero si lo comparamos con un conjunto de datos actual (COCO) donde se evaluá detección de objetos, podemos observar que se requiere muchas mas anotaciones (aproximadamente 10.000 anotaciones en promedio, ver \autoref{fig:COCOAnotaciones}).

\begin{figure}
	\centering
	\includegraphics[width=1\textwidth]{coco_anotaciones.png}
	\caption{Numero de anotaciones por clases en el conjunto de datos COCO.}
	\label{fig:COCOAnotaciones}
\end{figure}

Esto ejemplos nos pueden orientar un poco, pero aun así dar un numero exacto es difícil, ya que varia mucho dependiendo del problema. Lo que se necesita es un conjunto de imágenes que represente todos los escenarios posibles en el que puede aparecer el objeto que se quiere detectar. Conseguir este conjunto puede que resulte muy difícil, ya sea por el costo que conlleva generar esas anotaciones, por que los escenarios donde aparece el objeto son demasiados, o por que simplemente es complicado conseguir una imagen del objeto.\\

Por estos motivos es necesario encontrar algún método que reduzca el numero de anotaciones necesarias para la detección de objetos. Es así como nacen los métodos denominados ``Aprendizaje sin ejemplos'', los cuales analizaremos a continuación.
 
\section{Aprendizaje sin ejemplos} \label{sec:aprendizajesinejemplos}
El Aprendizaje sin ejemplos (ZSL, por su denominación en inglés \textit{Zero-shot Learning}), es un conjunto de problemas de aprendizaje automático, donde en el momento de la prueba, se observan muestras de clases que no se observaron durante el entrenamiento y se necesita predecir la categoría a la que pertenecen. Esta se diferencia de las configuraciones estándares en el aprendizaje automático, donde se espera que se clasifiquen correctamente las nuevas muestras en las clases que ya se han observado durante el entrenamiento. 

Esta técnica se puede aplicar en problemas de visión por computadora, algunos escenarios en el que se puede utilizar son:

\begin{itemize}
	\item \textbf{El número de clases objetivo es grande:} Generalmente, los seres humanos pueden reconocer una gran cantidad de clases. Sin embargo, recopilar suficientes instancias etiquetadas para un número tan grande de clases es un desafío.
	
	\item \textbf{Las clases objetivo son raras:} Supongamos que queremos reconocer flores de diferentes razas. Es difícil recopilar suficientes instancias de imágenes para todas las razas. Ademas, para muchas flores raras, no podemos encontrar las instancias etiquetadas correspondientes.
	
	\item \textbf{Las clases objetivo cambian con el tiempo:} Un ejemplo es reconocer imágenes de productos pertenecientes a cierto estilo o marca. Dado que los estilos y marcas cambian con frecuencia, para algunos productos nuevos, es difícil encontrar los casos etiquetados correspondientes.
	
	\item \textbf{El costo de obtener instancias etiquetadas:} El proceso de etiquetado de instancias es caro y requiere mucho tiempo. Por ejemplo, supongamos que queremos detectar un animal en extinción para poder rastrearlos y tener una forma de contabilizarlos y proteger las áreas donde se encuentran, por ejemplo el yaguareté que habita el Chaco Argentino. Poder conseguir un conjunto considerable de fotos del mismo en distintos entornos, situaciones y ubicaciones, como de día, noche, corriendo, descansando, etc. puede resultar un gran desafió y demasiado tarde si nos lleva mucho tiempo.
\end{itemize}

Todos estos problemas se puede intentar solucionar con ZSL, entrenando los modelos con clases similares y luego evaluarlas en las clases que nos interesan. Por ejemplo, para el problema del  yaguareté, se puede proponer un modelo de ZSL, entrenado con imágenes de animales similares como leopardo, puma, chita, etc. y luego utilizar este modelo para detectar el yaguareté.\\
 
Es necesario entender que en este tipo de configuración, existen dos tipos de clases. Las vistas, que son todas aquellas que tienen al menos una instancia en los datos de entretenimiento y las invisibles que no tienen ninguna instancia en los datos de entrenamiento. Dicho esto el aprendizaje sin ejemplos se puede dividir en categorías según los datos presentes durante la fase de entrenamiento y la fase de prueba:
\begin{itemize}
	\item En base a los datos disponibles en el momento de entrenar un modelo.
	\begin{itemize}
		\item \textbf{Zero-shot learning inductivo:} Se tiene acceso a los datos y a la información complementario de solo las clases vistas.
		\item \textbf{Zero-shot learning transductivo:} Además de los datos y la información complementario de las clases vistas,  se tiene acceso a los datos de las clases no vistas.
	\end{itemize}
	\item Basado en los datos disponibles en el momento de la inferencia.
	\begin{itemize}
		\item \textbf{Zero-shot learning convencional (ZSL):} En las pruebas solo se evalúan las clases no vistas.
		\item \textbf{Zero-shot learning generalizado (GZSL):} En las pruebas se evalúan tanto las clases vista como las no vistas.
	\end{itemize}
\end{itemize}

En el aprendizaje sin ejemplo, no se tiene información sobre las clases de test. La forma de suplir la falta de esa información sobre las clases que se quiere predecir, es utilizar otro espacio que pueda representar tanto las clases vista y las invisibles. Para poder generar ese espacio, se necesita algún tipo de información auxiliar, que pueda representar todas las clases.
Algunos ejemplos de información auxiliar son, una estructura de clases, una descripción textual en lenguaje natural, etc. La \autoref{fig:modales} muestra algunos tipos de información subsidiaria para un mismo dato de entrada.

\begin{figure}[]
	\centering
	\includegraphics[width=0.8\textwidth]{img/modales.png}
	\caption{Idealización de distintos tipos de información subsidiaria con sus respectivos espacios.}
	\label{fig:modales}
\end{figure}

En obras existentes, el enfoque de información auxiliar se inspira en la forma en que los seres humanos reconocen el mundo. Los seres humanos pueden realizar un aprendizaje sin ejemplo con la ayuda de algunos conocimientos semánticos. Por ejemplo, con el conocimiento de que ``una cebra se parece a un caballo y tiene rayas'', podemos reconocer una cebra incluso sin haberla visto antes, siempre que sepamos cómo es un caballo y cómo se ve el patrón rayas. De esta manera, la información auxiliar involucrada por los métodos de aprendizaje sin ejemplos existentes suele ser información semántica que forman un espacio que contiene tanto las clases visibles como las invisibles.\\

Esta técnica de utilizar dos tipos de información, semántica y visual se denomina multimodales. La modalidad se refiere a la forma en que algo sucede o se experimenta. Un problema de investigación se caracteriza como multimodal cuando incluye datos de distinta naturaleza. La \autoref{fig:EjemploZSD} muestra un ejemplo de como funciona un clasificador sin ejemplos usando multimodales, transformando las imágenes a un espacio que representa las clases.\\

A continuación se detalla como se puede procesar y representar los dos tipos de información utilizado en este trabajo, visuales y semánticos.

\begin{figure}[]
	\centering
	\includegraphics[width=0.8\textwidth]{img/Modelo.png}
	\caption{Descripción de como funciona un clasificador de imágenes usando aprendizaje sin ejemplo. ``Auto'', ``Camión'', ``Perro'' y ``Tigre'' se observan  durante el entrenamiento,  ``Gato'' y ``Camioneta'' son clases invisibles. El formato de las imágenes sufren una transformación y se mapean a un espacio donde se contiene la información de todas las clases. Este mapeo agrupa distintas imágenes de una clase en espacios cercanos y mientras mas distinta es la clase mas alejadas están.}
	\label{fig:EjemploZSD}
\end{figure}


\subsection{Redes neuronales convolucionales} \label{sec:redesneuronalesconvolucionales}
Las redes neuronales convolucionales, o CNN por sus siglas en inglés, son un tipo de modelos de aprendizaje profundo (\textit{Deep learning}) utilizadas  para procesar distintos tipos de datos, pero empleadas generalmente en el dominio de las imágenes. Está inspirado en la organización de la corteza visual de los animales, diseñada para aprender de forma automática y adaptativa patrones en jerarquías, de bajo a alto nivel, es decir, de formas simples como cuadrados y círculos a formas mas complejas como autos. Esto quiere decir que las primeras capas pueden detectar lineas, curvas y se van especializando hasta llegar a capas más profundas que reconocen formas complejas como un rostro. Por lo general una red CNN se compone de tres tipos de capas: convolución, agrupación y capas completamente conectadas, como se puede ver en la \autoref{fig:CNNEjemplo}. El rol de cada capa es:

\begin{itemize}
	\item Convolución: somete los datos de entrada a un conjunto de filtros convolucionales, cada uno de los cuales activa ciertas características de los datos. Generalmente esta acompañada de una capa de activación, que permite un entrenamiento más rápido y eficaz al asignar los valores negativos a cero y mantener los valores positivos, de esta manera permitir que solo las características activadas pasen a la siguiente capa.
	\item Agrupación: se coloca generalmente después de la capa convolucional. Su utilidad principal radica en la reducción de su entrada para la siguiente capa convolucional, reduciendo así el número de parámetros que la red necesita aprender.
	\item Capas completamente conectadas: Es la encargada de relacionar los datos de las capas anteriores y generar una salida, que por lo general es utilizada para clasificar los datos de entrada.
\end{itemize}


Algunos ejemplos de redes CNN son: VGG16~\cite{simonyan2014very} (que posee 13 capas de convolución, 5 de agrupación y una totalmente conectada) y AlexNet~\cite{krizhevsky2012imagenet} (que contiene 5 capas convolucionales, 3 capas de agrupación y 3 capas completamente conectadas).\\


\begin{figure}
	\centering
	\includegraphics[width=0.9\textwidth]{img/red_cnn.png}
	\caption{Una arquitectura simplificada de una red neuronal convolucional.}
	\label{fig:CNNEjemplo}
\end{figure}

\subsection{Vectores de palabras (Word embedding)} \label{sec:wordembedding}
Así como en las imágenes utilizamos las redes CNN, para obtener un vector que represente a la misma, es necesario un procedimiento para representar clases de interés con algún objeto matemático. Hay muchas formas de representar palabras, la más usada son los \textit{word embedding}. Esta es una técnica de aprendizaje en el campo de procesamiento del lenguaje natural (PLN), capaz de capturar el contexto de una palabra en un documento, calcular similitud semántica y sintáctica con otras palabras.\\

Para entender como funcionan, consideremos las oraciones con un significado similar: ``Que tengas un buen día.'' y ``Que tengas un gran día.''. Si construimos un vocabulario exhaustivo:
\[ V = \{que, tengas, un, buen, gran, dia\}. \]
A partir de esto, se puede crear un vector codificado para cada una de estas palabras, en donde cada vector tenga el tamaño de $V$, cuyos componentes sean todos 0 excepto por el elemento en el índice que representa la palabra correspondiente en el vocabulario, que contiene un 1. Esta representación no resulta conveniente ya que la distancia entre \textit{gran} y \textit{buen} es la misma que entre \textit{tengas} y \textit{buen}.  El objetivo es que las palabras con un contexto similar ocupen posiciones espaciales cercanas. Para lograr esto, se introduce cierta dependencia de una palabra con las otras.\\

Word2Vec~\cite{mikolov2013distributed} desarrollado por Tomas Mikolov en 2013. Es un modelo particularmente eficiente desde el punto de vista computacional. Este modelo se encuentra disponible de dos formas: \textit{Continuous Bag-of-Words} (CBOW) o el modelo \textit{Skip-Gram}. En CBOW, las representaciones distribuidas de contexto (o palabras circundantes) se combinan para predecir la palabra en el medio. En nuestro ejemplo \textit{gran} y \textit{buen} están rodeado de un contexto similar por lo cual resultan en vectores similares. Es varias veces más rápido de entrenar que el \textit{Skip-gram}, y tiene una precisión ligeramente mejor para las palabras frecuentes. Mientras que en el modelo \textit{Skip-gram}, la representación distribuida de la palabra de entrada se usa para predecir el contexto. Se entrena con una tarea falsa que, dada una palabra, intenta predecir las palabras vecinas. En realidad, el objetivo es solo aprender los pesos de la capa oculta que corresponden a los vectores de palabras que estamos tratando de aprender. Por ejemplo, \textit{Gran} se entrena para predecir el contexto \textit{un} y  \textit{día}, al igual que \textit{buen}. Funciona bien con una pequeña cantidad de datos de entrenamiento.

Los modelos de ZSL, aprovechan la capacidad de capturar similitudes semántica que tiene \textit{word embedding}, para relacionar las clases vistas con las clases invisibles.\\



\section{Detección de objetos sin ejemplos} \label{sec:detecciondeobjetossinejemplo}

\subsection{Introducción} 
El aprendizaje sin ejemplos identifica objetos invisibles para los que no hay imágenes de entrenamiento disponibles. Los enfoques de ZSL convencionales están restringidos a una configuración de reconocimiento donde cada imagen de prueba se clasifica en una de varias clases de objetos invisibles. Esta configuración no es adecuada para aplicaciones del mundo real donde los objetos invisibles aparecen como parte de una escena completa. Para abordar esta limitación, aparece una nueva configuración, la detección de objetos sin ejemplos (ZSD) por sus siglas en ingles \textit{Zero-shot Object Detection}, que tiene como objetivo reconocer y localizar simultáneamente instancias de objetos, incluso en ausencia de ejemplos visuales de esas clases durante la fase de entrenamiento.\\

Se podría considerar que un modelo de detección de objetos sin ejemplos, es un modelo de ZSL pero con un paso extra, ubicar todas las instancias de objetos que aparecen en una imagen. Este paso se denomina propuesta de objetos y tiene que ser capas de diferenciar fondos y generar una lista con cuadros delimitadores con posibilidad de contener algún objeto. Un algoritmo de localización de objetos generará las coordenadas de la ubicación de los objetos con respecto a la imagen. En visión artificial, la forma más popular de representar la ubicación de los objetos es con la ayuda de cuadros delimitadores (\textit{Bounding Boxes}). Existen muchos algoritmos y redes que intenta resolver este problema, algunos ejemplos son ventana deslizante (\textit{slide window}), Edge-Boxes~\cite{zitnick2014edge} y búsqueda selectiva (\textit{selective search})~\cite{uijlings2013selective}. En ZSD la propuesta de objetos cumple un papel importante, ya que se necesita extraer todas las instancias de los objetos, pero también tiene que discriminar fondos como cielo, ciudades, veredas, etc.\\ 

\begin{figure}[]
	\centering
	\includegraphics[width=1\textwidth]{img/NMS.png}
	\caption{Salida de un generador de propuesta de objetos y el resultado después de  usar NMS.}
	\label{fig:NMS}
\end{figure}

Como veremos en el \autoref{cap:experimentos} se experimentó con Edge-Boxes~\cite{zitnick2014edge} y \textit{selective search})~\cite{uijlings2013selective}, ya que estos generan una cantidad de propuestas significativamente menor a algoritmos del estilo de ventana deslizante. Esto se debe a que no recorren toda la imagen generando distintos cuadros para distinto tamaños, si no que extraen algún tipo de información de la imagen y generan los cuadros basándose en dicha información, refinando y reduciendo el numero de cuadros. Para el caso de Edge-Boxes~\cite{zitnick2014edge} utiliza un método para generar propuestas de cuadro delimitador utilizando bordes, que proporcionan una representación escasa pero informativa de una imagen. La principal observación es que el número de contornos que están totalmente contenidos en un cuadro delimitador es indicativo de la probabilidad de que el cuadro contenga un objeto. De esta maneara con 1.000 cuadros pueden detectar hasta el 90\% de los objetos. Por otro lado \textit{selective search})~\cite{uijlings2013selective} combina la fuerza de una búsqueda exhaustiva y la de segmentación, que agrupa los pixeles en segmentos, ya sea por color, por texturas, etc. Este algoritmo genera aproximadamente 10.000 propuestas que detecta hasta el 99\% de los objetos.

Pero aun así Edge-Boxes~\cite{zitnick2014edge} y \textit{selective search})~\cite{uijlings2013selective}, siguen generando una cantidad considerable de cuadros (del orden de los miles) y ademas, muchos cuadros con una gran superposición. Esto da lugar a una técnica de refinamiento denominada supresión de no máximos (NMS), ejemplificada en la~\autoref{fig:NMS}. La salida de NMS es un conjunto más reducido de propuestas, en la cual se filtraron todas las que se consideran repetidas y retorna solo las más representativa.\\

El requisito estricto de no utilizar ninguna imagen de clase invisible durante el entrenamiento es una condición difícil. Además, existen otras dificultades en la tarea de ZSD relacionadas al conjunto de datos de entrenamiento y prueba, es decir entre las clases vistas e invisibles. Estas dificultades son:

\begin{itemize}
	\item \textbf{Rareza}: los conjuntos de datos, por lo general, contiene un problema de distribución, es decir, muchas clases raras tienen menos cantidad de instancias. Este problema hace que las clases con mayor cantidad de instancias sesguen el modelo y las clases más raras sean marcadas incorrectamente en la etapa de prueba. Esto es un problema al momento de comparar dos modelos que fueron entrenados con distintas clases, ya que algunas separaciones  de las clases resultan mejores que otras.
	
	\item \textbf{Tamaño del objeto}: algunas clases de objetos raros (tijeras, lápices, celulares, etc.), suelen tener un tamaño pequeño. Los objetos más pequeños son difíciles de detectar y reconocer. También, tienen el problema de que por lo general están junto a objetos más grandes como una mesa o una persona y se ven opacadas por estas clases.
	
	\item \textbf{Diversidad}: cuando una clase invisible no tiene otras clases visualmente similares, resulta muy difícil aprender el aspecto visual de esta. Por ejemplo, ``auto'' tiene muchas clases similares en comparación con ``cartel''. Esto permite una descripción visual inadecuada de la clase invisible ``cartel'' que eventualmente afectará el rendimiento de ZSD, a diferencia de lo que sucede con la clase ``auto''.
	
	\item \textbf{Ruido en el espacio semántico}: cuando se utiliza los vectores de incrustación semántica no supervisados como Word2Vec~\cite{mikolov2013distributed} o GloVe~\cite{pennington2014glove}, las embeddings resultante en general son ruidosas, ya que se generan automáticamente a partir de la minería de texto no anotado. Esto también afecta significativamente el rendimiento de ZSD.
\end{itemize}

\subsection{Trabajos recientes en ZSD} \label{ssec:trabajosrecientesenzsd}

Existen muchas técnicas propuestas para resolver ZSD. Cuando se empezó a leer sobre este tema a fines del 2018, la más utilizada consistía en crear una combinación de aspectos visuales y semánticos de cada objeto. Puntualmente existían tres trabajos en paralelos con una metodología similar. Bansal \etal~\cite{bansal2018zero} propuso un enfoque basado en características donde las propuestas de objetos se generan mediante edge-box. Zhu \etal~\cite{zhu2018zero} propone un método basado en el detector YOLO~\cite{redmon2016you}. Rahman \etal~\cite{rahman2018zero} propuso una extensión de Faster R-CNN~\cite{ren2015faster} junto a un nuevo enfoque transductivo para asociar objetos novedosos en el espacio semántico.\\

En los últimos dos años se publicaron nuevos trabajos utilizando esta técnica. Rahman \etal~\cite{rahman2020zero} que mejora los modelos y resultados de su trabajo previo~\cite{rahman2018zero}. Gupta \etal~\cite{gupta2020multi} donde combina predicciones obtenidas en dos espacios de búsqueda diferentes, es decir, del espacio semántico al visual y viceversa. Rahman \etal~\cite{rahman2020improved} proponen  una función de pérdida novedosa que maneja el desequilibrio de clases y busca alinear adecuadamente los vectores visuales y semánticos.

Este trabajo se basa en el artículo científico de Bansal \etal ~\cite{bansal2018zero}, además utiliza muchos conceptos sobre zero-shot learning generalizado~\cite{zero-shot-generalizado}. Se propone un modelo zero-shot  inductivo, es decir, solo se observan imágenes de clases vistas y etiquetas que indican a que clase pertenece. Estas etiquetas son palabras del lenguaje natural sin ninguna estructura. Luego, se puede inferir todas las clases o solo las invisibles, dependiendo de si se quiere evaluar aprendizaje por zero-shot  generalizado o convencional, respectivamente.\\ 

\subsection{Formalización de ZSD} \label{ssec:formalizaciondezsd}
Para formalizar ZSD denotamos el conjunto de las clases como $\mathcal{C} = \mathcal{S} \cup \mathcal{U}$, donde $\mathcal{S}$ son las clases vistas para entrenamiento y $\mathcal{U}$ las clases no vistas, utilizadas en la etapa de pruebas. Además se tiene que $\mathcal{S} \cap \mathcal{U} = \emptyset$. Aunque no es necesario definir el conjunto de clases de pruebas, ya que el modelo tiene que ser capás de detectar tanto clases vista como las no vista, esto se hace para poder tener una evaluación cuantitativa.

Denotamos a una imagen como $\mathcal{I} \in \mathbb{D}^{\mathcal{H} \times \mathcal{W} \times 3}$. Donde $\mathbb{D} = \{0,...,255\}$, $\mathcal{H}$  es el largo de la imagen, $\mathcal{W}$ el ancho. Esta es la forma en la que se representa cada pixel de la imagen en el formato \textbf{RGB}, donde se tiene 3 canales que caracterizan la intensidad de los colores rojo, verde y azul. 

Por cada imagen se provee un conjunto de cuadros delimitadores $\mathbb{B} = \{b_0,...,b_k\mid b_i \in N^4\}$ (cada $b_i$ representa las coordenadas de un cuadro) y sus etiquetas asociadas como $\mathbb{Y} = \{y_0,...,y_k\mid y_i \in \mathcal{C}\}$. Para cada cuadro delimitador $b_i$ extraemos una característica profunda utilizando una red neuronal convolucional denotada como $\phi(b_i) \in \mathbb{R}^{D_1}$. 

Denotamos las incrustaciones semánticas $w_j \in \mathbb{R}^{D_2}$ obtenido por algún modelo como Word2Vec~\cite{mikolov2013distributed}. El conjunto de todas las imágenes de entrenamiento se indica con $\mathcal{X}^s$, que contiene ejemplos de todas las clases de objetos visibles.  El conjunto de todas las imágenes de prueba que contienen muestras de clases de objetos invisibles se indica con  $\mathcal{X}^u$. En particular, no hay ningún objeto de clase invisible en $\mathcal{X}^s$, pero $\mathcal{X}^u$ puede contener objetos vistos.\\

El objetivo es encontrar una matriz de proyección $W_p$, tal que 
\[ \psi_i = W_p\phi(b_i) \:\:\:,\:\:\: W_P \in \mathbb{R}^{D_2 \times D_1},\:\:\: \psi_i \in \mathbb{R}^{D_2} \] 
Note que $\psi_i$ y las incrustaciones semánticas se encuentran en el mismo dominio. Como mencionamos en secciones anteriores, el espacio vectorial semántico, tiene una gran capacidad de capturar similitudes semánticas. Por lo cual, resulta clave encontrar una matriz que para cada cuadro delimitador se proyecte lo más cerca posible a la incrustación semántica de su clase. 

El resultado es una función 
\[f : \mathcal{X}, W_p  \to \{y_0,...,y_k\mid y_i \in \mathcal{C}\} \quad \operatorname{con}\quad \mathcal{X} =  \mathcal{X}^s \cup \mathcal{X}^u\] 
con un riesgo empírico regularizado mínimo $\mathcal{R}$ definido de la siguiente manera: 
\[ \arg_{}\min_{f \in F} \mathcal{R}(f(x,W_p))\quad, \] 
donde $x \in \mathcal{X}^s$ durante el entrenamiento. La función de mapeo utilizada en la etapa de inferencia, tiene la siguiente forma \[ f(x,W_p) = \arg_{}\max_{y \in \mathcal{C}}\max_{b \in \mathbb{B}(x)} (F(x,y,b,W_p)) \quad,\] donde los $\mathbb{B}(x)$ es el conjunto de propuestas de la imagen $x$. Intuitivamente se buscan los cuadros delimitadores de mejor puntuación y se les asigna la categoría de objeto de puntuación máxima.\\

Ahora que tenemos una idea general del problema de ZSD y como se define formalmente, podemos proponer una arquitectura para resolver este problema, ademas de como y donde se puede evaluar dicha arquitectura.

\chapter{Modelado y conjuntos de datos}\label{cap:arquitecturayconjuntosdedatos}

En este capítulo se detalla la metodología que se utilizó en la presente tesis para resolver ZSD. Específicamente, se explayan las distintas etapas, se presentan los conjuntos de datos utilizados y se desarrolla como puede evaluarse el rendimiento de los modelos de ZSD Y GZSD en estos conjuntos de datos.\\

En la actualidad la arquitectura mas utilizado por la comunidad científica para resolver el problema de ZSD, es utilizar el espacio que forman modelos como Word2Vec~\cite{mikolov2013distributed} o GloVe~\cite{pennington2014glove} para transferir el conocimiento de las clases vistas a las inviables. Luego de analizar los distintos artículos (ver \autoref{ssec:trabajosrecientesenzsd}), y basándonos en la complejidad del modelo y sus resultados, se decidió apoyarse en el trabajo de Bansal \etal~\cite{bansal2018zero} para abordar el problema de ZSD. En las siguientes secciones se abordaran las distintas etapas y los detalles de la arquitectura.


\section{Modelado}\label{ssec:preprocesamiento} 
Antes de detallar el entrenamiento de la arquitectura, es necesario explayar como se modifican los datos, para poder ser utilizados. Cada dato de entrenamiento consiste de una imagen y un conjunto de cuadros delimitadores con el nombre de la clase del objeto que se encuentra dentro del cuadro. El objetivo de esta etapa es generar por cada imagen un conjunto de puntos en el espacio visual y semántico que corresponde a cada cuadro. Por un lado, se recortan los cuadros de la imagen $x_i$ y se recalan a un tamaño fijo, por ejemplo 224$\times$224. Luego se utiliza una CNN pre-entrenada como VGG16~\cite{simonyan2014very} para generar los vectores visuales de cada cuadro. La salida de este paso es:

\[B_i = [\phi(b_0),...,\phi(b_k) \mid \phi(b_i) \in \mathbb{R}^{D_1}]\] 

Donde $B_i$ son todos los vectores de características visuales de la imagen $x_i$.

Por otro lado, cada cuadro se asocia con el vector semántico de la clase que tiene asignado, que se puede obtener con modelos de vectores de palabras previamente entrenados, como Word2Vec~\cite{mikolov2013distributed} o GloVe~\cite{pennington2014glove}. Dando como resultado:

\[W_i = [w_0,...,w_k \mid w_i \in \mathbb{R}^{D_2}]\]

Donde $W_i$ son los vectores semánticos asociado a cada cuadro. En la \autoref{fig:arqutectura} se ejemplifica los pasos mencionados anteriormente.\\


En la inferencia se quiere predecir la clase a la que pertenece cada cuadro delimitador en una imagen, para esto, se computa el vector visual $\phi(b_i)$ obtenido por una CNN como VGG16~\cite{simonyan2014very}, y luego utilizando la matriz de proyección $W_p$ (que nos permite pasar del dominio visual al semántico, ver \autoref{ssec:formalizaciondezsd}), calculamos el vector semántico asociado al vector visual. Por ultimo se calcula la similitud coseno entre el vector semántico obtenido y el de las de todas las clases que se quiere evaluar, asignándole a esa propuesta la que obtenga un mejor puntaje. La \autoref{fig:arquitectura_prueba} muestra la arquitectura empleada en la etapa de inferencia.

\begin{figure}[H]
	\centering
	\includegraphics[width=0.9\textwidth]{img/arquitectura.png}
	\caption{Esquema del pre-procesamiento de las imágenes y de como se obtienen los vectores semánticos y visuales.}
	\label{fig:arqutectura}
\end{figure}

\begin{figure}[H]
	\centering
	\includegraphics[width=0.9\textwidth]{img/arquitectura_prueba.png}
	\caption{Arquitectura propuesta para inferir los vectores semánticos (por ende a la clase que pertenece un cuadro delimitador) a partir de un vector visual.}
	\label{fig:arquitectura_prueba}
\end{figure}

\section{Entrenamiento del modelo}\label{ssec:entrenamiento}

\subsection{Función de costo}

Utilizamos el espacio semántico (${R}^{D_2}$) para calcular una medida de similitud entre las proyecciones $\phi(b_j) \in X_i	$ y los vectores semánticos $w_j \in W_i$. Luego, para entrena la proyección $W_p$, que nos permite transformar un vector del espacio visual a uno del espacio semántico, definimos una función de costo, que imponga la restricción que el puntaje de la similitud de un cuadro delimitador, con su clase verdadera, debe ser más alto que el de otras clases. Por ejemplo, la proyección de un cuadro que tiene un ``perro'', tiene que estar lo mas cerca posible del vector semántico ``perro'', y a su ves lejos de cualquier otro vector semántico como ``gato'' o ``auto''.

Utilizaremos la función de costo definida como: 

\[\mathcal{L}(\psi_i, w_i) = \sum_{j \in \mathcal{S}, j\neq i} max(0, m - S_{ii} + S_{ij})\] 
donde $m$ es el margen máximo, y $S_{ij}$ es la similitud entre la proyección $i$-$esima$ y el vector semántico $j$-$esimo$. 

Para comprender mejor esta función supongamos un margen máximo de 1, y que para todo $S_{ij}$ se cumple $0 < S_{ij} < 1$. Si la similitud entre la proyección y su vector semántico ($S_{ii}$) es cercano a 1, la función sólo dependen de los $S_{ij}$, con $j \neq i$, cuando estos valores se acercan a 0 la función de costo se minimiza, y cuando aumentan la función de costo también lo hace. Por otro lado, si la similitud $S_{ii}$ es aproximadamente 0, estaríamos penalizando la función sin importar los $S_{ij}$, pero si estos aumentan la función de costo crecerá a la par de ellos. \\


También se agrega una función de costo de reconstrucción ($\mathcal{L}_r$) como sugiere Kodirov \etal~\cite{kodirov2017semantic}. En esta función se utilizan las características del cuadro delimitador proyectadas para reconstruir los vectores visuales originales, y calcular la pérdida de reconstrucción como la distancia $L2$  entre el vector reconstruido y el original:
\[\mathcal{L}_r = \Vert{\phi(b_i) - \psi_iW_p^T}\Vert^2 \] 
Luego, definimos $\lambda$ como un coeficiente de ponderación que controla la importancia del primer y segundo término, que corresponden a las pérdidas de proyección y reconstrucción, respectivamente. Por lo cual, la función de perdida total es: 
\[\mathcal{L}_t = \lambda \mathcal{L} + (1-\lambda) \mathcal{L}_r \]
 
Es común que en la detección de objetos se incluya una clase de fondo, para obtener un detector robusto que pueda discriminar eficazmente entre objetos de primer plano y objetos de fondo. En ZSD, esto no es un problema trivial, ya que no se sabe si un cuadro de fondo incluye elementos como cielo, tierra, bosque, etc. o una instancia de una clase de objeto invisible. En muchos trabajos se proponen distintas técnicas para abordar este problema, pero no presentan mejoras en evaluaciones cuantitativas. Es por esto que no se incluye una arquitectura que discrimine cuadros de fondos.

\section{Conjuntos de datos} \label{sec:conjuntosdedatos}

En la actualidad no existe un conjunto de datos pensado para evaluar ZSD, es por esto que se tiene que adaptar otros conjuntos de datos para poder medir el rendimiento de los modelos. Otra posibilidad es crear un conjunto de dato sintético que emule imágenes de la vida real para ser utilizados en ZSD.

\subsection{Common Objects in Context (COCO)}\label{ssec:commonobjectsincontext}

COCO es una base de datos que tiene como objetivo ayudar en la investigación de detección de objetos, posee varias características como segmentación de instancias, subtítulos de imágenes y localización de puntos clave de personas. Este conjunto de datos contiene 80 (65 para COCO 2014) tipos de objetos o  clases, con un total de 2.5 millones de instancias etiquetadas en 328.000 imágenes. La \autoref{fig:ejemplo_coco} muestra algunas imágenes que forman parte del conjunto COCO.

\begin{figure}
	\begin{center}
		\centering
		\includegraphics[width=1\textwidth]{img/coco_ejemplo.png}
		\caption{Ejemplos de imágenes del conjunto de datos COCO.}
		\label{fig:ejemplo_coco}
	\end{center}	
\end{figure}

La gran cantidad de instancias de objetos y de categorías, resulta en un conjunto ideal para entrenar y evaluar modelos de ZSD. Además, la mayoría de la imágenes constan de una gran cantidad de objetos, a diferencia de conjuntos como Visual Genome~\cite{krishnavisualgenome}. Estas tipo de imágenes generan un contexto en el que varios objetos se relacionan y se superponen, emulando de una mejor manera situaciones de la vida real. 

En este trabajo se utilizan las imágenes de entrenamiento del conjunto COCO 2014 e imágenes del conjunto de validación para realizar las pruebas de ZSD.

\begin{figure}
	\begin{center}
		\centering
		\includegraphics[width=1\textwidth]{img/data_set.png}
		\caption{División de las clases para entrenamiento (verde) y pruebas (azul).}
		\label{fig:data_set}
	\end{center}	
\end{figure}

Como COCO no provee una separación de los datos para evaluar modelos de ZSD, es necesario crear una forma de dividir las clases en vistas e invisibles. Esta separación resulta de suma importancia, ya que se debe cumplir que para todo objeto del conjunto prueba, exista otro de aspecto similar que este presente durante el entrenamiento. Además, no se puede encontrar ningún objeto de prueba en los datos de entrenamiento. Para esto, se aprovecha que COCO tiene agrupadas las clases por ``Clases superiores'' donde se agrupan objetos que tienen alguna relación. Por ejemplo, la clase superior animales contiene las clases zebra, perro, gato, etc. Por cada uno de estos grupos se elige de forma aleatoria un 70\% de clases para entrenamiento y un 30\% para pruebas. Es decir, 47 y 18 clases, respectivamente, de un total de 65 clases de COCO 2014. En la~\autoref{fig:data_set} se puede observar el resultado de esta división. Por último, se eliminaron todas las imágenes de entrenamiento que contengan al menos una instancia de las clases de prueba. Esto resulta en 42564 imágenes, con 261258 instancias de entrenamiento, y 3008 imágenes con 10878 instancias de prueba. 

Bansal \etal~\cite{bansal2018zero}, divide el conjunto de datos de manera similar, utilizando la misma cantidad de clases para las etapas de prueba y entrenamiento. Pero la diferencia radica en que utiliza los vectores densos de palabras para agrupar las clases, utilizando la  similitud coseno entre los vectores como métrica. Por último, elige de forma aleatoria las clases visibles e invisibles de cada grupo. En este trabajo también se utiliza esta separación para logra una comparación de modelos más justa.


\subsection{CIFAR-ZSD} \label{ssec:cifarzsd}
\begin{figure}[]
	\begin{center}
		\begin{subfigure}{.3\textwidth}
			\includegraphics[width=1\textwidth]{img/cifar-zsd-test400.jpg}
			\label{fig:ex1}
		\end{subfigure}
		\begin{subfigure}{.3\textwidth}
			\includegraphics[width=1\textwidth]{img/cifar-zsd-test379.jpg}
			\label{fig:ex2}
		\end{subfigure}
		\begin{subfigure}{.3\textwidth}
			\includegraphics[width=1\textwidth]{img/cifar-zsd-test283.jpg}
			\label{fig:ex3}
		\end{subfigure}
		\caption{Ejemplos de imágenes del conjunto de datos CIFAR-ZSD.}
		\label{fig:CIFAR-ZSD}
	\end{center}
\end{figure}

COCO puede resultar pesado en término computacional. Para soluciona esto se creó un conjunto de datos sintético basado en CIFAR-100 datasets, el cual denominamos CIFAR-ZSD. Éste consta de imágenes localizadas, rotadas y re-escalada aleatoriamente con un fondo de otra imagen (algunos ejemplos se pueden ver en la \autoref{fig:CIFAR-ZSD}). Con esto se intenta simular imágenes reales en la cual un objeto puede aparecer con distintos aspectos y escalas. Este conjunto esta dividido de tal forma que ninguna instancia de prueba  aparezca en el conjunto de entrenamiento.

Aunque este conjunto resulta muy útil para hacer pruebas de modelos, no es bueno para reportar métricas reales, pero en combinación con COCO, que si lo es, facilita los experimentos a realizar.


\subsection{Definición de métricas} \label{ssec:definiciondemetricas}
Entre los diferentes conjuntos de datos anotados utilizados por los desafíos de detección de objetos y la comunidad científica, la métrica más común utilizada para medir la precisión de las detecciones es el  \textit{Mean Average Precision (mAP)}, seguida por \textit{Recall}. Un dificultad que tienen los modelos de detección a diferencia de los de clasificación, es el calculo de estas métricas ya que no es trivial definirlas. Ademas, no existe una implementación estándar y publica para calcularlas, y aquellas implementaciones públicas están muy encapsuladas en el código, y resulta muy difícil adaptarlo para medir rendimientos de modelos propios. Como ya se mencionó anteriormente, el código de Bansal \etal~\cite{bansal2018zero} no esta disponible, por este motivo fue necesario encontrar alguna implementación de estas métricas. A partir de estas búsqueda se encontraron varias opciones, sin embargo los resultados variaban mucho de un código a otro. Esto se debe a la falta de consenso en diferentes trabajos e implementaciones de AP, que es un problema al que se enfrentan las comunidades académicas, tal como se plantea en artículo de Padilla \etal~\cite{padilla2020survey}. Además, \cite{padilla2020survey} propone una definición y un código para estandarizar las métricas, de manera que se puedan comprar distintos modelos de una forma ``justa''. Por estos motivos decidimos utilizar este trabajo y su implementación para calcular nuestras métricas, aunque los resultados no den exactos a los reportados por Bansal \etal~\cite{bansal2018zero}.\\

Ahora definamos las métricas, basándonos en el trabajo \cite{padilla2020survey}. Primero es necesario definir algunos conceptos:
\begin{itemize}
	\item Falso negativo (\textbf{FN}): Para un cuadro delimitador verdadero no se obtuvo ninguna detección en absoluto, o una propuesta tiene IoU $> umbral$ con algún cuadro verdadero y no se predijo correctamente la clase.
	\item Falso positivo (\textbf{FP}): Una propuesta predijo correctamente la clase de un cuadro delimitador verdadero pero el IoU $< umbral$, o es un predicción duplicada, es decir, ya se marco otra con mayor IoU como \textbf{TP}, o se detecto un objeto inexistente con IoU $< umbral$ para todo cuadro verdadero.
	\item Verdadero positivo (\textbf{TP}): Una propuesta predijo correctamente la clase y obtuvo un IoU $> umbral$ con algún cuadro verdadero.
	\item Verdadero negativo (\textbf{TN}): Esto sólo tiene sentido si se quisiera medir propuestas que no tienen un IoU $> umbral$ con todos los cuadros verdaderos, y además se predijo como clase de fondo. Pero en este trabajo no es utilizada.
\end{itemize}
El umbral por lo general es 0.5, pero se puede cambiar para exigir que tenga una mayor superposición.\\

La \textit{Recall}, también conocida como exhaustividad, mide la probabilidad de que los objetos anotados en la imagen se detecten correctamente, y viene dado por: 

\begin{equation}
	\label{eqn:recall}
	Recall =\frac{TP}{FN+TP}
\end{equation}

En otras palabras la \textit{recall} contabiliza cuantos objetos se detectaron correctamente de todos los anotados en una imagen.\\

El trabajo de Bansal \etal~\cite{bansal2018zero}, define \textit{Recall} de la siguiente manera: 
\begin{center}
	\textit{``Un cuadro delimitador predicho se marca como verdadero positivo solo si tiene una superposición de IoU mayor que un cierto umbral $t$ con un cuadro delimitador existente en la imagen y no se ha asignado ningún otro cuadro delimitador de mayor confianza al mismo cuadro. De lo contrario, se marca como falso positivo.''}\\
\end{center}

Según esta definición solo se tienen en cuenta los objetos que tuvieron al menos una propuesta con un IoU $> 0.5$, y el resto quedan fuera del cálculo de esta métrica. Esto genera una diferencia enorme entre los resultados calculados con esta definición y con los obtenidos usando la \autoref{eqn:recall}. Con el objetivo de poder comparar los resultados con otros modelos, en este trabajo se calcula la \textit{Recall} de ambas formas. 

Bansal \etal~\cite{bansal2018zero} además calcula una variación denominada \textit{K@Recall}, donde sólo se tienen en cuentan las \textit{K} mejores propuestas basándose en la confianza de la predicción y el resto son descartadas.\\


\textit{AP}, es una métrica popular para evaluar la precisión de los detectores de objetos mediante la estimación del área bajo la curva (AUC), que viene dada por la relación de la \textit{precisión} y la \textit{recall}. Donde la precisión consiste en medir el porcentaje de predicciones positivas correctas entre todas las predicciones realizadas y se define como:

\begin{equation} 
	\label{eqn:precision}
	Precision =\frac{TP}{FP+TP}
\end{equation}


Para dibujar la curva AUC necesitamos obtener múltiples pares de valores de \textit{precisión} y \textit{recall}, esto se logra cambiando un límite de puntuación. Este limite trata como un falso positivo o todas aquellas propuesta que tengan un puntaje de confianza menor.

Para entender mejor supongamos un limite tal que genera un numero de FP bajo, la \textit{precisión} será alta. Sin embargo, en este caso, se pueden pasar por alto muchos aspectos interesantes de analizar, como por ejemplo un numero de FN alto y por lo tanto una \textit{recall} baja. Pero si uno baja el limite se aceptaran más positivos y la \textit{recall} aumentará, pero el numero FP también puede aumentar, disminuyendo la \textit{precisión}. De esta manera a media que aumentamos la \textit{recall} (bajamos el limite) la \textit{presision} se tiene que mantener alta. Por esto una área alta bajo la curva (AUC) tiende a indicar tanto una alta \textit{recall} como una alta \textit{precisión}.


Se define \textit{mAP} para la detección de objetos como el promedio del AP calculado para todas las clases. Por lo general, se indica sobre que IoU se calcula, puede ser un único valor, como por ejemplo mAP@0.5, o un conjunto de umbrales, como \textit{mAP@[x, y]} promediando el valor de \textit{mAP} para cada IoU. El trabajo de Bansal \etal~\cite{bansal2018zero} reporta \textit{mAP}, pero no indica sobre que IoU se calcula, por lo cual se asume que se utilizo un valor de 0,5. Muchos trabajos que utilizan COCO, reportan \textit{mAP@[.5, .95]}. Esta métrica resulta muy útil si se quiere comparar rendimientos entre distintos trabajos.


\chapter{Experimentos}\label{cap:experimentos}

\section{Experimentación con Propuestas de objetos}
\begin{table}[]
	\centering
	\resizebox{12.5cm}{!} {
		\begin{tabular}{|l|c|r|r|r|c|r|}
			\hline
			\textbf{}                     & \multicolumn{4}{c|}{\textbf{Edge Boxes}}                                                                                                   & \multicolumn{2}{c|}{\textbf{Selective Search}}               \\ \hline
			\textbf{Algoritmo}            & \multicolumn{4}{c|}{\textbf{-}}                                                                                                            & \textbf{Single}         & \multicolumn{1}{c|}{\textbf{Fast}} \\ \hline
			\textbf{Numero de propuestas} & \textbf{100}                 & \multicolumn{1}{c|}{\textbf{500}} & \multicolumn{1}{c|}{\textbf{1000}} & \multicolumn{1}{c|}{\textbf{5000}} & \textbf{$\approx$ 5000} & \multicolumn{1}{c|}{\textbf{$\approx$ 1000}}  \\ \hline
			Tiempo promedio (s)           & \multicolumn{1}{r|}{0.11}    & 0,11                              & 0.12                               & 0,12                               & \multicolumn{1}{r|}{5,48}   &         1,41                           \\ \hline
			Propuetas Totales             & \multicolumn{1}{r|}{4.415.244} & 22.050.071                          & 43.802.935                           & 161.809.194                          & \multicolumn{1}{r|}{350.535.591}   &   95.643.172                                 \\ \hline
			Propuestas con IOU $> 0.5$    & \multicolumn{1}{r|}{86.233}   & 133.942                            & 155.584                             & 194.891                             & \multicolumn{1}{r|}{221.551}   & 203.563                                   \\ \hline
		\end{tabular}
	}
	\caption{Resultados de correr los distintos algoritmos de propuestas de regiones en los datos de entrenamiento. El numero de propuestas verdaderas es 261.258.}
	\label{tab:edgeVSselct}
\end{table}

Como se menciono anteriormente, el numero de propuestas es un parámetro clave. Algunas métricas son muy sensible a la cantidad de propuestas, afectando los resultados finales. Esto surgió, cuando se obtuvieron las primeras métricas, los valores estaban muy lejos de los esperados, y a medida que se aumentaba la cantidad de propuesta, los resultados empeoraban. Por este motivo se probaron dos algoritmos (\textbf{Edge Boxes} y \textbf{Selective Search}) con algunas combinaciones de sus parámetros. Con el objetivo de obtener una cantidad de propuestas que se superponga con el mayor numero de objetos sin afectar las métricas.\\

Para no sesgar al experimento con los datos de de prueba, se definió la metodología de la siguiente manera. Por cada imagen de entrenamiento se corrió el generador de propuestas, se calculo el tiempo y la cantidad de cuadros verdaderos que tenían un IoU $> 0.5$, con algún cuadro verdadero. El tiempo es un parámetro importante ya que algunos algoritmos soy muy lentos y resulta imposible usarlos. Como se puede observar en el Cuadro \ref{tab:edgeVSselct}, \textbf{Selective Search} obtiene una mayor cantidad de superposición, pero con un numero exageradamente grande de propuestas. La mejor opción es usar \textbf{Edge-boxes}. En cuanto numero de propuestas totales resulta mas conveniente entre 100 y 500 propuestas como máximo, ya que al aumentar este numero no se generan mejoras en superposición pero si aumenta el numero de propuestas. Si tenemos en cunta el tiempo, resulta mejor \textbf{Edge-boxes}, ya que demora una fracción de lo que tarda \textbf{Selective Search}.\\

\section{Experimentación con CNN}
Se deicidio analizar la CNN ya que el modelo final e muy dependiente de esta red y su capacidad de extraer caracteristicas visuales. Lo que se quiere aquí es que la red sea capas de asociar las caracteristicas visuales de objetos similares, y diferenciar los elementos de distinta naturaleza. En otras palabras, el espacio resultante tiene que distribuirse de tal manera que por ejemplo las imágenes de los animales estén muy cerca y a su ves alejado de vehículos o electrodomésticos, pero también tiene que mantener una separación entre los distintos animales como perro y gato. Bansal en su trabajo, propone utilizar \textbf{Inception ResNet V2}, pero esta red puede resulta muy pesada en cuanto a tiempo de ejecución y memoria. Por este motivo se decidió intentar con \textbf{VGG16}, que reduce el número de parámetros en las capas convolucionales y mejorar el tiempo de ejecución, ademas es una de la mas utilizada.\\

El experimento consistió en comparo miles de recuadros de 3 clases de entrenamiento, caballo, perro y camión.  Por cada cuadro se genero el vector de caracteristicas visuales. Luego se comparo utilizando la similitud coseno, entre todas las caracteristicas de caballo vs caballo, caballo vs camión y caballo vs perro. Se graficaron (Figura \ref{fig:vgg-vs-resnet}) las frecuencias de los resultados para cada CNN. Con esto se intenta observar como se distribuyen en el espacio visual, las distintas clases. Como se esperaba la similitud entre entre animales es mas grande que con un vehiculo. Pero, se observo que para \textbf{Inception ResNet V2} existe una mayor separación entre clases, aunque sus similitudes están mas dispersas. \textbf{VGG16}, parece tener una menor dispersión, pero la similitud coseno entre distintas clases tiene valores muy cercanos. Esto puede afectar de manera negativa ya que camión y caballo no poseen una gran diferencia y el modelo podría interpretarlo como clases similares.\\

% para generar tablas de latex
\begin{figure}
	\centering
	\includegraphics[width=1\linewidth]{img/vgg-vs-resnet}
	\caption{Frecuencia de la similitud coseno de los vectores de caracteristicas visuales, ente la misma y distintas clases, para las CNN  \textbf{Inception ResNet V2} y \textbf{VGG16}.}
	\label{fig:vgg-vs-resnet}
\end{figure}

\section{Definición de métricas}
Entre los diferentes conjuntos de datos anotados utilizados por los desafíos de detección de objetos y la comunidad científica, la métrica más común utilizada para medir la precisión de las detecciones es el  \textbf{Mean Average Precision (mAP)}, seguida por \textbf{Recall}. Un problema que tienen la métricas en detección de objetos, es la fata de una implementación estándar para calcularlas. Ademas, aquellas implementaciones publicas, están muy encapsuladas al código y resulta muy difícil adaptarlo, para medir rendimientos de modelos propios. Como ya se menciono, el código de Bansal \cite{bansal2018zero} no esta disponible, por este motivo fue necesario encontrar alguna implementación de estas métricas. Se encontraron varias y luego de hacer cambios para utilizarlos, los resultados variaban mucho de un código a otro.
Fue asi que se encontró el trabajo de Padilla Rafael \cite{padilla2020survey}, que explica lo planteado:\\

\begin{center}
	 \textit{``La falta de consenso en diferentes trabajos e implementaciones de AP es un problema al que se enfrentan las comunidades académicas y científicas. Las implementaciones métricas escritas en diferentes lenguajes y plataformas computacionales generalmente se distribuyen con los conjuntos de datos correspondientes que comparten una descripción determinada del cuadro delimitador. De hecho, estos proyectos ayudan a la comunidad con las herramientas de evaluación, pero exigen trabajo adicional para adaptarse a otros conjuntos de datos y formatos de cuadro delimitador.''}\\
\end{center}

Ademas Padilla, propone una definición y un código para estandarizar las métricas, de esta manera se pueden comprar distintos modelos, de una forma ``justa''. Por estos motivos decidimos utilizar este trabajo, aunque los resultados de nuestros modelos, no sean exactos a los reportados por Bansal \cite{bansal2018zero}.\\

Ahora definamos las métricas, basándonos en el trabajo \cite{padilla2020survey}. Primero es necesario, estandarizar cuando un cuadro es:
\begin{itemize}
	\item Falso negativo (\textbf{FN}): No se obtuvo ninguna detección en absoluto, o para un cuadro delimitador verdadero el IoU $> 0.5$ y no se predijo correctamente la clase
	\item Falso positivo (\textbf{FP}): Para un cuadro delimitador verdadero, se predijo correctamente la clase pero el IoU $< 0.5$, o es un predicción duplicada, es decir, ya se marco otra con mayor IOU como \textbf{TP}.
	\item Verdadero positivo (\textbf{TP}): Para un cuadro delimitador verdadero, se obtuvo  una propuesta con un IoU $> 0.5$ y se predijo correctamente la clase.
	\item Verdadero negativo (\textbf{TN}): Esto solo tiene sentido si, se quisiera medir propuestas que no tenían un IoU $> 0.5$ con todos los cuadros verdaderos, y ademas se predijo como clase de fondo. Pero en este trabajo no es utilizada.
\end{itemize}


La \textbf{Recall}, también conocida como sensibilidad, mide la probabilidad de que los objetos verdaderos (los que se encuentran en la imagen) se detecten correctamente, viene dado por: \[Recall =\frac{TP}{FN+TP}\] El trabajo de Bansal, define Recall de la siguiente manera: 
\begin{center}
	\textit{``Un cuadro delimitador predicho se marca como verdadero positivo solo si tiene una superposición de IoU mayor que un cierto umbral t con un cuadro delimitador de verdad del terreno y no se ha asignado ningún otro cuadro delimitador de mayor confianza al mismo cuadro de verdad del terreno. De lo contrario, se marca como falso positivo.''}\\
\end{center}

Según lo que se puede interpretar, utiliza los falsos positivos para calcular la \textbf{recall} en ves de usar Falso negativo. Sin poner en tela de juicio, si esto esta bien o mal, es claro que de esta forma solo se tiene en cuenta, los objetos que tuvieron al menos una propuesta con un IoU $> 0.5$ y el resto, quedan fuera del calculo de esta métrica. Esto genera una diferencia enorme en los resultados y dificulta la tarea de comprar con otros modelos, es  por esto que en este trabajo reportamos ambas. Bansal ademas, calcula una variación denominada K@Recall, donde solo se tienen en cuentan las K mejores propuestas basándose en la confianza de la predicción y el resto son descartadas.\\

\begin{equation} 
	\label{eqn:precision}
	Precision =\frac{TP}{FP+TP}
\end{equation}

AP, es una métrica popular para evaluar la precisión de los detectores de objetos mediante la estimación del área bajo la curva (AUC) de la relación \textbf{precisión} \ref{eqn:precision} x \textbf{recall}. La curva de \textbf{precisión} x \textbf{recall} puede verse como una compensación entre ambas métricas para diferentes valores de confianza asociados a los cuadros delimitadores generados por un detector. Si la confianza de un detector es tal que su FP es bajo, la precisión será alta. Sin embargo, en este caso, se pueden pasar por alto muchos aspectos positivos, lo que produce un FN alto y, por lo tanto, una recall baja. Por el contrario, si uno acepta más positivos, el recuerdo aumentará, pero el FP también puede aumentar, disminuyendo la precisión. Sin embargo, un buen detector de objetos debe encontrar todos los objetos reales  mientras identifica solo los objetos relevantes . Por lo tanto, un detector de objetos en particular puede considerarse bueno si su precisión permanece alta a medida que aumenta su recuperación, lo que significa que si el umbral de confianza varía, la precisión y la recall seguirán siendo altas. Por lo tanto, un área alta bajo la curva (AUC) tiende a indicar tanto una alta precisión como una alta recuperación. \textbf{mAP} para la detección de objetos es el promedio del AP calculado para todas las clases. Por lo general también se indica sobre que IoU se calcula, por ejemplo, mAP@0.5, o un conjunto de umbrales como mAP@[x, y]. El trabajo de Bansal reporta \textbf{mAP}, pero no indica sobre que IoU se calcula, a si que se asume que utilizo un valor de 0,5. Muchos trabajos que utilizan COCO, reportan mAP@[.5, .95]. Esta métrica resulta muy útil si se quiere comparar rendimientos entre distinto trabajos.


\section{Detalles de metodología de evaluación}
El principal experimento consto en replicar los resultados de Bansal \cite{bansal2018zero}. No se realizo exactamente sus experimentos, ya que esto no aportaría nada nuevo. Así que, se decidió analizar sus resultados y solo replicar los que consideramos indispensable y que seria un buen punto de partida. Por ejemplo, los experimentos con clases de fondo, no obtuvieron buenos resultados, en comparación con los que no la utiliza. Es por esto que no consideramos realizaros. El principal objetivo fue obtener un modelo que obtenga resultados lo mas similar posible a los reportados. Después de varias iteraciones, no se pudo lograr, y el principal motivo es la falta de una implementación para calcular las métricas. Se utiliza el código desarrollado por \cite{padilla2020survey}, pero sin ninguna garantía de que las métricas se calculen de la misma manera.\\

Ahora definamos la metodología de evaluación. El primer paso consiste generar propuestas para cada imagen, luego cada cuadro propuesto es reescalado al tamaño de la capa de entrada que tiene la CNN, y se le extrae el vector de características visuales. Después, se utiliza el modelo entrenado para inferir el vector de características semánticas, y se calcula la similitud coseno con los vectores semánticos de todas las clases o solo las invisibles, dependiendo si se quiere evaluar ZSDG o ZSD. Aquella clase que obtenga el mayor puntaje es asignada a la propuesta. También, se guarda la puntuación como la confianza de predicción.  Por ultimo, se agrupan todas las propuestas que se tengan asignada la misma clase y se corre un algoritmo de supresión no máxima. NMS elimina las predicciones repetidas y retorna las mejores propuestas de cada grupo. Al final obtenemos como resultado un conjunto de propuestas, sus clases y su respectivo puntaje. Estos datos se guardan en un archivo y luego se corre la implementación de Padilla, para obtener los resultados de las métricas.\\




\include{Capitulos/conclusiones}

\thispagestyle{empty}

%----------------------------------------------------------------------------------------
%	BIBLIOGRAFÍA
%----------------------------------------------------------------------------------------
\backmatter
\nocite{*}
\bibliographystyle{ieeetr}
\bibliography{bibliografia.bib} % Aquí ponen el nombre del archivo .bib


\includepdf[pages=-]{firmas_del_tribunal.pdf}

\end{document}